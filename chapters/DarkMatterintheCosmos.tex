%-----------------------------------------------------------------------------------%
\section{Introduction\label{sec:intro2dm}}
%-----------------------------------------------------------------------------------%

I'll attempt to explain the dark matter problem at an entry level with the following thought experiment.
Let's say you're the teacher for an elementary school classroom.
You take them on a field trip to your local science museum and among exhibits is one for mass and weight.
You see the gigantic scale and come up with a fun problem.

You say to your class, "What is the weight of the classroom?
Give me your guess to me in 30 minutes and then we'll check on the scale.
If your guess is within 10\% of the right answer, we will have a movie day tomorrow with a movie of your choice."

The kids are ecstactic to hear this, and they get to work.
The solution is straight forward.
The students should give each other their weight or best guess if they dont know.
Then all they have to do is add each students' weight and get a grand total for the class.

Alice and Bob return to you with a solution. They say, "We weren't sure of everyone's weight.
We used 65 lbs for the people we didn't know and added everyone who does know.
There's 30 of us, and we got 2000 lbs!
That a ton!"

You estimated 1900 lbs assuming the average weight of a student in your class was \textasciitilde 60 lbs.
So you're pleased with this answer.
You instruct your students to all gather on the giant scale and read off the weight together.
To all of your surprise, the scale reads \textit{10 thousand lbs}!
This is way more than a 10\% error.
In fact this is 5 times more massive than either your or your students' estimates.
You think to yourself and conclude there must be something wrong with the scale.
You ask an employee to check the scale and verify it is calibrated well.
They confirm that the scale is in fact in working order.
You then weigh a couple students individually, maybe the scale is not properly calibrated.
Alice weighs 59 lbs, and Bob weighs 62 lbs, typical weights for their age.
You then weigh each student individually and see that their weights individually do not deviate greatly from 60 lbs.
So where does all the extra weight come from?

This is in essence the Dark Matter problem.
The important substituion to make however is to replace the students with stars and classroom with a galaxy, say the Milky Way.
Individually the mass of stars is well measured and defined with the Sun as our nearest test case.
However, when we set out to measure a collection of stars as large as galaxies, our well motivated estimation is wildly incorrect.
There simply is not way to account for this discrepancy except without some unseen, or dark, contribution to mass and matter in galaxies.
I set out in my thesis to narrow the possibilities of what this Dark Matter can be.

This chapter is organized like the following\dots
\todo{Text should look like ... Chaper x has blah blah blah.}

%-----------------------------------------------------------------------------------%
\section{Dark Matter Basics\label{sec:basicDM}}
%-----------------------------------------------------------------------------------%

Dark Matter (DM) has been a whispering problem in physics for almost 100 years.
Anomolies have been detected by way of weird galaxy behaviour, budding Cosmology, and more \ns.
It was sometime in 1930's when the super duper smart Zwicky measured that it was defintely there.
It's kind of a big deal because we have no idea what the nature of this stuff and there's a lot of it.
According to Lambda CDM, the most legit model, \ns DM is about 85\% \fu, of all mass in the universe.
It's called dark in fact because we cannot see it. \ns
Finding out what the hell it is, is an active field of research and hopefully it interacts with the standard model.

Here's what we do know about DM so far\dots
DM is dark, it doesn't interact readily with light.
DM also doesn't interact noticably with the other standard model forces (EM, Strong, Weak) at a rate that matters \ns.
DM is cold.
By cold I mean that it is most likely not moving at relativisic speeds like neutrinos and photons. \ns
If it was moving that fast, the structures we see like galaxies would be much more diffuse than what is observed. \ns
DM is old.
DM played a critical role in the formation of the universe and the structure within it. \ns
We know this from Cosmology and computer universe simulations \ns.

The search for DM is basically summarized by trying a bunch of different models and performing measurements of all kinds to test them.
These models of course have to nominally agree with the known observations seen over the last century.
Whenever we perform a test and don't see anything, the parameter spaces gets more contrained.
I discuss some of the ideas ad approaches further on.
I Especially discuss the models that are relavent to my thesis.

We forunately have the largest volume and lifetime ever for a particle physics experiment in the universe.
This means we can do some pretty cool shit very efficiently.
The drawn back are the backgrounds.

%-----------------------------------------------------------------------------------%
\section{Evidence for Dark Matter\label{sec:evidence4dm}}
%-----------------------------------------------------------------------------------%

Let me show you why we're pretty sure DM is a thing and why it might be particle like in nature.
My thesis focuses on WIMP dark matter which is one of the better motivated things out there
There were some weird as fuck anomolies early in the last century but we weren't 100\% that it was legit.
Then some great scientsist made some keen measurements of stars and their minds were blown.
Read more to see what we know now.
I promise you're about to get mind fucked.

%$$$$$$$$$$$$$$$$$$$$$$$$$$$$$$$$$$$$$$$$$$$$$$$$$$$$$$$$$$$$$$$$$$$$$$$$$$$$$$$$$$$%
\subsection{First Clues: Stellar Velocities\label{sec:ev4dm_stars}}
%$$$$$$$$$$$$$$$$$$$$$$$$$$$$$$$$$$$$$$$$$$$$$$$$$$$$$$$$$$$$$$$$$$$$$$$$$$$$$$$$$$$%
Ok so someone \fu \ns started taking measurments with at.
They were curious about what speed stars were orbitting the galaxies they were contained in.
These measurements were done for things close by.
At the time we were even that sure galaxies were a thing.
Bu with the basical knowlwedge we had we used the virial theorem with the velocities of the stars to measure the mass inderectly of the galaxies.

\eqin{The Virial Eqn}

\todo{explain the virial equation}
\ns you probably want to source the theory behind why this important

The verdict wasnt clear however until Vera Rubin made some awesome discoveries with more precise equipment and 21cm lines of Hydrogren gas in the galaxies.
This really showed that there was some unexplained discrepancy between how much mass we were seeing in the stars and the mass measured indirectly.
The issue is that it we're pretty sure now that we're not just under-estimating the mass of the stars \ns.
The difference in mass was up to 5x which is way way too much for what our uncertainties were (somewhere around 20\%)\ns.

\tmpfig{velocity dispersion old here.}

Nowadays we have more measurements of the stellar velocities and have even discovered small DM dense bodies called dwarf spheroidals (dSph)
These measurements have been made by the community \fu and there are compiled lists of how much DM these objects have.
Most of these measurements are made from newtonian virial theorem measurments.
There has since emerged new evidence.
These innovative techs are discussed in the following sections. The evidence cullminates into a story of particle dark matter.

%$$$$$$$$$$$$$$$$$$$$$$$$$$$$$$$$$$$$$$$$$$$$$$$$$$$$$$$$$$$$$$$$$$$$$$$$$$$$$$$$$$$%
\subsection{Mounting Evidence for Dark Matter\label{secc:ev4dm_more}}
%$$$$$$$$$$$$$$$$$$$$$$$$$$$$$$$$$$$$$$$$$$$$$$$$$$$$$$$$$$$$$$$$$$$$$$$$$$$$$$$$$$$%
Modern evidence for dark matter comes from new avenues.
We got microlensing which supports DM in the general relativity sector.
The Cosmic Microwave Background shows that the universe has DM in it from a very early stage.
The CMB is the primordial light from the young universe.
Basically a baby photo.
Then we have computational models where we model the universe.
Then we look at how the simulated universes look like compared to what we see.
From those simulations we infer how much dark matter is in the universe.
The fuller explinations and shortcoming of each of these methods is explained further in this section.

someone took a an observation of the bullet cluster.
The microlensing of galaxy clusters are some of the most damning evidence that DM is actually matter and not just a flaw in our gravitational theories.
There were two galaxy clusters \fu.
They clearly passed through each other at some point in the past and are in the process of merging \ns.
Two observations of the clusters were made independantly of each other.
The first was the microlensing of light around the galaxies due to their gravitational influences.
When celestial bodies are large enough, the gravity they exert bends space and time itself.
This bending effects light and will deflect light in a smilar way to how lenses will bend light.

\tmpfig{gravitational lensing figure compared to glass lensing.}

With a sufficient understanding of light sources behind a celestial body, you can reconstruct the countours of the gravitational lenses.
The gradient of the contours then tells you how dense the matter is and where it is.

They then made measurements of the x-ray emmision from the clusters.
The idea is that since these galaxies are mostly gass and are merging, then they should be getting hotter.
If they're merging, the x-ray emmisions should be the strongest where the gas is mostly moving through each other.
The x-rays basically map out where the gas is in these merging galaxies.

\tmpfig{bullet cluster photo.}

The dope super interesting thing is that the map of the x-ray emmisions totally doesnt align with the gravitational countours from the microlensing.
This incongruence is really telling that there is a lot of matter somewhere that we jsut cannot see.
Moreover this matter is NOT BARYONIC.
So then what is it?
This measurement didn't really tell us what exactly, but it did suggest that this DM also doesn't interact with itself very strongly.
If it did, then it would have been more aligned with where the x-ray emmision was.
There's been other studies of galaxies with similar results altho there are a handful that resemble something we expect for strongly self-interacting DM. \ns.
This result really makes it hard to argue that DM is somehow something amiss in our gravitational theories.

we got the CMB and geometry of the universe.
So there's this thing called the cosmic Microwave Background (CMB).
It's the universes baby photo from when all of the hydrogen de-ionized to form atoms.
This happened cause it was cold enough finally from the expansion of the universe.
The recombination happened someitme around less than 1 mil years after the universe was born \fu \ns.
when hydrogen absorbs an electron, it releases a photon of a specific wavelength.
This wavelength amounts to 13 ev or so according to the qm eqn\dots

\eqin{hydrogen energy level}

However the universe has been expnding since it's creation.
In fact the time and space itself is exanding away from us for as long as the universe is old.
This red-shifts the combination light into the Microwave frequencies.
This is the light we can detect with microwave observatories and is what was first detected by so and so in the 19?? \ns \fu
This make a microwave image seen below after we subtract the average of the image.

\tmpfig{CMB photo}

We can do a funny thing with the photo but it's fairly straight forward.
Shove the photo into a spherical harmonic decomposition.
This gives you the vibrational modes of the CMB and therefore the early universe.
The important thing to note is that the harmoincs are based on primordial baryonic acoustic oscillations \fu
This is directly linked with the energy density of the universe and how these couple.
It's a cosmology and geometry thing.

\tmpfig{Planl harmonics of CMB}

The harmnics would look very different for a universe with less dmm (see fig bla) or a lot more dm (see fig bla)

\tmpfig{Plank harmonics vs DM content CMB}

The observations fit well with the Lambda CDM model and we derive the primordial dm concentration to be XX\% and primordial DM to be XX\%.
\todo{What are the shortcomings?} I think the most obcious arguement is simply that this is very old light, up to 13.6 billion years old.
It's not at all necessary that the universe shares the exact same DM, matter ratio.
There is a poorness in fit in the lower region of the graph and this is unexplained.
The way we measure distance can be really fucked sometimes so maybe that's a problem too.

Finally we have universe simulations like the millenium simultation and more \fu \ns.
These are computer simulations of the unverse with different fractions of DM and baryonic matters.
Additionaly hypotheses are tested like how hot the DM is and how strongly it interacts with itself and with baryonic matter.
These simulations are also done for smaller scales like galactic formation and galaxy clustering.
In alls cases the simulations most resemble out universe for a Lambda CDM like universe.

The main issues with the similations is mostly that we cant perfectly simulate the unverse.
They are often imcomplete with how they treat baryonic matter and make big assumptions about dark matter.
These simulations also have to contend with very real computational limitations.
The resultion of some of the universe simulations are as large at XX's of solar masses.
There's reason to beleive that the resultion might really matter as well. \ns \fu


Overall this forms a compelling arguement for dark matter.
However, these observations really only confirm that DM is there.
It takes another leap of theory to make observations of DM that are nongravitational.
One of which is the emergence of the Weakly Interacting Massive Particle hypothesis of DM.
This DM candidate theory is discussed futher in the next section.

%-----------------------------------------------------------------------------------%
% \section{The WIMP Miracle\label{sec:dm_wimps}}
%-----------------------------------------------------------------------------------%

%-----------------------------------------------------------------------------------%
\section{Searching for Dark Matter\label{dm_search}}
%-----------------------------------------------------------------------------------%
We've explored any options for what dark matter could be now.
The remainder of this thesis I will focus only on a particle dark matter hypothesis.
I will not be discussin alternative gravitational theories such as Modified Newtonian Dynamics.
I am also ignoring composite dark matter discussion like primordial black holes, dark atoms, or dark bound states of baryonic matter.
For this thesis I focus on the hypothesis that DM is a weakly interacting and massive particle (WIMP).

\tmpfig{Standard model. Square or Circle?}

The current status of the standard model does not have a WIMP candidate.
When looking at the standard model, we can immediately exclude any charged particle.
This is because charged particles interact with light and so much DM would be immediately visible if is had the same charge as SM particles.
Specifically this will rule out the following charged, fundamental particles: $e,\mu, \tau, W, u, d, s, c, t, b$ and their corresponding antiparticles.
Recalling from earlier that DM must be long lived and stable over the age of the universe.
This would exclude all SM particles with decay half-lives at or shorter than the age of the universe.
This constraint eliminates the $Z, $and $H$ bosons.
Finally, the candidate DM needs to be somewhat massive.
This follows from the DM needing to be cold or not relativistic through the universe.
This eliminates the remaining SM particles: $\nu_{e, \mu, \tau}, g, \gamma$.
This indicates the SM that is likely not the full story and hints to physics beyond the stadard model (BSM).


%$$$$$$$$$$$$$$$$$$$$$$$$$$$$$$$$$$$$$$$$$$$$$$$$$$$$$$$$$$$$$$$$$$$$$$$$$$$$$$$$$$$%
\subsection{Shake it, Break it, Make it\label{sec:bop_it}}
%$$$$$$$$$$$$$$$$$$$$$$$$$$$$$$$$$$$$$$$$$$$$$$$$$$$$$$$$$$$$$$$$$$$$$$$$$$$$$$$$$$$%

\tmpfig{Shake it, break it, make it}

The above figure demonstrates the different interaction modes possible with particle DM and the DM.
The figure is a simplified Feynman diagram where the arrow of time represents the interaction modes of: \textbf{Shake it, Break it, Make it}.

\textbf{Shake it} refers to the direct detection of dark matter.
Direct detection interactions start with a free DM particle and some SM particle.
The DM and SM interact under some elastic or inelastic collision and recoil away from each other.
The DM remains in the dark sector and imparts some momentum onto the SM particle.
The hope is that the momentum imparted onto the SM particle is sufficiently high enough to ick up with highly sensitive instruments.
Because we cannot create the DM in the lab, we have to wait until it is incident on the detector.
We do this by increasing the interaction volume of the detector with some inert chemical.
We then leverage the hypothesis that the DM is everywhere around us and Earth's motion through the cosmos creates a sort of DM wind.
Direct detectors are live now and taking data.
Some active experiments include XENON \todo{look up and name direct DM experiments.}

\tmpfig{windy dark matter. Look at Jodi's DM lectures}

\textbf{Make it} refers to the production of DM from SM initial states.
The experiment starts with particles in the SM.
These SM particles are accelerated to incredibly high energies and then collided with each other.
In the confluence of energy DM emerges as a byproduct of the SM annilation.
Often it is the collider experiments that are able to generate energies high enough to probe DM.
These experiments include the renown ATLAS and CMS collaborations at CERN where protons are collided together at exterme energies.
The DM searches however are complex.
DM likely does not interact with the detectors and lives long enough to escape the detection apparati of CERN's colliders.
This means any DM search with production searches for an excess of events with missing energy in the events.
The missing energy with no particle tracks implies a neutral particle carried the energy out of the detector.
However, there are other neutral particles in the SM and so any analysis have to discriminate between SM signatures of missing energy and a potential DM candidate.

\tmpfig{A particle event in CMS/ATLAS with Missing E}

%$$$$$$$$$$$$$$$$$$$$$$$$$$$$$$$$$$$$$$$$$$$$$$$$$$$$$$$$$$$$$$$$$$$$$$$$$$$$$$$$$$$%
\subsection{Break it: Standard Model Signatures of Indirect Dark Matter Searches\label{sec:break_it}}
%$$$$$$$$$$$$$$$$$$$$$$$$$$$$$$$$$$$$$$$$$$$$$$$$$$$$$$$$$$$$$$$$$$$$$$$$$$$$$$$$$$$%

\textbf{Break it} refers to the creation of SM particles from the dark sector, and it is the primary concern of this thesis.
The interaction begins with dark matter or in the dark sector.
The hypothesis is that this DM will either annihilate with itself or decay and produce a SM byproduct which we can detect.
This method is often refered to the Indirect detection of DM because we have no lab to directly control or manipulate the DM.
Therefor most DM primary observations will be performed from observations of known DM densities among the cosmos.
The strength is that we have the entireity of the universe and it's lifespan to use as the detector or particle accelerator.
Additionally, locations of dark matter are also well understood since it was astrophysical observations that presented the problem of DM in the first place.

However, anything can happen in the universe.
So there are many difficult to deconvolve backgrounds when searching for a DM signal.
Once prominant example is the galactic center.
There's a lot of DM there since the Milky Way definitely has a lot of DM.
But any signal coming from there is hard to parse apart from the extreme environment of our supermassive black hole, Sagitatrius A*
In fact, there has been known \textgamma -ray excesses from the galactic center \ns, yet the environment presents a difficult problem in sussing out what the fuck is actually going on.
Despite the challenges, any DM model that yields evidence in the other observation two methods, \textbf{Shake it or Make it} must be corraborated with indirect observations of the known DM overdensities.
Without corroborating Evidence, DM observation in the lab is hard pressed to demonstrate that it is the model contributing to the DM seen at the universal scale.

In the case of WIMP DM, signals are typically described in terms of primary SM particles produced from a DM decay or annihilation.
These particles are then simulated to stable final states such as: $\gamma, \nu, p, \text{or } e$ which can traverse galactic lengths to reach the earth.

\tmpfig{particle cascade from DM}

The figure shows the quagmire of SM particles that emerges from SM initial states that are not stable.
There's a lot of different things with different energies that can pop out.

For any neutral messenger, the DM flux from DM annihilating to some particle in the SM, \textphi, from a region in the sky is

\eqin{DM ann flux equation}

\todo{explain the equation}
And for decay it is\dots

\eqin{DM decay flux eq}

\todo{explain the equation}

The integral over a line of sight is a simplification made because we mostly observe a 2d surface with our Astrophysics experiments.
This also translates the equation into observables in our detector like solid angle.
The spectral shape is mostly determined by the SM primary products.
From HDMSpectra, they look like the following figures for the bb, tau, and Z spectra.

\tmpfig{HDMSpectra: bb, tautau, WW}

Additionaly, when DM primarily goes into one of the neutral messengers (nu or gamma), the spectra will typically have a line feature.
These messengers are very unlikely to be attentuated in any way from their primary state.
These line spectra are usually considered smoking gun signals as their energy will be half the COM of the DM -> SM process.
For DM in the GeV+ scale, there is no similar SM process and so seeing the signal would almost certainly be an indication of the presence of dark matter.

\tmpfig{Line spectra, nu and gamma}

%-----------------------------------------------------------------------------------%
\section{Multi-Messenger Dark Matter}
%-----------------------------------------------------------------------------------%

Astrophysics entered a dope as fuck new phase in the past few decades that leverages our new knowlwedge of the SM and general relativity.
Up until the 21st century, astrophysical observatations were done with photons.
At first, observations were optical in nature.
You can confirm this yourself by going outside at night.
The moon and constellations are observabke to the naked eye.
In darker places on Earth, celestial bodies like our Milky Way galaxy become visible.
Novel observations of the universe have since only adjusted the sensitivity of the wavelength of light that's observed.
Gems like the CMB, MEERkat, \ns and more have ultimately been observations of different wavelengths of light.
Light can also be thought of as a particle in the SM is referred to as a photon, or a packet of light.

\tmpfig{multimessenger sectors from the NSF}

Come the 21st century and we've started to use more of the SM and general relativity.
The expirements LIGO and VIRGO had an iconic dicovery in 2015??\fu with the first chirps of black hole mergers.
This opened an entirely new method of observing the universe through gravitational waves.
They litterally use the bending of space-time to do astrophysics like holy shit.
There's also been a surge of interested in the neutrino sector.
We're now finally having some sensitivity to neutrinos that we're able to detect them from astrophysical sources.
Neutrinos, like gravitational waves and light, travels mostly unimpeded from their source to our observatories.
This makes pointing to the oringinating source of the these messengers much each than it is for cosmic rays that are almost always deflected from their source.

\tmpfig{Milky way at different wavelengths}

Being able to see the same objects under different regimes was demonstrated already with just photons.
From the previous figure you can see different ways to look at the milky way galaxy.
Each panel corresponds do a different wavelength of light which has different penetrations through gas and galactic dust.
Some sources are more apparent in some panels, while others are not.
Recently, the IceCube collaboration published a groundbreaking result of the milky way in neutrinos.
This new channel is very unique because we can really see through the galaxy.
This new image also refines our understanding of how high energy particles are accelerated since the fit to IceCube data prefers one standard model process over the other.

Exposing our observations to more cosmic messengers greatly increases our sensitivity to rare processes.
In the case of DM, from fig (SM ann), you can see there are many SM particles at the end of the particle cascade.
Among the final states are gammas and neutrinos.
The charged particles however would not likely make it to earth since they'll be deflected.
This means observatories that can see the neutral messengers are especially good for DM searches and for combining data for a multi-messenger search.

% Observatories exists for the neutral messengers like HAWC, gammas, and IceCube, neutrinos.
% Each of these observatories already independatly search for DM \ns.

% \tmpfig{Sensitivities for HAWC and IC with IC?}

%-----------------------------------------------------------------------------------%
\section{Search Targets for Dark Matter\label{sec:dm_targets}}
%-----------------------------------------------------------------------------------%
We of course have to know where to look.
Thankfully, we have a good idea of where.
Out first detection of DM relied on optical observations.
Since then, we've developed new techniques to find large DM dense regions.
We first found out about DM through observing galactic rotation curves.
This includes our nearest galaxy, the Milky Way.
The Milky Way thus is the largest nearby DM dense region to look at.
Additionally, the DM halo surrounding the Milky Way is somewhat clumpy \ns.
There are regions in the DM halo of the Milky Way that have more DM than others and it's captured gas over time.
In some cases these sub-haloes were dense enough to creat stars.
These apparent sub galaxies are known was dwarf spheroidal galaxies and are the main sources studied in this thesis.

%$$$$$$$$$$$$$$$$$$$$$$$$$$$$$$$$$$$$$$$$$$$$$$$$$$$$$$$$$$$$$$$$$$$$$$$$$$$$$$$$$$$%
\subsection{Dwarf Spheroidal Galaxies\label{sec:dSphs}}
%$$$$$$$$$$$$$$$$$$$$$$$$$$$$$$$$$$$$$$$$$$$$$$$$$$$$$$$$$$$$$$$$$$$$$$$$$$$$$$$$$$$%

The way we look for dwarf spheroidal galaxies (dSph's) is through mostly Newtonian physics.
We use either the virial theorem to determine the DM density of the dSph's or a Jeans analysis /ns.
DSphs tend to be ideal sources to look at for DM searches.
The reason is that these environments are fairly quiet.
Unlike the galactic center, the most active components of dSph's are the stars within them.
There are few compact objects, like black holes, and much less gas that would contribute to a large backgrounds.
The DM to mass ratio here is also massive. \ns.
The signal to background ratio is really large and we expect a lot of signal from how much dark matter there is.
All this together means that dSph's are among the best sources to look at for indirect DM searches.