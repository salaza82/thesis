%-----------------------------------------------------------------------------------%
\section{The Detector}\label{sec:THE_hawc}
%-----------------------------------------------------------------------------------%

\begin{figure}[h!]
    \centering{
    \includegraphics[scale=0.155]{figures/hawc/HAWC.jpg}
    }
    \caption{Photo of the HAWC detector that I took on May 17, 2023. Main array is centered in the photo and comprised of the larger tanks. Outriggers are the smaller tanks around the main array.}
    \label{fig:HAWC}
\end{figure}

The High Altitude Water Cherenkov (HAWC) Observatory is a specialized instrument designed for the observation of high energy gamma-rays and cosmic rays \cite{HAWC_NIM}.
Located on the Sierra Negra volcano in Mexico, HAWC observes gamma rays and cosmic rays in the energy range of approximately 100 GeV to 100'ss of TeV.
HAWC is strategically situated to maximize observational efficiency due to its high altitude.
At an elevation of 4,100 meters, it monitors about two-thirds of the sky every day with an uptime above 90\%.
This capability is essential for studying high-energy astronomical phenomena.

HAWC comprises of 300 water Cherenkov detectors (WCDs) spread over 22,000 square meters.
Each main array detector is filled with purified water and equipped with four, upward-facing photomultiplier tubes (PMTs).
These PMTs detect Cherenkov radiation from charged particles passing through the tanks.
These charged particles are generated when a high energy gamma or cosmic ray collides with gas in the atmosphere to create a charged particle shower, see \cref{fig:airshowers}.
The observatory includes a separate tank configuration which are refered to as the outriggers.
They are a secondary array of 345 smaller WCD's.
Surrounding the main array, each outrigger tank measures 1.55 meters in diameter and height and contain a single upward-facing eight-inch PMT.
This expansion increases the instrumented footprint fourfold.
It improves the reconstruction of showers extending beyond the main array, especially for events above 10 TeV.
However, at the time of writing this thesis, the outriggers have not been fully integrated into HAWC's reconstruction software.

\begin{figure}
    \centering{
    \includegraphics[scale=0.5]{figures/hawc/high_energy_air_shower.png}
    }
    \caption{A particle physics illustration of high energy particle showers. Left shower is an electromagnetic shower from a high energy gamma-ray. Most particles in the shower will be a combination of photons and charged leptons, in this case electrons (e). Right figure shows a cosmic ray particle shower. The cosmic ray will produce many more types of particles including pions ($\pi$), neutrinos, and charged leptons. Figured pulled from \cite{lopez_thesis}.}
    \label{fig:airshowers}
\end{figure}

%-----------------------------------------------------------------------------------%
\subsection{Construction and Hardware} \label{sec:hawc_hardware}
%-----------------------------------------------------------------------------------%

\begin{figure}
    \centering{
        \includegraphics[scale=0.14]{figures/hawc/WCDs.jpg}
        \includegraphics[scale=0.45]{figures/hawc/WCD_schematic.png}
    }
    \caption{The WCDs. Left image features several WCDs looking from within the main array of HAWC. Right image shows a schematic of a WCD pulled from \cite{HAWC_NIM}.}
    \label{fig:WCD_schematic}
\end{figure}
\todo{fact check the content below. GPT may have hallucinated}
Each main array WCD is a cylindrical tank with dimensions of 7.3 m in diameter and 5.4 m in height and filled with 180,000 liters of water \cite{HAWC_NIM}.
The metal shell of these tanks is made from bolted together, corrugated, galvanized steel panels.
The tanks are placed into 0.6 m deep trenches filled with rammed earth to secure it against seismic activity.
The interior of each tank is lined with a black, low-density polyethylene bladder, designed to be impermeable to external light and to prevent reflection of Cherenkov light within the tank.
This bladder is approximately 0.4 mm thick and composed of two layers of three-substrate film.
To further minimize light penetration, a black agricultural foil covers the bladder.
The ground and walls inside the tank are protected with felt and sand to safeguard against punctures.
The tanks are filled 4.5 m deep of purified water, achieving a photon attenuation length for Cherenkov photons that exceeds the tank's dimensions.
This purification level ensures the optimal detection environment for the photons generated by traversing charged particles.

At the base of each tank, four photomultiplier tubes (PMTs) are installed to detect the Cherenkov radiation emitted by charged particles.
Three 8-inch diameter PMTs surround a larger 10 inch PMT from Hamamatsu \cite{hawc_pmt}.
The variation in PMT response is carefully accounted for in event reconstruction algorithms.
Signals from the PMTs travel  610 ft cables to the counting house, where they are processed by Front-End Boards (FEBs).
These FEBs, along with Time to Digital Converters (TDCs), digitize the signals and manage the high voltage supply to the PMTs.

%-----------------------------------------------------------------------------------%
\subsection{Data Acquisition and Signal Processing} \label{sec:hawc_daq}
%-----------------------------------------------------------------------------------%

\begin{figure}
    \centering{
        \includegraphics[scale=0.5]{figures/hawc/tank_basic_schem.png}
    }
    \caption{\todo{copied from nim}. Top-level diagram of the HAWC electronics showing a summary of the critical subsystems and the interconnections, including HV and optical fiber cabling. NMEA refers to the National Marine Electric Association format in which GPS presents data [66,67]; CLR, TRG and RST are control signals for the TDC system. The LoToT andHiToT time over threshold signals are discussed in Section 4.1}
    \label{fig:basic_tanks_schem}
\end{figure}

\begin{figure}
    \centering{
        \includegraphics[scale=0.6]{figures/hawc/digital_schematic.png}
    }
    \caption{\todo{copied from NIM}. Schematic overview [68] of the HAWC data acquisition and online processing system, as described in the text of Section 4}
    \label{fig:dig_schem}
\end{figure}

The HAWC data acquisition (DAQ) and signal processing systems convert the physical detection of particles into analyzable data.
This process involves a series of steps from initial signal detection by PMTs to digital conversion and preliminary analysis, see \cref{fig:basic_tanks_schem} and \cref{fig:dig_schem}.

\begin{figure}
    \centering{
        \includegraphics[scale=0.6]{figures/hawc/ToT_threshold.png}
    }
    \caption{\todo{text copied from nim}. The analog PMT signals are split and passed through two paths. In each path, there is an amplifier and discriminator circuit. The ratio of the amplifier gains is 7 to 1. The higher gain circuit has an effectively lower threshold (Low Th). There is a time (T) delay in the high threshold (High Th) path. The 2-edge event is related with the Low Th, while the 4 edge event is related to the High Th.}
    \label{fig:tot_threholds}
\end{figure}

Once the signal from the PMTs arrive at the counting house, they enter the Front-End Boards (FEBs).
The FEBs are responsible for the initial processing of these signals, which includes amplification and integration \cite{Milagro_DAQ}.
Each PMT signal is compared against preset LOW/HIGH voltage thresholds in the FEBs \cref{fig:tot_threholds}, identifying signals that correspond to about 1/4 and 4 photoelectrons, respectively.
This differentiation allows the system to gauge the strength of the detected Cherenkov radiation.
The processed signals are then digitized by Time to Digital Converters (TDCs).
These converters measure the time over threshold (ToT) for each signal, a parameter that reflects both the duration and amplitude of the signal.
This digitization facilitates reconstruction of the original event for translating the physical interactions within the detectors into data \cite{nim:hawc_detect,HAWC_DAQ_NIM,Milagro_DAQ}.

Synchronization across the HAWC observatory is maintained by a central GPS Timing and Control (GTC) system, which achieves a timing resolution of 98 ps.
This high-resolution timing is vital for accurately reconstructing the timing and location of air showers initiated by cosmic and gamma rays.
The GTC system ensures that all components of the DAQ operate in unison to preserve the temporal integrity of the detected events \cite{nim:hawc_detect,hawc_daq_thesis}.

Once digitized, the data are transferred to an online event reconstruction system.
This system runs the Reconstruction Client, which utilizes the raw PMT data to reconstruct the characteristics of the air showers, such as their direction and energy \cite{HAWC_DAQ_NIM}.
The capacity for real-time analysis allows HAWC to promptly respond to astrophysical phenomena like Gamma Ray Bursts (GRBs) and to participate in multi-messenger astronomy by following up on alerts from other observatories.
This real-time processing system is designed to handle high data throughput, using ZeroMQ \cite{zeromq} for efficient data transfer between software components.
Analysis Clients perform specific online analyses that require immediate data, including monitoring for GRBs, solar flare activity, and participation in global efforts to track gravitational waves and neutrinos \cite{nim:hawc_detect}.

The DAQ system is overseen by an Experiment Control system and crew that manage the operational aspects of data collection.
This includes initiating and terminating data collection runs and monitoring the experiment for errors.
In the event of a system crash, often caused by environmental factors such as lightning, the Experiment Control system is designed to automatically restart the experiment and minimize downtime \cite{nim:hawc_detect,HAWC_DAQ_NIM}.

%-----------------------------------------------------------------------------------%
\section{Event Reconstruction} \label{sec:hawc_reconstruction}
%-----------------------------------------------------------------------------------%

Event reconstruction at the HAWC Observatory is a critical procedure that converts the raw data from the observatory's WCDs into a coherent framework for understanding cosmic and gamma-ray events.
This process includes several distinct steps.
Core Fitting determines the geometric center of the air shower on the detector plane.
Angle Reconstruction assesses the trajectory of the incoming particle, revealing its origin in the sky.
Energy Estimation is performed using both \textit{f}-hit and Neural Network (NN) methods to quantify the energy of the detected events.
Gamma-hadron (G\H) discrimination differentiates between gamma-ray and hadronic cosmic ray initiated showers, a vital step for astrophysical interpretations.
Each of these steps is integral to the observatory's objective of investigating the high-energy universe and enable the transformation of signals into detailed insights about high energy cosmic phenomena.

%$$$$$$$$$$$$$$$$$$$$$$$$$$$$$$$$$$$$$$$$$$$$$$$$$$$$$$$$$$$$$$$$$$$$$$$$$$$$$$$$$$$%
\subsection{Core Fitting} \label{sec:hawc_core_fitting}
%$$$$$$$$$$$$$$$$$$$$$$$$$$$$$$$$$$$$$$$$$$$$$$$$$$$$$$$$$$$$$$$$$$$$$$$$$$$$$$$$$$$%

\begin{figure}[h!]
    \centering{
    \includegraphics[scale=0.5]{figures/hawc/shower_shape.png}
    }
    \caption{\todo{copied from A's thesis}. An illustration of the angle reconstruction of the original particle. The secondary particles of an air shower travel in a plane perpendicular to the direction of the original particle, allowing for the reconstruction of the initial angle after corrections due to the curvature of the plane. Figure from \cite{thesis_Zigg}.}
    \label{fig:shower_shape}
\end{figure}

In the study of air showers, accurately determining the location of the air shower core on the ground is crucial for reconstructing the direction of the originating primary particle.
An illustration of this can be seen in a HAWC event plot, where the lateral charge distribution across the array is displayed.
The core is identified and marked with a red star, reconstructed using a predetermined functional form.

The signal $S_i$ from the \textit{i}th PMT is given by the following equation:
\showercore
In this model, $\tilde{x}$ represents the core location and $\tilde{x}_i$ is the position of the \textit{i}th PMT.
$R_m$ stands for the Molière radius, which is approximately 120 meters at the altitude of HAWC, while $\sigma$, is the standard deviation of the Gaussian distribution.
The equation incorporates fixed values of $\sigma = 10$ m and $N=5.10^{-5}$.
$N$ is the normalization factor for the tail of the distribution.
This leaves the core location and overall amplitude $A$ as the free parameters to be determined during fitting.

\begin{figure}
    \centering{
        \includegraphics[scale=0.5]{figures/hawc/core_fitting.png}
    }
    \caption{\todo{pulled for thesis}. Charge deposited in each PMT for a reconstructed gamma-ray event. Each large circle represents a WCD and each of the 4 smaller circles within represent a PMT. The color scale represents the amount of charge deposited in each PMT. The red star in the center of the dashed circle shows the location of the shower core fit by the SFCF algorithm. \cite{Abeysekara_2017}}
    \label{fig:core_fitter}
\end{figure}

The chosen functional form for the Super Fast Core Fit (SFCF) algorithm is a simplified version of a modified Nishimura-Kamata-Greisen (NKG) function \cite{cosmic_ray_shape}, selected for its computational efficiency which is essential for rapid fitting of air shower cores.
The SFCF form allows numerical minimization to converge more quickly due to the function's simplicity, the analytical computation of its derivatives, and the absence of a pole at the core location.
Figure 2 provides a visualization of a recorded event, with the plot depicting the charge recorded by each PMT as a function of the distance to the reconstructed shower core.
Through the application of the SFCF, core locations can be identified with a median error of approximately 2 meters for large events and about 4 meters for smaller ones, assuming the gamma-ray event core impacts directly upon the HAWC detector array \cite{Abeysekara_2017}.
It is noted that as the core's distance from the array increases, the precision in locating the core diminishes, highlighting the importance of proximity in the accuracy of core reconstruction.


%$$$$$$$$$$$$$$$$$$$$$$$$$$$$$$$$$$$$$$$$$$$$$$$$$$$$$$$$$$$$$$$$$$$$$$$$$$$$$$$$$$$%
\subsection{Angle Reconstruction} \label{sec:hawc_angleReco}
%$$$$$$$$$$$$$$$$$$$$$$$$$$$$$$$$$$$$$$$$$$$$$$$$$$$$$$$$$$$$$$$$$$$$$$$$$$$$$$$$$$$%

After establishing the core position, the next step is angle reconstruction.
This process determines the primary particle's trajectory.
The angle of arrival is indicative of the originating gamma ray's direction.
It correlates to the cosmic source of the gamma-ray.
We deduce this angle using the timing of PMT hits \cite{Abeysekara_2017}.

The air shower's front is conically shaped, not flat.
This shape arises from the travel patterns of secondary particles.
An event example is illustrated in \cref{fig:shower_shape}.
Far from the core, secondary particles undergo multiple scattering.
They also travel longer distances \cite{wcd_Sensitivity}.
Particle sampling decreases with distance from the core.
This decrease results in measurable delays in arrival times \cite{wcd_Sensitivity,Abeysekara_2017}.
Simulations provide a corrective measure for these effects.
The correction is a function of shower parameters \cite{Abeysekara_2017}.
It adjusts both curvature and sampling.
The distance from the shower core and the charge recorded by PMTs are crucial to this correction.
A function based on simulation and Crab Nebula observations is used for this purpose \cite{Abeysekara_2017}.
This correction is essential for accurate reconstruction.

Corrections lead to the χ2 minimization step.
This technique fits a plane to the timing data of the PMTs.
It then calculates the shower's angle of arrival.
The zenith and azimuth angles are the result of this fitting \cite{wcd_Sensitivity}.
The local angles are converted to celestial coordinates.
These coordinates allow correlation with gamma-ray sources.
Right ascension (RA) and declination (Dec) are used for this purpose.
RA is akin to longitude, and Dec to latitude.

The reconstructed angle's resolution ranges from 0.1° to 1°.
This range depends on the incoming particle's energy and zenith angle \cite{wcd_Sensitivity}.
The analysis uses a curvature/sampling correction.
This correction applies a quadratic function based on distance from the core \cite{Abeysekara_2017}.
The adjustment improves angular resolution.
However, discrepancies between simulation and observation persist.
These discrepancies introduce systematic errors into the analysis \cite{Abeysekara_2017}.

Angle reconstruction is vital for the HAWC Observatory.
It accurately traces primary particles back to their cosmic sources.
This tracing allows for exact correlations with known gamma-ray sources.

%$$$$$$$$$$$$$$$$$$$$$$$$$$$$$$$$$$$$$$$$$$$$$$$$$$$$$$$$$$$$$$$$$$$$$$$$$$$$$$$$$$$%
\subsection{$f_\mathrm{hit}$ Energy Estimation}\label{sec:hawc_fhit}
%$$$$$$$$$$$$$$$$$$$$$$$$$$$$$$$$$$$$$$$$$$$$$$$$$$$$$$$$$$$$$$$$$$$$$$$$$$$$$$$$$$$%

\begin{figure}
    \centering{
        \includegraphics[scale=0.6]{figures/hawc/fhit_bins.png}
    }
    \caption{\todo{coped from paper}. Normalized histogram of the energy distribution of each fhit bin defined in Table 2.1. This figure was made using Monte Carlo simulations assuming a spectral distribution E about 2.63 at a declination of 20◦. Figure from \cite{Abeysekara_2017}.}
    \label{fig:fhit_bins}
\end{figure}

\begin{table}
    \centering{
    \begin{tabular}{c?cc|c}
        \hline
        Bin &
        Lower Edge \% &
        Upper Edge \% &
        $\Theta_{68}$ ($^\circ$) \\
        \hline
        1       &
        6.7     &
        10.5    &
        1.05    \\

        2       &
        10.5    &
        16.2    &
        0.69    \\

        3       &
        16.2     &
        24.7    &
        0.50    \\

        4       &
        24.7     &
        35.6    &
        0.39    \\

        5       &
        35.6     &
        48.5    &
        0.30    \\

        6       &
        48.5     &
        61.8    &
        0.28    \\

        7       &
        61.8     &
        74.0    &
        0.22    \\

        8       &
        74.0     &
        84.0    &
        0.20    \\

        9       &
        84.0     &
        100    &
        0.17    \\

    \end{tabular}
    }
    \caption{\todo{copied}. Definitions of the fhit bins, given by the fraction of available PMTs that are triggered during an air shower event. The angular  resolution, denoted Ψ68 as the bin containing 68\% of the events, improves with larger events \cite{Abeysekara_2017}}
    \label{tab:fhit_bins}
\end{table}

The HAWC Observatory quantifies the primary particle energy of air showers using a metric known as $f_{\text{hit}}$.
This ratio compares the count of PMTs involved in the event reconstruction to the total number of functional PMTs at the time \cite{Abeysekara_2017}.
The main array consists of about 1200 PMTs, but the count may vary due to maintenance or other operational factors.

Events are stratified into several $f_{\text{hit}}$ bins.
Each bin corresponds to a specific range of angular resolutions, enabling a structured approach to event analysis based on the extent of the shower footprint, \cref{fig:core_fitter}.
The $f_{\text{hit}}$ metric, while effective, has several limitations.
It is dependent on the zenith angle and the spectral characteristics presumed for the observed source.
The variable also reaches a saturation point around 10 TeV, after which the detector's ability to discriminate between higher energy levels diminishes \cite{Abeysekara_2017}.
Furthermore, the energy distribution for each $f_{\text{hit}}$ bin is notably broad, see \cref{fig:fhit_bins}.
In response to these limitations, HAWC has developed more intricate algorithms for energy estimation.
These algorithms incorporate the zenith angle and the distribution of charge around the shower core for a more accurate assessment of the primary particle's energy, particularly at energies surpassing 10 TeV \cite{wcd_Sensitivity}.

The relationship between $f_{\text{hit}}$ and primary energy is complex.
Atmospheric attenuation can cause high-energy showers to present a smaller footprint, misrepresenting their energy in the $f_{\text{hit}}$ metric.
This effect is captured in simulations that chart the actual energy distribution across $f_{\text{hit}}$ categories \cite{wcd_Sensitivity}.
Such distributions vary with the declination of the source and the theoretical energy spectrum used in the model.

%$$$$$$$$$$$$$$$$$$$$$$$$$$$$$$$$$$$$$$$$$$$$$$$$$$$$$$$$$$$$$$$$$$$$$$$$$$$$$$$$$$$%
\subsection{Neural Network Energy Estimation}\label{sec:hawc_nn}
%$$$$$$$$$$$$$$$$$$$$$$$$$$$$$$$$$$$$$$$$$$$$$$$$$$$$$$$$$$$$$$$$$$$$$$$$$$$$$$$$$$$%

\begin{figure}
    \centering{
        \includegraphics[clip, trim=9.1cm 0cm 0cm 0cm, scale=0.9]{figures/hawc/NN_performance.jpg}
    }
    \caption{\todo{copied}NN (right) energy estimators. The dotted line is the identity line; events that fall along this line are reconstructed perfectly. Gamma/hadron separation cuts have been applied. The first energy bin starts at , and the last energy bin ends at , which accounts for the sharp features in the figures. Figure pulled from \cite{100TEV_Crab_HAWC}}
    \label{fig:NN_performance}
\end{figure}

The energy estimation for photon events at the HAWC Observatory is refined through an artificial neural network (NN) algorithm.
This method, based on the Toolkit for Multivariate Analysis NN, adopts a multilayer-perceptron model with logistic activation functions across its layers.
The structure includes two hidden layers, the first with 15 nodes and the second with 14, designed to process input variables through a framework optimized to closely estimate primary energies \cite{thesis_SamM}.

The NN is trained to minimize a specific error function that measures discrepancies between the NN's energy predictions and the actual energies from Monte Carlo simulations.
This minimization targets an error function that incorporates the relative importance of each event, weighting the importance to mimic an E-2 power law spectrum.
This approach helps achieve a uniform error rate across energies ranging from 1 to 100 TeV.
The optimization process leverages the Broyden-Fletcher-Goldfarb-Shanno algorithm, ensuring a precise calibration of the NN's 479 weights \cite{100TEV_Crab_HAWC}.

The spectral analysis employs a binned likelihood method, using a forward-folding technique to accommodate the energy estimate's bias and resolution.
This analysis benefits from a 2D binning scheme that categorizes events by both their $f_{\text{hit}}$ value and estimated energy.
The decision to use this scheme over a simple energy-based binning lies in the correlation between gamma/hadron separation parameters and the angular resolution with both the size and energy of the event.
The spectrum of interest is dissected into nine $f_{\text{hit}}$ bins, each further divided into 12 energy bins, spanning from 0.316 TeV to 316 TeV, encompassing a total of 108 bins \cite{100TEV_Crab_HAWC}.
However, not all bins contribute to the final analysis—those with low probability of event population or insufficient Monte Carlo simulation accuracy are excluded.
This selective approach focuses on the central 99\% of events by estimated energy within each $f_{\text{hit}}$ category, effectively removing outliers and ensuring a robust fit \cite{100TEV_Crab_HAWC}.

Input variables for the NN are selected to capture key characteristics of the air shower: energy deposition, containment, and atmospheric attenuation.
The algorithm calculates energy deposition using the fraction of PMTs and tanks activated, alongside the logarithm of the normalization from the lateral distribution fit.
Containment is inferred from the distance between the shower core and the array's center, while atmospheric attenuation is evaluated using the reconstructed zenith angle and a detailed analysis of the shower's lateral charge distribution \cite{thesis_SamM,100TEV_Crab_HAWC}.

This refined NN energy estimation methodology is an integral component of HAWC's toolkit, enabling precise analysis of high-energy gamma-ray events.
It represents a significant advancement in the field, offering deeper insights into cosmic phenomena by accurately mapping observed shower characteristics to primary particle energies.

%$$$$$$$$$$$$$$$$$$$$$$$$$$$$$$$$$$$$$$$$$$$$$$$$$$$$$$$$$$$$$$$$$$$$$$$$$$$$$$$$$$$%
\subsection{G/H Discrimination}\label{hawc:gammaHadron}
%$$$$$$$$$$$$$$$$$$$$$$$$$$$$$$$$$$$$$$$$$$$$$$$$$$$$$$$$$$$$$$$$$$$$$$$$$$$$$$$$$$$%

\begin{figure}
    \centering{
        \includegraphics[scale=0.45]{figures/hawc/LDF_particles.png}
    }
    \caption{\todo{copied}Lateral distribution functions of an obvious cosmic ray (left) and a photon candidate from the Crab Nebula (right).The cosmic ray has isolated high-charge hits far from the shower core due to penetrating particles in the hadronic air shower. These features are absent in the gamma-ray shower}
    \label{fig:ldf_particleshower}
\end{figure}

At the HAWC Observatory, distinguishing between air showers initiated by gamma rays and those by hadronic cosmic rays is fundamental for astrophysical data purity.
The separation process leverages differences in shower characteristics: electromagnetic showers from gamma rays typically display fewer muons and a smoother distribution, whereas hadronic showers are more chaotic due to the abundance of muons and hadronic sub-showers.

Discrimination Parameters
Two primary parameters facilitate the identification of cosmic-ray events \cite{Abeysekara_2017}:

Compactness (C): This parameter evaluates the charge captured by PMTs, particularly focusing on the PMT with the highest effective charge beyond a 40-meter radius from the shower core.
Compactness is inversely proportional to this effective charge, as higher charges at extended distances from the core are indicative of hadronic showers.
It is mathematically expressed as:
\compactness
PINCness (P): PINCness quantifies the "clumpiness" of a shower using the charges recorded by PMTs.
It is computed from the logarithm of the effective charge of each PMT hit, compared to an expected average for that annular region.
A higher PINCness suggests a less smooth distribution, typical of hadronic showers.
The formula is:
\pincness
These parameters are accurately modeled in simulations, aligning well with observational data near gamma-ray sources like the Crab Nebula.
Figures illustrating the lateral distributions for representative cosmic-ray and photon candidate showers, as well as the distribution of these discrimination parameters, affirm their efficacy \cite{Abeysekara_2017}.

Optimization and Application
The discrimination technique has remained consistent, but cut values have been reoptimized in each 2D bin based on $f_{\text{hit}}$ and estimated energy.
This refinement enhances the selection of high-energy events.
Each bin ensures at least 50\% efficiency for gamma-ray detection, with efficiencies extending up to nearly 100\% in certain bins \cite{Abeysekara_2017,100TEV_Crab_HAWC}.

%$$$$$$$$$$$$$$$$$$$$$$$$$$$$$$$$$$$$$$$$$$$$$$$$$$$$$$$$$$$$$$$$$$$$$$$$$$$$$$$$$$$%
\section{Background Estimation: Direct Integration}\label{hawc:direc_int}
%$$$$$$$$$$$$$$$$$$$$$$$$$$$$$$$$$$$$$$$$$$$$$$$$$$$$$$$$$$$$$$$$$$$$$$$$$$$$$$$$$$$%

The ratio of cosmic rays to gamma rays can be as high as 10,000 to 1, depending on the energy.
At HAWC, we confront a significant challenge even after gamma/hadron cuts: our gamma-ray data is still inundated with cosmic-ray events.
To tackle this, we rely on the direct integration method developed by Milagro \cite{Milagro_crab}.
This method capitalizes on the cosmic rays' isotropic nature resulting from their deflection by interstellar magnetic fields.

The direct integration method estimates background events by integrating over a stable two-hour period of detector operation.
The expected number of background events at a particular sky location ($\phi, \theta$) is determined by integrating the normalized detector's efficiency with the all-sky event rate:
\directInt
Here, $E(\text{ha}, \theta)$, represents the detector's efficiency, which varies with local coordinates (hour angle and declination).
$R(t)$ is the event rate as a function of time \cite{Milagro_crab}.

Our background estimation is expected to falter in high-energy ranges where cosmic-ray events are less frequent due to enhanced gamma/hadron discrimination. Sparsity in our background and data also arise at the limits of HAWC's sensitivity and during short-term analyses of transient events.
HAWC addresses these issues by using a pixel size of 0.5$^\circ$ in our background calculations to maintain robustness in our estimation \cite{Abeysekara_2017,wcd_Sensitivity}.
In constructing the background model, it's crucial to exclude areas of the sky with known gamma-ray sources.
Regions containing the Crab Nebula, Mrk 421, Mrk 501, and the Galactic Plane are masked to prevent their significant gamma-ray signals from biasing our background estimate \cite{Abeysekara_2017}.
