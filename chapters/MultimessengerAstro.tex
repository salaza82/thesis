%-----------------------------------------------------------------------------------%
\section{Introduction\label{sec:chap3_intro}}
%-----------------------------------------------------------------------------------%

Before the 20th century, all asttrophysics observations were optical in nature.
We litterally only saw things with highly magnifiied optical observations.
Then we discovered cosmic rays.
cosmic rays are charged particles, typically naked protons or H+.
This was seen by Victor Hess in 19??.
Around the same time we discovered neutrinos from beta decay.
Sometime around 1950 we started to build neutrino detectors which were mostly sensitive to neutrinos from the sun.
Finally, it was theorized that compact objects like black holes and neutron stars would create waves in space-time when they experience mergers or collisions.

In the 21st century, we have developed new observation techniques and detectors that are no only sensitive to these four messengers - photons (\todo{photon}), neutrinos (\todo{nu}), Cosmic Rays (CR), and Gravitational Wave (WV) - we're collect high energy versions of these events.
For the standad model particles, we're now sensitive to all messengers above the MeV eneryg range.
Additionally, the GW's were sensitive to are in the stellar mass black hole region and above within our galactic neighborhood.
This means were becoming sensitive to the fundamental physics occuring within the universe and we can rely on the universe as a TeV+ particle accelerator.
We also have the abaility to correlate high energy events across messengers and gain new insights on the processes that occur in our universe.

This thesis focuses on very high energy (VHE) gamma rays and neutrinos.
These can both be observed through the water cherenkov detection technique altho not exclusively.
Methods on how to detect and observe these neutral messengers are discussed \cref{sec:mtm_gamma} and \cref{sec:mtm_nu}

%-----------------------------------------------------------------------------------%
\section{Charged Particles in a Medium\label{sec:cherenkov}}
%-----------------------------------------------------------------------------------%

For high enery gamma-rays and neutrinos, we can exploit the same effect that charged particles have with water.
This effect is known as Cherenkov radiation.
Cherenkov Radiation occurs when a charged particle, usually electrons ($e$) or muons ($\mu$), traverse a medium, like water, faster than the speed of light in that medium.
This is similar to sonic boom where an object moves through air faster than the speed of sound in air.
Cherenkov radiation can therefor be thought of as an 'optic boom'.
Many astro-particle physics experiments will use water as the medium as because water has a unique set of properties ideal for charged particle tracking.

\tmpfig{Show a nuclear reactor with cherenkov radiation}

The frequency of light emmited due to cherenkov radiation follows the equation:
\eqin{Cherenkov wavelength calc}
The absorption spectra is shown in the following figure:
\tmpfig{absorption spectrum of liquid and solid water}

%-----------------------------------------------------------------------------------%
\section{Photons ($\gamma$)\label{sec:mtm_gamma}}
%-----------------------------------------------------------------------------------%

%-----------------------------------------------------------------------------------%
\section{Neutrinos ($\nu$)\label{sec:mtm_nu}}
%-----------------------------------------------------------------------------------%

%-----------------------------------------------------------------------------------%
\section{Opportunities to Combine for Dark Matter\label{sec:ic3_hawc_combo}}
%-----------------------------------------------------------------------------------%
