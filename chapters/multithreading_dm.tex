%%%%%%%%%%%%%%%%%%%%%%%%%%%%%%%%%%%%%%%%%%%%%%%%%%%%%%%%%%%%%%%%%%%%%%%%%%%%%%%%%%%%%
\section{Introduction}\label{sec:mtd_intro}
%%%%%%%%%%%%%%%%%%%%%%%%%%%%%%%%%%%%%%%%%%%%%%%%%%%%%%%%%%%%%%%%%%%%%%%%%%%%%%%%%%%%%

%%%%%%%%%%%%%%%%%%%%%%%%%%%%%%%%%%%%%%%%%%%%%%%%%%%%%%%%%%%%%%%%%%%%%%%%%%%%%%%%%%%%%
\section{Dataset and Background}\label{sec:mtd_databgd}
%%%%%%%%%%%%%%%%%%%%%%%%%%%%%%%%%%%%%%%%%%%%%%%%%%%%%%%%%%%%%%%%%%%%%%%%%%%%%%%%%%%%%

This section enumerates the data and background methods used for HAWC's multi-threaded study of dSphs.
\Cref{sec:mtd_data} and \Cref{sec:mtd_tools} are most useful for fellow HAWC collaborators looking to replicate a multithreaded dSph DM search.

%%%%%%%%%%%%%%%%%%%%%%%%%%%%%%%%%%%%%%%%%%%%%%%
\subsection{Itemized HAWC files}\label{sec:mtd_data}
%%%%%%%%%%%%%%%%%%%%%%%%%%%%%%%%%%%%%%%%%%%%%%%

\begin{itemize}
    \item Detector Resolution: \texttt{refit-Pass5-Final-NN-detRes-zenith-dependent.root}
    \item Data Map: \texttt{Pass5-Final-NN-maptree-ch103-ch1349-zenith-dependent.root}
    \item Spectral Dictionary: \texttt{HDMSpectra\_dict\_gamma.npy}
\end{itemize}

% %%%%%%%%%%%%%%%%%%%%%%%%%%%%%%%%%%%%%%%%%%%%%%%
\subsection{Software Tools and Development}\label{sec:mtd_tools}
%%%%%%%%%%%%%%%%%%%%%%%%%%%%%%%%%%%%%%%%%%%%%%%

This analysis was performed using HAL and 3ML \cite{Abeysekara_2017, vianello2015multimission} in Python version 3.
I built software in collaboration with Michael Martin and Letrell Harris to implement the \emph{Dark Matter Spectra from the Electroweak to the Planck Scale} (HDM) \cite{HDMSpectra} and dSphs spatial model from \cite{DM_Strigari20} for HAWC analysis.
A NumPy version of this dictionary was made for Py3.
The analysis was performed using the $f_{\textrm{hit}}$ framework performed in the HAWC Crab paper \cite{Abeysekara_2017}.
The Python2 NumPy dictionary file for gamma-ray final states is \texttt{dmCirSpecDict.npy}.
The corresponding Python3 file is \texttt{DM\_CirrelliSpectrum\_dict\_gammas.npy}.
These files can also be used for decay channels and the PPPC describes how \cite{Cirelli_2011}.
All other software used for data analysis, DM profile generation, and job submission to SLURM are also kept in my sandbox for \href{https://gitlab.com/hawc-observatory/sandboxes/salaza82/glory-duck-hawc}{the Glory Duck} project.

%%%%%%%%%%%%%%%%%%%%%%%%%%%%%%%%%%%%%%%%%%%%%%%
\subsection{Data Set and Background Description} \label{sec:mtd_data_bkgd}
%%%%%%%%%%%%%%%%%%%%%%%%%%%%%%%%%%%%%%%%%%%%%%%

The HAWC data maps used for this analysis contain 1017 days of data between runs 2104 (2014-11-26) and 7476 (2017-12-20).
They were generated from pass 4.0 reconstruction.
The analysis is performed using the $f_{hit}$ energy binning scheme with bins (1-9) similar to what was done for the Crab and previous HAWC dSph analysis \cite{Abeysekara_2017,Albert_2018}.
Bin 0 was excluded as it has substantial hadronic contamination and poor angular resolution.

This analysis was done on dSphs because of their large DM mass content relative to baryonic mass.
We consider the following to estimate the background to this study.

\begin{itemize}
    \item The dSphs are small in HAWC's field of view, so the analysis is not sensitive to large or small scale anisotropies.
    \item The dSphs used in this analysis are off the galactic plane.
    \item The dSphs are baryonically faint relative to their expected dark matter content and are not expected to contain high energy gamma-ray sources.
\end{itemize}

Therefor we make no additional assumptions on the background from our sources and use HAWC's standard direct integration method for background estimation \cite{Abeysekara_2017}.
It is possible for gamma rays from DM annihilation to scatter in transit to HAWC via Inverse Compton Scattering (ICS).
This was investigated and its impact on the flux is basically zero.
Supporting information on this is in \Cref{sec:gd_ics}
