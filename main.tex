\documentclass[lscape,PhD]{msu-thesis}

\usepackage[T1]{fontenc}    % use 8-bit T1 fonts
\usepackage[colorlinks=True,linkcolor=black,urlcolor=blue,citecolor=black]{hyperref}       % hyperlinks
\usepackage{xurl}            % simple URL typesetting
\usepackage{booktabs}       % professional-quality tables
\usepackage{amsfonts}       % blackboard math symbols
\usepackage{textgreek}
\usepackage{bm}
\usepackage{units}
\usepackage{nicefrac}       % compact symbols for 1/2, etc.
\usepackage{xcolor}
\usepackage{tabularx,booktabs} % pretty captions and table formatting
\usepackage{gensymb}        % Litterally needed just for degree symbol
\usepackage{mathtools}
\usepackage{xspace}         % Helpful for formatting fancy math text
\usepackage{graphicx}
\usepackage{multirow}       % merge cells in table
% \usepackage{afterpage}      % Page breaking for single tables
\usepackage{pdflscape}      % Page manipulate like landscape/portrait
\usepackage{arydshln}       % Tables with dashed lines
\usepackage{newtxtext,newtxmath}
\usepackage[justification=centering]{caption}
\usepackage[capitalize]{cleveref}
\usepackage[style=numeric,sorting=none]{biblatex}
\usepackage{listings}
\usepackage{makecell}
\usepackage{dark_matter}    % User defined macros

% \setlength{\afterchapskip}{\onelineskip}
\addbibresource{refs.bib}

\title{Leveraging Multi-Messenger Astrophysics for Dark Matter Searches}
\author{Daniel Nicolas Salazar-Gallegos}
\dualmajor{Physics and Astronomy}{Computational Mathematics, Science, and Engineering}
\date{2024}

\begin{document}
\frontmatter
\maketitlepage

%%%%%%%%%%%%%%%%%%%%%%%%%%%%%%%%%%%%%%%%%%%%%%%%%%%%%%%%%%%%%%%%%%%%%%%%%%%%%%%%%%%%%
\begin{abstract}

    This dissertation enhances the search for dark matter (DM) through a multi-wavelength and multi-messenger approach by combining data from leading gamma-ray and neutrino observatories.
    The first analysis combines data across five gamma-ray observatories, including the High Altitude Water Cherenkov (HAWC) Observatory and four others.
    This foundational study probed the largest DM mass space that spans from 5 GeV up to 100 TeV.
    This collaboration  achieved a threefold improvement in sensitivity to DM annihilation from dwarf galaxies.
    The collaborative framework developed among these five experiments also set a goal for including neutrino observatories in DM searches.
    Though no significant dark matter signals were detected, more stringent upper limits were set on the velocity-weighted annihilation cross-section, \sv.

    A pivotal aspect of the research involved developing new analytical methods within the IceCube Neutrino Observatory that achieved an order of magnitude increase in sensitivity for heavy dark matter annihilation.
    These improvements set the stage for multi-messenger dark matter searches with IceCube and gamma-ray observations.

    Within HAWC, computational methods were optimized to speed up the analysis pipeline by an order of magnitude.
    HAWC is therefore able to take on more ambitious DM searches and observe many more potential DM sources.

    The combined efforts from HAWC and IceCube form the basis for a novel multi-messenger strategy to probe dark matter annihilation searches.
    The research culminates in a preliminary mock limit from HAWC data and IceCube simulation.
    These analyses showcase the enhanced capability of detecting dark matter in the mass region above 1 PeV.
    With these two instruments alone, we could see two to threefold improvement to our sensitivity to heavy dark matter compared to these observatories searching for DM independently.
    This dissertation presents a comprehensive multi-instrument and multi-wavelength approach that provides a robust framework for future dark matter research.
    This work will be the foundation for a multi-messenger approach in the pursuit of unveiling the universe's dark sector.

\end{abstract}
%%%%%%%%%%%%%%%%%%%%%%%%%%%%%%%%%%%%%%%%%%%%%%%%%%%%%%%%%%%%%%%%%%%%%%%%%%%%%%%%%%%%%

\clearpage
\pagestyle{plain}

%%%%%%%%%%%%%%%%%%%%%%%%%%%%%%%%%%%%%%%%%%%%%%%%%%%%%%%%%%%%%%%%%%%%%%%%%%%%%%%%%%%%%
\chapter*{Acknowledgments}
\DoubleSpacing

This dissertation was ambitious and benefited greatly from support and love of my friends, family, and mentors.
These people believed in me at times when I found it hard to believe in myself.
Thank you so much, all of you, for encouraging me to follow my ambition and achieve great things.
Several of you come to mind and below I thank you individually in no particular order.

\textbf{My mentors}

Ben Hooberman, my undergraduate mentor who took a risk on an inexperienced undergraduate researcher.
He gave me the opportunities in physics I was craving and exposed me to the world of research.
The first of many doors opened when I started with Ben, and I am tremendously grateful.

Kirsten Tollefson, my graduate advisor and chair to my thesis committee.
She guided my ambitious program and kept me grounded.
Her direction and training taught me a lot about successful research and building strong, collaborative teams.
I will miss the days of gaming out research and project plans.
Thanks a ton for making graduate school a pleasant experience for me.

Pat Harding, member on my committee and mentor at Los Alamos National Lab.
I learned ways of thinking pragmatically towards solving problems with Pat.
He helped me lower the stakes and focused on the core goals in research.
Thanks for taking a chance on me and introducing me to Los Alamos, truly an enchanting place.

Mehr Nisa, previously a MSU postdoc, now faculty.
We arrived at MSU at about the same time and was the only postdoc/peer who was also working on both HAWC and IceCube.
Her expertise and guidance was critical in developing the ambitious research of this dissertation.
It's also been inspiring and a source of community to grow in our careers in parallel.
Thanks a ton for the mentorship, and supporting my many questions.

Jim Linnemann, faculty at MSU in HAWC and rigorous physics mentor.
Jim has been a role model of scientific rigor who has guided my physics skill set since arriving at MSU.
He is among the most exposed to my research and has similarly enabled my ambition.
It's been a pleasure to work with Jim and I thank you tremendously for your input to my research.

Thank you to Kendall Mahn, who saw my potential from the moment of application to MSU.
You cultivated the fight in me that's been exercised in leadership positions at MSU.

Thank you to the remaining members of my thesis committee: Claudio Kopper, Brian O'Shae, and Filomena Nunes.
You have been a great committee providing constructive feedback on this dissertation.

Thank you to the MSU postdocs Hans Niederhausen, Brian Clark, and Rob Halliday who provided substantial intellectual training and were great people to work with.

Thank you to the Los Alamos HAWC cast, Andrea Albert, Brenda Dingus, Kelly Malone and Mora Durocher for holding space for me and creating a second research home.
My experience at LANL was invaluable and left a positive, indelible mark on my experience as a grad student.

Thank you to grad students, Allison Peisker, Joe Lundeen, Jessie Micallef, and Minjin Jeong who showed me the ropes on research with HAWC and IceCube.
You provided a strong foundation for me to build on with my research.

\textbf{My Family}

Familia is core principle and social support within the Mexican culture, and I am no exception.
Thank you to my mom, and brothers, Miguel and Chris, especially who have kept me grounded through my Ph.D. and came to my rescue at my lowest points in it.
Thank you to my extended family in Texas and Mexico who have always had a door open for me and supported me as I redefined who I am.

We came from simple means, lives, and beginnings.
I hope that this effort and title inspires those within the family of the great things we are capable of.

\textbf{My Friends}

Thank you to the friends I made at Los Alamos and being the first to accept me as my authentic self.
You were among the first to show me to embrace myself.
Thank you specifically to Eames Bennet and Jeremy Monroe who opened their homes to me and did not ask for repayment.
A place to stay was invaluable for enabling my visits to LANL.

Thank you to the climbing belaytionships that are the backbone to my sanity through the COVID-19 pandemic and grad school.
David Shane, Jon Fleck, and Alex Harnisch, thank y'all for belaying me on so many routes and motivating me to push hard.
The grit and adventures we've had are cherished memories.

Thank you to Tristan Wells.
It's crazy that I met someone so chill and easy to get along with by accident.
I'm so happy to have grown in my Ph.D. with you and would not have chosen another roommate.
Thank you for being along for the ride and being a part of my life for these six years.

Thank you to Jenny and her late child Kaydince.
I only got to know you for a short amount of time, but you made a big splash.
I wish the best for you and that you are successful in your pursuit to cure the blind.

Thank you to my partner Corey Snyder.
You came at the best time to see me under the most stress and loved me nonetheless.
Your support and love propped me up through the roughest moment of my dissertation.
Quite literally, this work would not have been completed without your support.
Thank you for your kindness, motivation, love, laughs, safety, adventure, and thoughtful conversation.

\textbf{Institutions}

Thank you to the National Science Foundation for funding me for three years of my Ph.D.
Thank you to MSU for funding another two through their fellowship program.
Thank you to GEM for their fellowship program in partnership with LANL.
Thank you to LANL for always leaving a desk for me after my first summer in 2019.
Thank you to HAWC and IceCube for providing me with the resources and expertise to thrive.

%%%%%%%%%%%%%%%%%%%%%%%%%%%%%%%%%%%%%%%%%%%%%%%%%%%%%%%%%%%%%%%%%%%%%%%%%%%%%%%%%%%%%

\clearpage

\SingleSpacing
\tableofcontents*
% \DoubleSpacing

\clearpage
\mainmatter

%%%%%%%%%%%%%%%%%%%%%%%%%%%%%%%%%%%%%%%%%%%%%%%%%%%%%%%%%%%%%%%%%%%%%%%%%%%%%%%%%%%%%
\chapter{Dark Matter in the Cosmos}\label{sec:dm_cosmos}
%%%%%%%%%%%%%%%%%%%%%%%%%%%%%%%%%%%%%%%%%%%%%%%%%%%%%%%%%%%%%%%%%%%%%%%%%%%%%%%%%%%%%
%-----------------------------------------------------------------------------------%
\subsection{Introduction\label{sec:intro2dm}}
%-----------------------------------------------------------------------------------%

Dark Matter (DM) has been a whispering problem in physics for almost 100 years.
Anomolies have been detected by way of weird galaxy behaviour, budding Cosmology, and more \ns.
It was sometime in 1930's when the super duper smart Zwicky measured that it was defintely there.
It's kind of a big deal because we have no idea what the nature of this stuff and there's a lot of it.
According to Lambda CDM, the most legit model, \ns DM is about 85\% \fu, of all mass in the universe.
It's called dark in fact because we cannot see it. \ns
Finding out what the hell it is, is an active field of research and hopefully it interacts with the standard model.

Here's what we do know about DM so far\dots
DM is dark, it doesn't interact readily with light.
DM also doesn't interact noticably with the other standard model forces (EM, Strong, Weak) at a rate that matters \ns.
DM is cold.
By cold I mean that it is most likely not moving at relativisic speeds like neutrinos and photons. \ns
If it was moving that fast, the structures we see like galaxies would be much more diffuse than what is observed. \ns
DM is old.
DM played a critical role in the formation of the universe and the structure within it. \ns
We know this from Cosmology and computer universe simulations \ns.

The search for DM is basically summarized by trying a bunch of different models and performing measurements of all kinds to test them.
These models of course have to nominally agree with the known observations seen over the last century.
Whenever we perform a test and don't see anything, the parameter spaces gets more contrained.
I discuss some of the ideas ad approaches further on.
I Especially discuss the models that are relavent to my thesis.

%-----------------------------------------------------------------------------------%
\subsection{Evidence for Dark Matter\label{sec:evidence4dm}}
%-----------------------------------------------------------------------------------%

Let me show you why we're pretty sure DM is a thing and why it might be particle like in nature.
My thesis focuses on WIMP dark matter which is one of the better motivated things out there.

%$$$$$$$$$$$$$$$$$$$$$$$$$$$$$$$$$$$$$$$$$$$$$$$$$$$$$$$$$$$$$$$$$$$$$$$$$$$$$$$$$$$%
\subsubsection{First Clues: Stellar Velocities\label{sec:ev4dm_stars}}
%$$$$$$$$$$$$$$$$$$$$$$$$$$$$$$$$$$$$$$$$$$$$$$$$$$$$$$$$$$$$$$$$$$$$$$$$$$$$$$$$$$$%
THE STARS ARE MOVING TO FAST AAAAHAHHAHA


%$$$$$$$$$$$$$$$$$$$$$$$$$$$$$$$$$$$$$$$$$$$$$$$$$$$$$$$$$$$$$$$$$$$$$$$$$$$$$$$$$$$%
\subsubsection{Mounting Evidence for Dark Matter\label{secc:ev4dm_more}}
%$$$$$$$$$$$$$$$$$$$$$$$$$$$$$$$$$$$$$$$$$$$$$$$$$$$$$$$$$$$$$$$$$$$$$$$$$$$$$$$$$$$%
SPACE TIME CURVY
THE UNIVERSE IS A SIMULATION.
THE WIGGLES IN GEOMETRY ARE THICCCC

%-----------------------------------------------------------------------------------%
\subsection{The WIMP Miracle\label{sec:dm_wimps}}
%-----------------------------------------------------------------------------------%

%-----------------------------------------------------------------------------------%
\subsection{Searching for Dark Matter\label{dm_search}}
%-----------------------------------------------------------------------------------%

%$$$$$$$$$$$$$$$$$$$$$$$$$$$$$$$$$$$$$$$$$$$$$$$$$$$$$$$$$$$$$$$$$$$$$$$$$$$$$$$$$$$%
\subsubsection{Shake it, Break it, Make it\label{sec:bop_it}}
%$$$$$$$$$$$$$$$$$$$$$$$$$$$$$$$$$$$$$$$$$$$$$$$$$$$$$$$$$$$$$$$$$$$$$$$$$$$$$$$$$$$%

%$$$$$$$$$$$$$$$$$$$$$$$$$$$$$$$$$$$$$$$$$$$$$$$$$$$$$$$$$$$$$$$$$$$$$$$$$$$$$$$$$$$%
\subsubsection{Break it: Standard Model Signatures of Annihilating Dark Matter\label{sec:break_it}}
%$$$$$$$$$$$$$$$$$$$$$$$$$$$$$$$$$$$$$$$$$$$$$$$$$$$$$$$$$$$$$$$$$$$$$$$$$$$$$$$$$$$%

%-----------------------------------------------------------------------------------%
\subsection{Multi-Messenger Dark Matter}
%-----------------------------------------------------------------------------------%

%-----------------------------------------------------------------------------------%
\subsection{Search Targets for Dark Matter\label{sec:dm_targets}}
%-----------------------------------------------------------------------------------%

%$$$$$$$$$$$$$$$$$$$$$$$$$$$$$$$$$$$$$$$$$$$$$$$$$$$$$$$$$$$$$$$$$$$$$$$$$$$$$$$$$$$%
\subsubsection{Dwarf Spheroidal Galaxies\label{sec:dSphs}}
%$$$$$$$$$$$$$$$$$$$$$$$$$$$$$$$$$$$$$$$$$$$$$$$$$$$$$$$$$$$$$$$$$$$$$$$$$$$$$$$$$$$%

%%%%%%%%%%%%%%%%%%%%%%%%%%%%%%%%%%%%%%%%%%%%%%%%%%%%%%%%%%%%%%%%%%%%%%%%%%%%%%%%%%%%%
\chapter{High Altitude Water Cherenkov (HAWC) Observatory}\label{sec:hawc}
%%%%%%%%%%%%%%%%%%%%%%%%%%%%%%%%%%%%%%%%%%%%%%%%%%%%%%%%%%%%%%%%%%%%%%%%%%%%%%%%%%%%%
%-----------------------------------------------------------------------------------%
\section{The Detector}\label{sec:THE_hawc}
%-----------------------------------------------------------------------------------%

\begin{figure}[h!]
    \centering{
    \includegraphics[scale=0.155]{figures/hawc/HAWC.jpg}
    }
    \caption{Photo of the HAWC detector that I took on May 17, 2023. Main array is centered in the photo and comprised of the larger tanks. Outriggers are the smaller tanks around the main array.}
    \label{fig:HAWC}
\end{figure}

The High Altitude Water Cherenkov (HAWC) Observatory is a specialized instrument designed for the observation of high energy gamma-rays and cosmic rays \cite{HAWC_NIM}.
Located on the Sierra Negra volcano in Mexico, HAWC observes gamma rays and cosmic rays in the energy range of approximately 100 GeV to 100's of TeV.
HAWC is strategically situated to maximize observational efficiency due to its high altitude.
At an elevation of 4,100 meters, it monitors about two-thirds of the sky every day with an uptime above 90\%.
This capability is essential for studying high-energy astrophysical phenomena.

HAWC consists of 300 water Cherenkov detectors (WCDs) spread over 22,000 $m^2$.
Each main array detector is filled with purified water and equipped with four, upward-facing photomultiplier tubes (PMTs).
See \cref{fig:WCD_schematic} for schematic of WCDs.
These PMTs detect Cherenkov radiation from charged particles passing through the tanks.
These charged particles are generated when a high energy gamma or cosmic ray collides with gas in the atmosphere to create a charged particle shower, see \cref{fig:airshowers}.
The observatory includes a separate tank configuration which are refered to as the outriggers.
They are a secondary array of 345 smaller WCD's.
Surrounding the main array, each outrigger tank measures 1.55 meters in diameter and height and contain a single upward-facing eight-inch PMT.
This add-on increases the instrumented footprint fourfold.
The outriggers are meant to improve the reconstruction of showers extending beyond the main array, especially for events above 10 TeV.
However, at the time of writing this thesis, the outriggers have not been fully integrated into HAWC's reconstruction software.

\begin{figure}
    \centering{
    \includegraphics[scale=0.5]{figures/hawc/high_energy_air_shower.png}
    }
    \caption{A particle physics illustration of high energy particle showers. Left shower is an electromagnetic shower from a high energy gamma-ray. Most particles in the shower will be a combination of photons and charged leptons, in this case electrons (e). Right figure shows a cosmic ray particle shower. The cosmic ray will produce many more types of particles including pions ($\pi$), neutrinos, and charged leptons. Figured pulled from \cite{lopez_thesis}.}
    \label{fig:airshowers}
\end{figure}

%-----------------------------------------------------------------------------------%
\subsection{Construction and Hardware} \label{sec:hawc_hardware}
%-----------------------------------------------------------------------------------%

\begin{figure}
    \centering{
        \includegraphics[scale=0.14]{figures/hawc/WCDs.jpg}
        \includegraphics[scale=0.45]{figures/hawc/WCD_schematic.png}
    }
    \caption{The WCDs. Left image features several WCDs looking from within the main array of HAWC. Right image shows a schematic of a WCD pulled from \cite{HAWC_NIM}.}
    \label{fig:WCD_schematic}
\end{figure}

Each main array WCD, see \cref{fig:WCD_schematic}, is a cylindrical tank with dimensions of 7.3 m in diameter and 5.4 m in height and filled with 180,000 L of water \cite{HAWC_NIM}.
The metal shell of these tanks is made from bolted together, corrugated, galvanized steel panels.
The tanks are placed into 0.6 m deep trenches filled with rammed earth to secure it against seismic activity \cite{HAWC_NIM}.
The interior of each tank is lined with a black, low-density polyethylene bladder, designed to be impermeable to external light and to prevent reflection of Cherenkov light within the tank.
This bladder is approximately 0.4 mm thick and composed of two layers of three-substrate film.
To further minimize light penetration, a black agricultural foil covers the bladder.
The ground and walls inside the tank are protected with felt and sand to safeguard against punctures.
The tanks are filled 4.5 m deep of purified water, achieving a photon attenuation length for Cherenkov photons that exceeds the tank's dimensions \cite{HAWC_NIM}.
This purification level ensures the optimal detection environment for the photons generated by traversing charged particles.

At the base of each tank, four photomultiplier tubes (PMTs) are installed to detect the Cherenkov radiation emitted by charged particles in watre.
Three 8-inch diameter PMTs surround a larger 10 inch PMT from Hamamatsu \cite{hawc_pmt}.
The variation in PMT response is carefully accounted for in event reconstruction algorithms.
Signals from the PMTs traverse 610 ft cables to the counting house, where they are processed by Front-End Boards (FEBs), see \cref{fig:basic_tanks_schem,fig:dig_schem}.
These FEBs, along with Time to Digital Converters (TDCs), digitize the signals and manage the high voltage supply to the PMTs.

%-----------------------------------------------------------------------------------%
\subsection{Data Acquisition and Signal Processing} \label{sec:hawc_daq}
%-----------------------------------------------------------------------------------%

\begin{figure}
    \centering{
        \includegraphics[scale=0.5]{figures/hawc/tank_basic_schem.png}
    }
    \caption{Overview of HAWC control and data electronics. The LoToT and HiToT threshold signals are discussed in \cref{sec:hawc_daq}. Figure from \cite{HAWC_NIM}}
    \label{fig:basic_tanks_schem}
\end{figure}

\begin{figure}
    \centering{
        \includegraphics[scale=0.6]{figures/hawc/digital_schematic.png}
    }
    \caption{Schematic of data flow in HAWC data acquisition and online processing system. Pulled from \cite{HAWC_DAQ_NIM}.}
    \label{fig:dig_schem}
\end{figure}

The HAWC data acquisition (DAQ) and signal processing systems convert the physical detection of particles into analyzable data.
This process involves a series of steps from initial signal detection by PMTs to digital conversion and preliminary analysis, see \cref{fig:dig_schem,fig:tot_threholds}.

\begin{figure}
    \centering{
        \includegraphics[scale=0.6]{figures/hawc/ToT_threshold.png}
    }
    \caption{How HAWC FEB intially processes analog PMT signals. Signals are split through an amplifier and discriminator circuit. Each path is designated for either the HIGH or LOW threshold for the signal. The 2-edge event corresponds to LOW, while the 4 edge corresponds to HIGH.}
    \label{fig:tot_threholds}
\end{figure}

Once the signal from the PMTs arrive at the counting house, they enter the Front-End Boards (FEBs).
The FEBs are responsible for the initial processing of these signals, which includes amplification and integration \cite{Milagro_DAQ}.
Each PMT signal is compared against preset LOW/HIGH voltage thresholds in the FEBs, see \cref{fig:tot_threholds}, identifying signals that correspond to about 1/4 and 4 photoelectrons, respectively.
This differentiation allows the system to gauge the strength of the detected Cherenkov radiation.
The processed signals are then digitized by Time to Digital Converters (TDCs).
These converters measure the time over threshold (ToT) for each signal, a parameter that reflects both the duration and amplitude of the signal.
This digitization facilitates reconstruction of the original event for translating the physical interactions within the detectors into data \cite{HAWC_NIM,HAWC_DAQ_NIM,Milagro_DAQ}.

Synchronization across the HAWC observatory is maintained by a central GPS Timing and Control (GTC) system, which achieves a timing resolution of 98 ps.
This high-resolution timing is vital for accurately reconstructing the timing and location of air showers initiated by cosmic and gamma rays.
The GTC system ensures that all components of the DAQ operate in unison to preserve the temporal integrity of the detected events \cite{HAWC_NIM,hawc_daq_thesis}.

Once digitized, the data are transferred to an online event reconstruction system.
This system runs the Reconstruction Client, which utilizes the raw PMT data to reconstruct the characteristics of the air showers, such as their direction and energy \cite{HAWC_DAQ_NIM}.
The capacity for real-time analysis allows HAWC to promptly respond to astrophysical phenomena like Gamma Ray Bursts (GRBs) and to participate in multi-messenger astronomy by following up on alerts from other observatories.
This real-time processing system is designed to handle high data throughput, using ZeroMQ \cite{zeromq} for efficient data transfer between software components.
Analysis Clients perform specific online analyses that require immediate data, including monitoring for GRBs, solar flare activity, and participation in global efforts to track gravitational waves and neutrinos \cite{HAWC_NIM}.

The DAQ system is overseen by an Experiment Control system and crew that manage the operational aspects of data collection.
This includes initiating and terminating data collection runs and monitoring the experiment for errors.
In the event of a system crash, often caused by environmental factors such as lightning, the Experiment Control system is designed to automatically restart the experiment and minimize downtime \cite{HAWC_NIM,HAWC_DAQ_NIM}.

%-----------------------------------------------------------------------------------%
\section{Event Reconstruction} \label{sec:hawc_reconstruction}
%-----------------------------------------------------------------------------------%

Event reconstruction at the HAWC Observatory is a critical procedure that converts the raw data from the observatory's WCDs into a coherent framework for understanding cosmic and gamma-ray events.
This process includes several distinct steps.
Core Fitting determines the geometric center of the air shower on the detector plane.
Angle Reconstruction assesses the trajectory of the incoming particle, revealing its origin in the sky.
Energy Estimation is performed using both \textit{f}-hit and Neural Network (NN) methods to quantify the energy of the detected events.
Gamma/Hadron discrimination differentiates between gamma-ray and hadronic cosmic ray initiated showers, a vital step for astrophysical interpretations.
Each of these steps is integral to the observatory's objective of investigating the high-energy universe and enable the transformation of signals into detailed insights about high energy cosmic phenomena.

%$$$$$$$$$$$$$$$$$$$$$$$$$$$$$$$$$$$$$$$$$$$$$$$$$$$$$$$$$$$$$$$$$$$$$$$$$$$$$$$$$$$%
\subsection{Core Fitting} \label{sec:hawc_core_fitting}
%$$$$$$$$$$$$$$$$$$$$$$$$$$$$$$$$$$$$$$$$$$$$$$$$$$$$$$$$$$$$$$$$$$$$$$$$$$$$$$$$$$$%

\begin{figure}[h!]
    \centering{
    \includegraphics[scale=0.5]{figures/hawc/shower_shape.png}
    }
    \caption{An particle shower incident on WCDs. Secondary particles of an air shower travel in a cone centered on primary incident particle. Reconstruction of the initial angle is possible with arrival time of hits in PMTs inside WCDs. Figure from \cite{thesis_Zigg}.}
    \label{fig:shower_shape}
\end{figure}

In the study of air showers, accurately determining the location of the air shower core on the ground is crucial for reconstructing the direction of the originating primary particle.
An illustration of this can be seen in a HAWC event plot, \cref{fig:airshowers,fig:ldf_particleshower}, where the lateral charge distribution across the array is displayed.
The core is identified and marked with a red star, reconstructed using a predetermined functional form, \cref{eq:hawc_showercore}.

We model signal $S_i$ from the \textit{i}th PMT is given by the following equation:
\showercore
In this model, $\tilde{x}$ represents the core location and $\tilde{x}_i$ is the position of the \textit{i}th PMT.
$R_m$ stands for the Molière radius, which is approximately 120 meters at the altitude of HAWC.
$\sigma$ is the standard deviation of the Gaussian distribution.
$N$ is the normalization factor for the tail of the distribution.
The equation incorporates fixed values of $\sigma = 10$ m and $N=5.10^{-5}$.
This leaves the core location and overall amplitude $A$ as the free parameters to be determined during fitting.

\begin{figure}
    \centering{
        \includegraphics[scale=0.5]{figures/hawc/core_fitting.png}
    }
    \caption{Charge deposition in each PMT for a reconstructed gamma-ray event. WCDs are outlined in black surrounding the 4 smaller circles that represent PMTs. The color scale indicates the charge deposition in each PMT. The best shower core fit from SFCF is noted with a red star in the center of the dashed circle \cite{Abeysekara_2017}.}
    \label{fig:core_fitter}
\end{figure}

The chosen functional form for the Super Fast Core Fit (SFCF) algorithm is a simplified version of a modified Nishimura-Kamata-Greisen (NKG) function \cite{cosmic_ray_shape}, selected for its computational efficiency which is essential for rapid fitting of air shower cores.
The SFCF form allows numerical minimization to converge more quickly due to the function's simplicity, the analytical computation of its derivatives, and the absence of a pole at the core location \cite{Abeysekara_2017}.
\Cref{fig:core_fitter} provides a visualization of a recorded event, with the plot depicting the charge recorded by each PMT as a function of the distance to the reconstructed shower core.
Through the application of the SFCF, core locations can be identified with a median error of approximately 2 m for large events and about 4 m for smaller ones, assuming the gamma-ray event core impacts directly upon the HAWC detector array \cite{Abeysekara_2017}.
It is noted that as the core's distance from the main array increases, the precision in locating the core diminishes \cite{Abeysekara_2017}, highlighting the importance of proximity in the accuracy of core reconstruction.


%$$$$$$$$$$$$$$$$$$$$$$$$$$$$$$$$$$$$$$$$$$$$$$$$$$$$$$$$$$$$$$$$$$$$$$$$$$$$$$$$$$$%
\subsection{Angle Reconstruction} \label{sec:hawc_angleReco}
%$$$$$$$$$$$$$$$$$$$$$$$$$$$$$$$$$$$$$$$$$$$$$$$$$$$$$$$$$$$$$$$$$$$$$$$$$$$$$$$$$$$%

After establishing the core position, the next step is angle reconstruction.
This process determines the primary particle's trajectory.
The angle of arrival is indicative of the originating gamma ray's direction.
It correlates to the cosmic source of the gamma-ray.
We deduce this angle using the timing of PMT hits \cite{Abeysekara_2017}.

The air shower's front is conically shaped, not flat.
This shape arises from the travel patterns of secondary particles.
An event example is illustrated in \cref{fig:shower_shape}.
Far from the core, secondary particles undergo multiple scattering.
They also travel longer distances \cite{wcd_Sensitivity}.
Particle sampling decreases with distance from the core.
This decrease results in measurable delays in arrival times \cite{wcd_Sensitivity,Abeysekara_2017}.
Simulations provide a corrective measure for these effects.
The correction is a function of shower parameters \cite{Abeysekara_2017}.
It adjusts both curvature and sampling.
The distance from the shower core and the charge recorded by PMTs are crucial to this correction.
A function based on simulation and Crab Nebula observations is used for this purpose \cite{Abeysekara_2017}.
This curvature correction allows us to fit the particle front as a plane wave.

Corrections lead to the $\chi^2$ minimization step.
This technique fits a plane to the timing data of the PMTs.
It then calculates the shower's angle of arrival.
The zenith and azimuth angles are the results of this fit \cite{wcd_Sensitivity}.
The local angles are converted to celestial coordinates.
These coordinates allow correlation with gamma-ray sources.
Right ascension (RA) and declination (Dec) are used for this purpose.
RA is akin to longitude, and Dec to latitude.

The reconstructed angle's resolution ranges from 0.1° to 1°.
This range depends on the incoming particle's energy and zenith angle \cite{wcd_Sensitivity}.
The analysis uses a curvature/sampling correction.
This correction applies a quadratic function based on distance from the core \cite{Abeysekara_2017}.
The adjustment improves angular resolution.
However, discrepancies between simulation and observation persist.
These discrepancies introduce systematic errors into HAWC analyses \cite{Abeysekara_2017}.

%$$$$$$$$$$$$$$$$$$$$$$$$$$$$$$$$$$$$$$$$$$$$$$$$$$$$$$$$$$$$$$$$$$$$$$$$$$$$$$$$$$$%
\subsection{$f_\mathrm{hit}$ Energy Estimation}\label{sec:hawc_fhit}
%$$$$$$$$$$$$$$$$$$$$$$$$$$$$$$$$$$$$$$$$$$$$$$$$$$$$$$$$$$$$$$$$$$$$$$$$$$$$$$$$$$$%

\begin{figure}
    \centering{
        \includegraphics[scale=0.6]{figures/hawc/fhit_bins.png}
    }
    \caption{Simulated normalized energy distribution of each $f_\mathrm{hit}$ bin defined in \cref{tab:fhit_bins}. Monte Carlo simulation of gamma-rays with $E^-{2.63}$ spectral shape and simulated source at 20$^\circ$ declination. Figure from \cite{Abeysekara_2017}.}
    \label{fig:fhit_bins}
\end{figure}

\begin{table}
    \centering{
    \begin{tabular}{c?cc|c}
        \hline
        Bin &
        Lower Edge \% &
        Upper Edge \% &
        $\Theta_{68}$ ($^\circ$) \\
        \hline
        1       &
        6.7     &
        10.5    &
        1.05    \\

        2       &
        10.5    &
        16.2    &
        0.69    \\

        3       &
        16.2     &
        24.7    &
        0.50    \\

        4       &
        24.7     &
        35.6    &
        0.39    \\

        5       &
        35.6     &
        48.5    &
        0.30    \\

        6       &
        48.5     &
        61.8    &
        0.28    \\

        7       &
        61.8     &
        74.0    &
        0.22    \\

        8       &
        74.0     &
        84.0    &
        0.20    \\

        9       &
        84.0     &
        100    &
        0.17    \\

    \end{tabular}
    }
    \caption{\todo{copied}. Definitions of the fhit bins, given by the fraction of available PMTs that are triggered during an air shower event. The angular  resolution, denoted Ψ68 as the bin containing 68\% of the events, improves with larger events \cite{Abeysekara_2017}}
    \label{tab:fhit_bins}
\end{table}

The HAWC Observatory quantifies the primary particle energy of air showers using a metric known as $f_{\text{hit}}$.
This ratio compares the count of PMTs involved in the event reconstruction to the total number of functional PMTs at the time \cite{Abeysekara_2017}.
The main array consists of about 1200 PMTs, but the count may vary due to maintenance or other operational factors.

Events are stratified into several $f_{\text{hit}}$ bins.
Each bin corresponds to a specific range of angular resolutions, enabling a structured approach to event analysis based on the extent of the shower footprint, see \cref{tab:fhit_bins}.
The $f_{\text{hit}}$ metric, while effective, has several limitations.
It is dependent on the zenith angle and the spectral characteristics presumed for the observed source.
The variable also reaches a saturation point around 10 TeV, after which the detector's ability to discriminate between higher energy levels diminishes \cite{Abeysekara_2017}.
Furthermore, the energy distribution for each $f_{\text{hit}}$ bin is notably broad, see \cref{fig:fhit_bins}.
In response to these limitations, HAWC has developed more intricate algorithms for energy estimation.
These algorithms incorporate the zenith angle and the distribution of charge around the shower core for a more accurate assessment of the primary particle's energy, particularly at energies surpassing 10 TeV \cite{wcd_Sensitivity}.

The relationship between $f_{\text{hit}}$ and primary energy is complex.
Atmospheric attenuation can cause high-energy showers to present a smaller footprint, misrepresenting their energy in the $f_{\text{hit}}$ metric.
This effect is captured in simulations that chart the actual energy distribution across $f_{\text{hit}}$ categories \cite{wcd_Sensitivity}.
Such distributions vary with the declination of the source and the theoretical energy spectrum used in the model.

%$$$$$$$$$$$$$$$$$$$$$$$$$$$$$$$$$$$$$$$$$$$$$$$$$$$$$$$$$$$$$$$$$$$$$$$$$$$$$$$$$$$%
\subsection{Neural Network Energy Estimation}\label{sec:hawc_nn}
%$$$$$$$$$$$$$$$$$$$$$$$$$$$$$$$$$$$$$$$$$$$$$$$$$$$$$$$$$$$$$$$$$$$$$$$$$$$$$$$$$$$%

\begin{figure}
    \centering{
        \includegraphics[clip, trim=9.1cm 0cm 0cm 0cm, scale=0.9]{figures/hawc/NN_performance.jpg}
    }
    \caption{Neural Network energy estimator performance compared to true energy. The dotted line is the identity line where the estimator and injection agree. Gamma/hadron separation cuts were applied with the energy estimation. Figure pulled from \cite{100TEV_Crab_HAWC}}
    \label{fig:NN_performance}
\end{figure}

The energy estimation for photon events at the HAWC Observatory is refined through an artificial neural network (NN) algorithm.
This method, based on the Toolkit for Multivariate Analysis NN, adopts a multilayer-perceptron model with logistic activation functions across its layers.
The structure includes two hidden layers, the first with 15 nodes and the second with 14, designed to process input variables through a neural network optimized to estimate primary particle energies \cite{thesis_SamM}.

The NN is trained to minimize a specific error function that measures discrepancies between the NN's energy predictions and the actual energies from Monte Carlo simulations.
This minimization targets an error function that incorporates the relative importance of each event, weighting more the importance to mimic an $E^{-2}$ power law spectrum.
This approach helps achieve a uniform error rate across energies ranging from 1 to 100 TeV.
The optimization process leverages the Broyden-Fletcher-Goldfarb-Shanno algorithm that calibrates the NN's 479 weights \cite{100TEV_Crab_HAWC}.

The spectral analysis employs a binned likelihood method, using a forward-folding technique to accommodate the energy estimate's bias and resolution \cite{100TEV_Crab_HAWC}.
This establishes a 2D binning scheme that categorizes events by both their $f_{\text{hit}}$ value and estimated energy.
The decision to use this scheme over a simple energy-based binning lies in the correlation between gamma/hadron separation parameters and the angular resolution with both the size and energy of the event.
The spectrum of interest is partitioned into nine $f_{\text{hit}}$ bins, each further divided into 12 energy bins, spanning from 0.316 TeV to 316 TeV, encompassing a total of 108 bins \cite{100TEV_Crab_HAWC}.
However, not all bins contribute to the final estimate.
Bins with low event populations or insufficient Monte Carlo simulation are excluded.
This approach focuses on the central 99\% of events by estimated energy within each $f_{\text{hit}}$ bin, effectively removing outliers \cite{100TEV_Crab_HAWC}.

Input variables for the NN are selected to capture key characteristics of the air shower: energy deposition, containment, and atmospheric attenuation.
The algorithm calculates energy deposition using the fraction of PMTs and tanks activated, alongside the logarithm of the normalization from the lateral distribution fit.
Containment is inferred from the distance between the shower core and the array's center, while atmospheric attenuation is evaluated using the reconstructed zenith angle and a detailed analysis of the shower's lateral charge distribution \cite{thesis_SamM,100TEV_Crab_HAWC}.

This refined NN energy estimation methodology is an integral component of HAWC's toolkit, enabling precise analysis of high-energy gamma-ray events.
It represents a significant advancement in the field by more accurately mapping observed shower characteristics to primary particle energies.

%$$$$$$$$$$$$$$$$$$$$$$$$$$$$$$$$$$$$$$$$$$$$$$$$$$$$$$$$$$$$$$$$$$$$$$$$$$$$$$$$$$$%
\subsection{G/H Discrimination}\label{hawc:gammaHadron}
%$$$$$$$$$$$$$$$$$$$$$$$$$$$$$$$$$$$$$$$$$$$$$$$$$$$$$$$$$$$$$$$$$$$$$$$$$$$$$$$$$$$%

\begin{figure}
    \centering{
        \includegraphics[scale=0.45]{figures/hawc/LDF_particles.png}
    }
    \caption{Lateral distribution functions (LDFs) for cosmic ray (left) and a photon candidate from the Crab Nebula (right). Cosmic ray LDF has clearly isolated hits far from the reconstructed shower core. Gamma-ray shower shows a more cuspy event \cite{Abeysekara_2017}.}
    \label{fig:ldf_particleshower}
\end{figure}

At the HAWC Observatory, distinguishing between air showers initiated by gamma rays and those by hadronic cosmic rays is fundamental for astrophysical data purity.
The separation process leverages differences in shower characteristics: electromagnetic showers from gamma rays typically display fewer muons and a smoother lateral distribution, whereas hadronic showers are more chaotic due to the abundance of muons and hadronic sub-showers.

Two primary parameters facilitate the identification of cosmic-ray events \cite{Abeysekara_2017}:

Compactness (C): This parameter evaluates the charge captured by PMTs, particularly focusing on the PMT with the highest effective charge beyond a 40-meter radius from the shower core.
Compactness is inversely proportional to this effective charge, as higher charges at extended distances from the core are indicative of hadronic showers.
It is mathematically expressed as:
\compactness
where $N_\mathrm{hit}$ is the number of PMTs hit and $CxPE_{40}$ is the effective charge measured outside a 40 m radius from the shower cores \cite{Abeysekara_2017}.

PINCness (P): PINCness quantifies the "clumpiness" of a shower using the charges recorded by PMTs and is short for Parameter for Identifying Nuclear Cosmic Rays.
It is computed from the logarithm of the effective charge, $Q_{\mathrm{eff},i}$, of each PMT hit, $i$, compared to an expected average for that annular region.
A higher PINCness suggests a less smooth distribution, typical of hadronic showers.
The formula is:
\pincness
where $\zeta_i = \log_{10}(Q_{\mathrm{eff},i})$.
The average, $\langle \zeta \rangle$ is the average over an anular region surrounding the shower core.
The errors, $\sigma_{\zeta_i}$, are computed and allocated from gamma-ray candidates close to the Crab.

These parameters are tested and modeled in simulations and with observational data near the Crab Nebula.
\Cref{fig:ldf_particleshower} illustrating the lateral distributions for representative cosmic-ray and photon candidate showers, as well as the distribution of these discrimination parameters, affirm their efficacy \cite{Abeysekara_2017}.

The discrimination technique has remained consistent, but cut values have been reoptimized for the 2D bins based on $f_{\text{hit}}$ and NN estimated energy.
This refinement enhances the selection of high-energy events.
Each bin ensures at least 50\% efficiency for gamma-ray detection, with efficiencies extending up to nearly 100\% in certain bins \cite{Abeysekara_2017,100TEV_Crab_HAWC}.

%$$$$$$$$$$$$$$$$$$$$$$$$$$$$$$$$$$$$$$$$$$$$$$$$$$$$$$$$$$$$$$$$$$$$$$$$$$$$$$$$$$$%
\section{Background Estimation: Direct Integration}\label{hawc:direc_int}
%$$$$$$$$$$$$$$$$$$$$$$$$$$$$$$$$$$$$$$$$$$$$$$$$$$$$$$$$$$$$$$$$$$$$$$$$$$$$$$$$$$$%

The ratio of cosmic rays to gamma rays can be as high as 10,000 to 1, depending on the energy.
At HAWC, we confront a significant challenge even after gamma/hadron cuts: our gamma-ray data is still inundated with cosmic-ray events.
To tackle this, we rely on the direct integration method developed by Milagro \cite{Milagro_crab}.
This method capitalizes on the cosmic rays' isotropic nature resulting from their deflection by interstellar magnetic fields.

The direct integration method estimates background events by integrating over a stable two-hour period of detector operation.
The expected number of background events at a particular sky coordinate ($\phi, \theta$) is determined by integrating the normalized detector's efficiency with the all-sky event rate:
\directInt
Here, $E(\text{ha}, \theta)$, represents the detector's efficiency, which varies with local coordinates (hour angle and declination).
$R(t)$ is the event rate as a function of time \cite{Milagro_crab}.

Our background estimation is expected to falter in high-energy ranges where cosmic-ray events are less frequent due to enhanced gamma/hadron discrimination. Sparsity in our background and data also arise at the limits of HAWC's sensitivity and during short-term analyses of transient events.
HAWC addresses these issues by using a pixel size of 0.5$^\circ$ in our direct integration to maintain robustness in our estimation \cite{Abeysekara_2017,wcd_Sensitivity}.
In constructing the background model, it's crucial to exclude areas of the sky with known gamma-ray sources.
Regions containing the Crab Nebula, Mrk 421, Mrk 501, and the Galactic Plane are masked to prevent their significant gamma-ray signals from biasing our background estimate \cite{Abeysekara_2017}.


%%%%%%%%%%%%%%%%%%%%%%%%%%%%%%%%%%%%%%%%%%%%%%%%%%%%%%%%%%%%%%%%%%%%%%%%%%%%%%%%%%%%%
\chapter{Glory Duck: Multi-wavelength Search for Dark Matter Annihilation Towards Dwarf Spheroidal Galaxies}\label{sec:glory_duck}
%%%%%%%%%%%%%%%%%%%%%%%%%%%%%%%%%%%%%%%%%%%%%%%%%%%%%%%%%%%%%%%%%%%%%%%%%%%%%%%%%%%%%
%%%%%%%%%%%%%%%%%%%%%%%%%%%%%%%%%%%%%%%%%%%%%%%%%%%%%%%%%%%%%%%%%%%%%%%%%%%%%%%%%%%%%
\section{Introduction} \label{sec:gd_intro}
%%%%%%%%%%%%%%%%%%%%%%%%%%%%%%%%%%%%%%%%%%%%%%%%%%%%%%%%%%%%%%%%%%%%%%%%%%%%%%%%%%%%%

\begin{figure}[h]
    \centering{
        \includegraphics[scale=0.35]{figures/glory_duck/gd_motivation.png}
    }
    \caption{Sensitivities of five gamma-ray experiments compared to percentages of the Crab nebula's emission and dark matter annihilation. Solid lines present estimated sensitivities to power law spectra for each experiment. Green lines are Fermi-LAT sensitivities where lighter green is the sensitivity to the galactic center and dark green is its sensitivity to higher declinations. Orange, red, and purple solid lines represent the MAGIC, HESS, and VERITAS 50 hour sensitivities respectively. The maroon and brown lines are the HAWC 1 year and 5 year sensitivities. Across four decades of gamma-ray energy, these experiments have similar sensitivities on the order $10^{-12}$ erg cm$^{-2}$s$^{-1}$. The dotted lines are estimated dark matter fluxes assuming $m_{\chi} = 5$~TeV DM annihilating to bottom quarks (red), tau leptons (blue), and W bosons (green). Faded gray lines outline percentage flux of the Crab nebula. Figure is an augmented version of \cite{2020Galax...8...25R}}
    \label{fig:gd_motivation}
\end{figure}

The field of astrophysics now has several instruments and observatories sensitive to high energy gamma rays.
The energy sensitivity for the modern gamma-ray program spans many orders of magnitude.
\Cref{fig:gd_motivation} demonstrates these comparable sensitivities across energies for five experiments: Fermi-LAT, HAWC, HESS, MAGIC, and VERITAS.

Each of the five experiments featured in \cref{fig:gd_motivation} have independently searched for DM annihilation from dwarf spheroidal galaxies (dSph) and set limits on annihilation cross-section of WIMPs.
Intriguingly, there are regions of substantial overlap in their energy sensitivities.
This clearly motivates an analysis that combines data from these five.
Each experiment has unique gamma-ray detection methods and their weaknesses and strengths can be leveraged with each other.
The HAWC gamma-ray observatory is extensively introduced in \cref{sec:hawc}, so it is not introduced here.
A brief description of the remaining experiments are in the following paragraphs.

The Large Area Telescope (LAT) is a pair conversion telescope mounted on the NASA Fermi satellite in orbit $\sim$550 km above the Earth \cite{FermiLAT}.
LAT's field of view covers about 20\% of the sky and sweeps the whole sky approximately every 3 hours.
LAT's energy sensitivity ranges from 20 MeV up to 1 TeV.
Previous DM searches towards dSphs using Fermi-LAT are published in \cite{FermiLAT:dm1} and \cite{FermiLAT:dm2}.

\sloppy The High Energy Spectroscopic System (HESS), Major Atmospheric Gamma Imaging
Cherenkov (MAGIC), and Very Energetic Radiation Imaging Telescope Array System (VERITAS) are arrays of Imaging Atmospheric Cherenkov Telescopes (IACT).
These telescopes observe the Cherenkov light emitted from gamma-ray showers in the Earth's atmosphere.
The field of view for these telescopes is no larger than $5\degree$ with energy sensitivities ranging from ~ 30 GeV up to 100 TeV \cite{HESS,MAGIC,VERITAS}.
IACTs are able to make precise observations in selected regions of the sky, however can only be operated in ideal dark conditions.
HESS's  observations of the dwarves Sculptor and Carina were between January 2008 and December 2009.
HESS's observations of Coma Berenices were taken from 2010 to 2013, and Fornax was observed in 2010 \cite{HESS:dm_sculptor_carina,HESS:dm_dwarves,HESS:dm_gamma_lines}.
MAGIC provided deep observations of Segue1 between 2011 and 2013 \cite{MAGIC:dm_segue1}.
MAGIC also provides data for three additional dwarves: Coma Berenices, Draco, and Ursa Major II where observations were made in: January - June 2019 \cite{MAGIC:dm_comab_draco}, March - September 2018 \cite{MAGIC:dm_comab_draco}, and 2014 - 2016 \cite{MAGIC:dm_uma2} respectively.
VERITAS provided data for Boötes I, Draco, Segue 1, and Ursa Minor from 2009 to 2016 \cite{VERITAS:dm_dwarves}.

This chapter presents the Glory Duck analysis, the name given for the search for dark matter annihilation from dSph by combining data from the five gamma-ray observatories: Fermi-LAT, HAWC, HESS, MAGIC, and VERITAS.
Specifically, the methods in analysis and modeling are presented for the HAWC gamma-ray observatory.
This work will be published in the Journal of Cosmology and Astroparticle Physics and was presented at the International Cosmic Ray Conference in 2019, 2021, and 2023 \cite{glory_duck:ICRC2019,glory_duck:ICRC2021,glory_duck:ICRC2023}.

%%%%%%%%%%%%%%%%%%%%%%%%%%%%%%%%%%%%%%%%%%%%%%%%%%%%%%%%%%%%%%%%%%%%%%%%%%%%%%%%%%%%%
\section{Dataset and Background}\label{sec:gd_databgd}
%%%%%%%%%%%%%%%%%%%%%%%%%%%%%%%%%%%%%%%%%%%%%%%%%%%%%%%%%%%%%%%%%%%%%%%%%%%%%%%%%%%%%

This section enumerates the data analysis and background estimation methods used for HAWC's study of dSphs.
\Cref{sec:gd_data,sec:gd_tools} are most useful for fellow HAWC collaborators looking to replicate the Glory Duck analysis.

%%%%%%%%%%%%%%%%%%%%%%%%%%%%%%%%%%%%%%%%%%%%%%%
\subsection{Itemized HAWC files}\label{sec:gd_data}
%%%%%%%%%%%%%%%%%%%%%%%%%%%%%%%%%%%%%%%%%%%%%%%
These files are only available within HAWC's internal documentation.
They are not meant for public access, and are presented here so that HAWC collaborators can reproduce results accurately.

\begin{itemize}
    \item Detector Response: \url{response\_aerie\_svn\_27754\_systematics\_best\_mc\_test\_nobroadpulse
    _10pctlogchargesmearing\_0.63qe\_25kHzNoise\_run5481\_curvature0\_index3.root}
    \item Data Map: maps-20180119/liff/maptree\_1024.root
    \item Spectral Dictionary: DM\_CirrelliSpectrum\_dict\_gammas.npy
    \item Analysis wiki: \url{https://private.hawc-observatory.org/wiki/index.php/Glory_Duck_Multi-Experiment_Dark_Matter_Search}
\end{itemize}

% %%%%%%%%%%%%%%%%%%%%%%%%%%%%%%%%%%%%%%%%%%%%%%%
\subsection{Software Tools and Development}\label{sec:gd_tools}
%%%%%%%%%%%%%%%%%%%%%%%%%%%%%%%%%%%%%%%%%%%%%%%

This analysis was performed using HAL and 3ML \cite{Abeysekara_2017, vianello2015multimission} in Python version 2.
I built software to implement the \emph{A Poor Particle Physicist Cookbook for Dark Matter Indirect Detection} (PPPC) \cite{Cirelli_2011} DM spectral model and dSphs spatial model from \cite{Geringer_Sameth_2015} for HAWC analysis.
A NumPy version of this dictionary was made for both Py2 and Py3.
The code base for creating this dictionary is linked on my GitLab sandbox:

\begin{itemize}
    \item Py2: \href{https://gitlab.com/hawc-observatory/sandboxes/salaza82/glory-duck-hawc/-/tree/master/GD_spectrum}{Dictionary Generator (Deprecated)}
    \item Py3: \href{https://gitlab.com/hawc-observatory/sandboxes/salaza82/pppc2dict}{PPPC2Dict}
\end{itemize}

The analysis was performed using the $f_{\textrm{hit}}$ framework and described in the HAWC Crab paper \cite{Abeysekara_2017} and \cref{sec:hawc_fhit}.
The Python2 NumPy dictionary file for gamma-ray final states is dmCirSpecDict.npy.
The corresponding Python3 file is DM\_CirrelliSpectrum\_dict\_gammas.npy.
These files can also be used for decay channels and the PPPC describes how in \cite{Cirelli_2011}.
All other software used for data analysis, DM profile generation, and job submission to SLURM are also kept in my sandbox for \href{https://gitlab.com/hawc-observatory/sandboxes/salaza82/glory-duck-hawc}{the Glory Duck} project.

%%%%%%%%%%%%%%%%%%%%%%%%%%%%%%%%%%%%%%%%%%%%%%%
\subsection{Data Set and Background Description} \label{sec:gs_data_bkgd}
%%%%%%%%%%%%%%%%%%%%%%%%%%%%%%%%%%%%%%%%%%%%%%%

The HAWC data maps used for this analysis contain 1017 days of data between runs 2104 (2014-11-26) and 7476 (2017-12-20).
They were generated from pass 4.0 reconstruction.
The analysis is performed using the $f_{hit}$ energy binning scheme with bins (1-9) similar to what was done for the Crab and previous HAWC dSph analysis \cite{Abeysekara_2017,Albert_2018}.
Bin 0 was excluded as it has substantial hadronic contamination and poor angular resolution.

This analysis was done on dSphs because of their large DM mass content relative to baryonic mass.
We consider the following to estimate the background for this study.

\begin{itemize}
    \item The dSphs' angular extent are small relative to HAWC's spatial resolution, so the analysis is not sensitive to large or small scale anisotropies.
    \item The dSphs used in this analysis are off the galactic plane and therefore not contaminated by diffuse emission from the galaxy.
    \item The dSphs are baryonically faint relative to their expected dark matter content and are not expected to contain high energy gamma-ray sources.
\end{itemize}

Therefor we make no additional assumptions on the background from our sources and use HAWC's standard direct integration method for background estimation \cite{Abeysekara_2017}.
The largest background under this consideration is from an isotropic flux of cosmic rays.
The contamination of this hadronic flux is worse at lower energies where HAWC's gamma/hadron is discrimination worse.
It is possible for gamma rays from DM annihilation to scatter in transit to HAWC via Inverse Compton Scattering (ICS).
This was investigated and its impact on the flux is negligible.
Supporting information on this is in \cref{sec:gd_ics}

%%%%%%%%%%%%%%%%%%%%%%%%%%%%%%%%%%%%%%%%%%%%%%%%%%%%%%%%%%%%%%%%%%%%%%%%%%%%%%%%%%%%%
\section{Analysis}\label{sec:gd_analysis}
%%%%%%%%%%%%%%%%%%%%%%%%%%%%%%%%%%%%%%%%%%%%%%%%%%%%%%%%%%%%%%%%%%%%%%%%%%%%%%%%%%%%%

The expected differential photon flux from DM-DM annihilation to standard model
particles, $d\Phi_{\gamma}/dE_{\gamma}$ is described by the familiar equation.
\iddmannilation

Where \sv~is the velocity weighted annihilation cross-section.
$\frac{dN}{dE}$ is the expected differential number of photons produced at each energy per annihilation.
$m_\chi$ is the rest mass of the supposed DM particle.
$J$ is the astrophysical J-factor, over solid angle, $\Omega$, and is defined as
\jfactor
$\rho_{\chi}$ is the DM density.
$l$ is the distance to the source from Earth.
$r$ is the radial distance from the center of the source.
$\theta'$ is the half angle defining a cone containing the DM source.
How each component is synthesized and considered for HAWC's analysis is presented in the following sections.
\cref{sec:gd_particlephysics} presents the particle physics model for DM annihilation.
\cref{sec:gd_spatialmodel} presents the spatial distributions built for each dSph.

%%%%%%%%%%%%%%%%%%%%%%%%%%%%%%%%%%%%%%%%%%%%%%%%%%
\subsection{$\frac{dN_\gamma}{dE_\gamma}$ - Particle Physics Component}\label{sec:gd_particlephysics}
%%%%%%%%%%%%%%%%%%%%%%%%%%%%%%%%%%%%%%%%%%%%%%%%%%

For these spectra, we import the PPPC with Electroweak (EW) corrections \cite{Cirelli_2011}.
Public versions of the imported tables are provided by the \href{http://www.marcocirelli.net/PPPC4DMID.html}{authors online}.
The spectrum is implemented as a model script in astromodels for 3ML.
The EW corrections were previously not considered for HAWC and are significant for DM annihilating to EW coupled SM particles such as all leptons, and the $\gamma$, $Z$, and $W$ bosons \cite{Albert_2018}.
\Cref{fig:ew_vs_noew} demonstrates the significance of EW corrections for W boson annihilation.
Across EW SM channels, the gamma-ray spectra become harder than spectra without EW corrections.
Tables from the PPPC were reformatted into Python NumPy dictionaries for collaboration-wide use.
A class in astromodels was developed to include the EW correction from the PPPC and is aptly named PPPCSpectra within DM\_models.py.

\begin{figure}[t]
\centering{
    \includegraphics[scale=0.8]{figures/glory_duck/EW_vs_NoEW.png}
}
\caption{Effect of Electroweak (EW) corrections on expected DM annihilation spectrum for $\chi\chi \rightarrow W^-W^+$. Solid lines are spectral models that consider EW corrections. Dash-dot lines are spectral models without EW corrections. Red lines are models for $M_\chi = 1$ TeV. Blue lines represent models for $M_\chi = 100$ TeV. All models are sourced from the PPPC4DMID \cite{Cirelli_2011}.}
\label{fig:ew_vs_noew}
\end{figure}

%%%%%%%%%%%%%%%%%%%%%%%%%%%%%%%%%%%%%%%%%%%%%%%%%%
\subsection{\J - Astrophysical Component}\label{sec:gd_spatialmodel}
%%%%%%%%%%%%%%%%%%%%%%%%%%%%%%%%%%%%%%%%%%%%%%%%%%

The J-factor profiles for each source are imported from Geringer-Sameth's 2015 publication (referred to with \GS) \cite{Geringer_Sameth_2015}.
\GS fits the Zhao DM profile to the dSphs which has a DM density described as \cite{Zhao:1995cp}
\zhaoProfile
$R_s$ is the scale radius and free parameter in the model.
$\gamma$ is the logarithmic slope in the region $r << R_s$.
$\beta$ is the logarithmic slope in the region $r >> R_s$.
$\alpha$ is known as the sharpness of transition where $r \approx R_S$.
The classic Navarro-Frenk-White DM distribution \cite{NFWProfile} (NFW) can be retrieved from Zhao by fixing $(\alpha, \beta, \gamma) = (1,3,1)$.

\GS best fits were pulled from the publication as $\J(\theta)$, where $\theta$ is the angular separation from the center of the source.
HAWC requires maps in terms of $\frac{d\J}{d\Omega}$.
The description on how this was done follows.

First, convert the angular distances to solid angles
\angleTOsolidangle
which reduces with a small angle approximation to $\pi \theta^2$.
Next, the central difference for both the $\Delta \J$ and $\Delta \Omega$ value were calculated from the discretized $\J(\theta)$ with the central difference stencil:
\centerDiff
Where $\phi$ is either $\Omega$ or \J.
These were done separately in case the grid spacing in $\theta$~was not uniform.
Finally, these lists are divided so that we are left with an approximation of the $d\J/d\Omega$~profile that is a function of $\theta$.
Admittedly, this is an approximation method for the map which introduces small errors compared to the true profile estimate.
This was checked as a systematic against the author's profiling of the spatial distribution and is documented in \cref{sec:gd_jfacintegration}.

With $\frac{d\J}{d\Omega}(\theta, \phi)$, a map is generated, first by filling in the north-east quadrant of the map.
This quadrant is then reflected twice, vertically then horizontally, to fill the full map.
Maps are then normalized by dividing the discrete 2D integral of the map.
The 2D integral was a simple height of bins, Newton's integral:
\newtonIntegral
Here, $p$ is the side length of a HEALpix pixel used for the map.
$\theta_i$ and $\phi_i$ are the discrete coordinates whose origin is the center of the modeled dSph.
These maps are HEALpix maps with NSIDE 16384 and saved in the ".fits" format.
The hyper fine resolution was selected to better preserve the total expected counts after integrating \cref{eq:id_dm_flux} with the detector response.

\begin{figure}
    \centering{
    \includegraphics[scale=0.84]{figures/glory_duck/hawc/Segue1_J_plot.pdf}
    \includegraphics[scale=0.84]{figures/glory_duck/hawc/ComaBerenices_J_plot.pdf}
    }
    \caption{Dark Matter density ($\frac{d\J}{d\Omega}$) maps for Segue1 (top) and Coma Berenices (bottom). Origin is centered on the specific dwarf spheroidal galaxies (dSph). X and Y axes are the angular separation from the center of the dwarf. Profile is truncated at the scale radius. Plots of the remaining 11 dSph HAWC studied are provided in \cref{fig:apx_gd_spatialmodels}.}\label{fig:gd_spatialmodel}
\end{figure}

Another DM spatial distribution model from Bonnivard (\B) \cite{Bonnivard:2014kza} was used for the Glory Duck study.
However, to save computational time, limits from \GS were scaled to \B instead of each experiment performing a full study a second time.
How these models compare is demonstrated for each dSph in \cref{fig:comparison_J_1,fig:comparison_J_2}.
Examples of the two most impactful dSphs derived from \GS, Segue1 and Coma Berenices are featured in \cref{fig:gd_spatialmodel}

%%%%%%%%%%%%%%%%%%%%%%%%%%%%%%%%%%%%%%%%%%%%%%%%%%
\subsection{Source Selection and Annihilation Channels}\label{sec:gd_srcs_y_chan}
%%%%%%%%%%%%%%%%%%%%%%%%%%%%%%%%%%%%%%%%%%%%%%%%%%

Many of the dSphs presented in HAWC's previous dSph DM search are used \cite{Albert_2018}.
HAWC's sources for the Glory Duck analysis include Boötes I, Coma Berenices, Canes Venatici I + II, Draco, Hercules, Leo I, II, + IV, Segue 1, Sextans, and Ursa Major I + II.
A full description of all sources used in Glory Duck are found in \Cref{tab:gd_J_factor}.
Triangulum II was excluded from the Glory Duck analysis because of large uncertainties in its \J factor.
Ursa Minor was excluded from HAWC's contribution to the combination because the source extension model extended Ursa Minor beyond HAWC's field of view.
Ursa Minor was not expected to contribute significantly to the combined limit, so work was not invested in a solution to include Ursa Minor.

This analysis improves on the previous HAWC dSph paper \cite{Albert_2018} in the following ways.
Previously, the dSphs were treated and implemented as point sources.
For this analysis, dSphs are modeled and treated as extended source.
The impact of this change with respect to the upper limit is source dependent and is explored in \Cref{sec:gd_ext_limitvs_ptsrc}.
Previously, the particle physics model used for gamma-ray spectra from DM annihilation did not have EW corrections where the PPPC includes them.
Finally, the gamma-ray ray dataset is much larger.
The study performed here analyzes over 1000 days of data compared to 507.

The SM annihilation channels probed for the Glory Duck combination include $b\bar{b}$, $e\bar{e}$, $\mu\bar{\mu}$, $\tau\bar{\tau}$, $t\bar{t}$, $W^+W^-$, and $ZZ$.
A summary of all sources, with a description of each experiments' sensitivity to the source, is provided in \Cref{tab:gd_tabSummary}.

\clearpage\pagestyle{lscape}
\afterpage{%
    \newgeometry{margin=2.5cm} % modify this if you need even more space
    \begin{landscape}
%
\vspace*{\fill}
\begin{table}
\caption{Summary of dSph observations by each experiment used in this work. A `-' indicates the experiment did not observe the dSph for this study. For Fermi-LAT, the exposure at 1~GeV is given. For HAWC, $|\Delta\theta|$ is the absolute difference between the source declination and HAWC latitude. HAWC is more sensitive to sources with smaller $|\Delta\theta|$. For IACTs, we show the zenith angle range, the total exposure, the energy range, the angular radius $\theta$  of the signal or ON region, the ratio of exposures between the background-control (OFF) and signal (ON) regions ($\tau$), and the significance of gamma-ray excess in standard deviations, $\sigma$.}
\centering
{\begin{tabular}{c ? c ? c ? c c c c c c c }
\hline
\hline
& \multicolumn{1}{c?}{Fermi-LAT}  & \multicolumn{1}{c?}{HAWC} & \multicolumn{7}{c}{H.E.S.S, MAGIC, VERITAS} \\
\hline
Source name & Exposure ($10^{11}$ s m$^2$) & $|\Delta\theta|$ (\degree) & IACT & Zenith (\degree) & Exposure (h) & Energy range (GeV) & $\theta$ (\degree) & $\tau$ & S ($\sigma$) \\
\hline\hline
Bo\"{o}tes I            & $2.6$ & $4.5$ & VERITAS  & $15-30$ & $14.0$ & $100 $-$ 41000$ & $0.10$ & $8.6$ & $-1.0$  \\
Canes Venatici I        & $2.9$ & $14.6$ & $-$ & $-$ & $-$ & $-$ & $-$ & $-$ & $-$  \\
Canes Venatici II       & $2.9$ & $15.3$ & $-$ & $-$ & $-$ & $-$ & $-$ & $-$ & $-$  \\
Carina                  & $3.1$ & $-$ & H.E.S.S. & $27-46$ & $23.7$ & $310 - 70000$ & $0.10$ & $18.0$ & $-0.3$  \\
\hdashline
\multirow{2}{*}{Coma Berenices}          & \multirow{2}{*}{$2.7$} & \multirow{2}{*}{$4.9$} & H.E.S.S. & $47-49$ & $11.4$ & $550 - 70000$ & $0.10$ & $14.4$ & $-0.4$  \\
                                         & & & MAGIC & $5 - 37 $ & $49.5$ & $60 - 10000$ & $0.17$ & $1.0$ & $-$\\
\hdashline
\multirow{2}{*}{Draco}                   & \multirow{2}{*}{$3.8$} & \multirow{2}{*}{$38.1$} & MAGIC & $29 - 45$ & $52.1$ & $70 - 10000$ & $0.22$ & $1.0$ & $-$  \\
                                         & & & VERITAS & $25-40$ & $49.8$ & $120 - 70000$ & $0.10$ & $9.0$ & $-1.0$ \\
\hdashline
Fornax                  & $2.7$ & $-$ & H.E.S.S. & $11-25$ & $6.8$ & $230 - 70000$ & $0.10$ & $45.5$ & $-1.5$ \\
Hercules                & $2.8$ & $6.3$ & $-$ & $-$ & $-$ & $-$ & $-$ & $-$ & $-$  \\
Leo I                   & $2.5$ & $6.7$ & $-$ & $-$ & $-$ & $-$ & $-$ & $-$ & $-$  \\
Leo II                  & $2.6$ & $3.1$ & $-$ & $-$ & $-$ & $-$ & $-$ & $-$ & $-$  \\
Leo IV                  & $2.4$ & $19.5$ & $-$ & $-$ & $-$ & $-$ & $-$ & $-$ & $-$ \\
Leo V                   & $2.4$ & $-$ & $-$ & $-$ & $-$ & $-$ & $-$ & $-$ & $-$  \\
Leo T                   & $2.6$ & $-$ & $-$ & $-$ & $-$ & $-$ & $-$ & $-$ & $-$  \\
Sculptor                & $2.7$ & $-$ & H.E.S.S. & $10-46$ & $11.8$ & $200 - 70000$ & $0.10$ & $19.8$ & $-2.2$  \\
\hdashline
\multirow{2}{*}{Segue I}& \multirow{2}{*}{$2.5$} & \multirow{2}{*}{$2.9$} & MAGIC & $13-37$ & $158.0$ & $60 - 10000$ & $0.12$ & $1.0$ & $-0.5$ \\
                        & & & VERITAS & $15-35$ & $92.0$ & $80 - 50000$ & $0.10$ & $7.6$ & $0.7$ \\
\hdashline
Segue II                & $2.7$ & $-$ & $-$ & $-$ & $-$ & $-$ & $-$ & $-$ & $-$ \\
Sextans                 & $2.4$ & $20.6$ & $-$ & $-$ & $-$ & $-$ & $-$ & $-$ & $-$  \\
Ursa Major I            & $3.4$ & $32.9$ & $-$ & $-$ & $-$ & $-$ & $-$ & $-$ & $-$  \\
Ursa Major II           & $4.0$ & $44.1$ & MAGIC & $35-45$ & $94.8$ & $120 - 10000$ & $0.30$ & $1.0$ & $-2.1$  \\
Ursa Minor              & $4.1$ & $-$ &  VERITAS & $35-45$ & 60.4 & $160-93000$ & 0.10 & 8.4 & $-0.1$ \\
\hline
\end{tabular}}
\label{tab:tabSummary}
\end{table}
\vspace*{\fill}
%
\end{landscape}
\restoregeometry
}
%

\pagestyle{plain}

\clearpage
\begin{table}[h!]
% \captionsetup{font=small}
\centering
    \caption{Summary of the relevant properties of the dSphs used in the present work. Column 1 lists the dSphs. Columns 2 and 3 present their heliocentric distance and galactic coordinates, respectively. Columns 4 and 5 report the \J-factors of each source given from the \GS and \B independent studies and their estimated $\pm 1\sigma$ uncertainties. The values $\log_{10}J$~(\GS set) \cite{Geringer_Sameth_2015} correspond to the mean \J-factor values  for a source extension truncated at the outermost observed star. The values $\log_{10}J$~(\B set) \cite{Bonnivard:2014kza} are provided for a source extension at the tidal radius of each dSph. \textbf{Bolded sources are within HAWC's field of view and provided to the Glory Duck analysis.}}
    {\begin{tabular}{ccccc}
    \hline
    \hline
    \CellTopTwo{}
    Name & Distance & $l, b$ & $\log_{10}J$~(\GS set) & $\log_{10}J$~(\B set)\\
    & \scriptsize{(kpc)} &  \scriptsize{($\degree$)} & \scriptsize{$\log_{10}(\mathrm{GeV}^2 \mathrm{cm}^{-5}\mathrm{sr})$} & \scriptsize{$\log_{10}(\mathrm{GeV}^2 \mathrm{cm}^{-5}\mathrm{sr})$}  \\
    \hline
    \CellTopTwo{}
    \textbf{Bo\"otes} I & $66$ & $358.08,\: 69.62$ & $18.24^{+0.40}_{-0.37}$ & $18.85^{+1.10}_{-0.61}$  \\
    \CellTopTwo{}
    \textbf{Canes Venatici I} & $218$ & $74.31,\: 79.82$ & $17.44^{+0.37}_{-0.28}$ & $17.63^{+0.50}_{-0.20}$  \\
    \CellTopTwo{}
    \textbf{Canes Venatici II} & $160$ & $113.58,\: 82.70$ & $17.65^{+0.45}_{-0.43}$ & $18.67^{+1.54}_{-0.97}$  \\
    \CellTopTwo{}
    Carina & $105$ & $260.11,\: -22.22$ & $17.92^{+0.19}_{-0.11}$ & $18.02^{+0.36}_{-0.15}$ \\
    \CellTopTwo{}
    \textbf{Coma Berenices} & $44$ & $241.89,\: 83.61$ & $19.02^{+0.37}_{-0.41}$ & $20.13^{+1.56}_{-1.08}$  \\
    \CellTopTwo{}
    \textbf{Draco} & $76$ & $86.37,\: 34.72$ & $19.05^{+0.22}_{-0.21}$ &  $19.42^{+0.92}_{-0.47}$  \\
    \CellTopTwo{}
    Fornax & $147$ & $237.10,\: -65.65$ & $17.84^{+0.11}_{-0.06}$ &  $17.85^{+0.11}_{-0.08}$  \\
    \CellTopTwo{}
    \textbf{Hercules} & $132$ & $28.73,\: 36.87$ & $16.86^{+0.74}_{-0.68}$ &  $17.70^{+1.08}_{-0.73}$  \\
    \CellTopTwo{}
    \textbf{Leo I} & $254$ & $225.99,\: 49.11$ & $17.84^{+0.20}_{-0.16}$ &  $17.93^{+0.65}_{-0.25}$  \\
    \CellTopTwo{}
    \textbf{Leo II} & $233$ & $220.17,\: 67.23$ & $17.97^{+0.20}_{-0.18}$ &  $18.11^{+0.71}_{-0.25}$  \\
    \CellTopTwo{}
    \textbf{Leo IV} & $154$ & $265.44,\: 56.51$ & $16.32^{+1.06}_{-1.70}$ & $16.36^{+1.44}_{-1.65}$  \\
    \CellTopTwo{}
    Leo V & $178$ & $261.86,\: 58.54$ & $16.37^{+0.94}_{-0.87}$ & $16.30^{+1.33}_{-1.16}$  \\
    \CellTopTwo{}
    Leo T & $417$ & $214.85,\: 43.66$ & $17.11^{+0.44}_{-0.39}$ & $17.67^{+1.01}_{-0.56}$  \\
    \CellTopTwo{}
    Sculptor & $86$ & $287.53,\: -83.16$ & $18.57^{+0.07}_{-0.05}$ &  $18.63^{+0.14}_{-0.08}$  \\
    \CellTopTwo{}
    \textbf{Segue I} & $23$ & $220.48,\: 50.43$ & $19.36^{+0.32}_{-0.35}$ &  $17.52^{+2.54}_{-2.65}$  \\
    \CellTopTwo{}
    Segue II & $35$ & $149.43,\: -38.14$ & $16.21^{+1.06}_{-0.98}$ & $19.50^{+1.82}_{-1.48}$  \\
    \CellTopTwo{}
    \textbf{Sextans} & $86$ & $243.50,\: 42.27$ & $17.92^{+0.35}_{-0.29}$ &  $18.04^{+0.50}_{-0.28}$  \\
    \CellTopTwo{}
    \textbf{Ursa Major I} & $97$ & $159.43,\: 54.41$ & $17.87^{+0.56}_{-0.33}$ & $18.84^{+0.97}_{-0.43}$  \\
    \CellTopTwo{}
    \textbf{Ursa Major II} & $32$ & $152.46,\: 37.44$ & $19.42^{+0.44}_{-0.42}$ & $20.60^{+1.46}_{-0.95}$  \\
    \CellTopTwo{}
    Ursa Minor & $76$ & $104.97,\: 44.80$ & $18.95^{+0.26}_{-0.18}$ & $19.08^{+0.21}_{-0.13}$  \\
    \hline
    \hline
    \CellTopTwo{}
\end{tabular}} \label{tab:gd_J_factor}
\end{table}

%%%%%%%%%%%%%%%%%%%%%%%%%%%%%%%%%%%%%%%%%%%%%%%%%%%%%%%%%%%%%
\section{Likelihood Methods} \label{sec:gd_ll_methods}
%%%%%%%%%%%%%%%%%%%%%%%%%%%%%%%%%%%%%%%%%%%%%%%%%%%%%%%%%%%%%

%%%%%%%%%%%%%%%%%%%%%%%%%%%%%%%%%%%%%%%%%%%%%%%%%%%%%%%%%%%%%
\subsection{HAWC Likelihood}\label{sec:gd_hawc_llh}
%%%%%%%%%%%%%%%%%%%%%%%%%%%%%%%%%%%%%%%%%%%%%%%%%%%%%%%%%%%%%

For every analysis bin in energy, $f_{hit}$ bins (1-9), and location, we can expect $N$ signal events and $B$ background events.
The expected number of excess signal events from dark matter annihilation, $S$, is estimated by convolving \Cref{eq:id_dm_flux} with HAWC's energy response and pixel point spread functions.
The test statistic (TS) is computed with the log-likelihood ratio test,
\gdTS
where $\mathcal{L}_0$~is the null hypothesis, or no DM emission, likelihood.
$\mathcal{L}^\mathrm{max}$~is the best fit signal hypothesis where \sv~maximizes the likelihood.
We calculate the likelihood of each source and model, assuming events are Poisson distributed, with
\hwcpsLLH
where $S_i$ is the sum of expected number of signal counts.
$B_i$ is the number of background counts observed.
$N_i$ is the total number of counts.

An upper limit on \sv~is calculated using the $95\%$ confidence level (CL).
For the CL, we define a parameter, $TS_{95}$, as
\gdHAWCCL
where the expected signal counts from a dSph is scaled by $\epsilon$.
$S_\mathrm{ref}$ is the expected number of excess counts in a bin for DM emission from a dSph with a corresponding annihilation cross-section, \sv.
We scan $\epsilon$ such that
\CLbyTS
HAWC's exclusive results are provided in \cref{sec:hawc_results}.

%%%%%%%%%%%%%%%%%%%%%%%%%%%%%%%%%%%%%%%%%%%%%%%%%%%%%%%%%%%%%
\subsection{Glory Duck Joint Likelihood}\label{sec:gd_joint_llh}
%%%%%%%%%%%%%%%%%%%%%%%%%%%%%%%%%%%%%%%%%%%%%%%%%%%%%%%%%%%%%

The joint likelihood for the 5-experiment combination was done similarly as \Cref{sec:gd_hawc_llh}.
We calculate upper limits on \sv~from the TS, \cref{eq:gd_TS}, and define the likelihood ratio more generally
\gdLHratio
$\bm{\mathcal{D}_{\mathrm{dSphs}}}$ is the totality of observations across experiments and dSphs.
$\bm{\nu}$ are the nuisance parameters which are the \J factors in this study.
$\widehat{\svtex}$ and $\hat{\bm{\nu}}$ are the respective estimate that maximize $\mathcal{L}$ globally.
Finally, $\hat{\hat{\bm{\nu}}}$ is the set of nuisance parameters that maximize $\mathcal{L}$ for a fixed value of \sv.

The \textit{complete} joint likelihood, $\mathcal{L}$ that encompasses all observations from all instruments and dSphs can be factorized into \textit{partial} functions for each dSph $l$ (with $\mathcal{L}_{\mathrm{dSph}, l}$) and its \J factor ($\mathcal{J}_l$):
\CompleteGDLLH
For this study, $N_{\mathrm{dSphs}}=20$ is the number of dSphs studied.
$\bm{\mathcal{D}_{l}}$ are the gamma-ray observations of dSph, $l$.
$\bm{\nu_l}$ are the nuisance parameters modifying the gamma-ray observations of dSph, $l$, but excludes $\mathcal{J}_l$.
$\mathcal{J}_l$ is the \J factor for dSph, $l$, as defined in \cref{eq:jfactor}, and it is a nuisance parameter whose value is unknown.
$\log_{10} J_{l,\mathrm{obs}}$ and $\sigma_{\log{J_l}}$ are obtained by fitting a log-normal function of $J_{l,\mathrm{obs}}$ to the posterior distribution of $J_{l}$~\cite{2015PhRvL.115w1301A}.
$\log_{10} J_{l,\mathrm{obs}}$ and $\sigma_{\log{J_l}}$ values are provided in \cref{tab:gd_J_factor}.
The term $\mathcal{J}_l$ constraining $J_l$ is written as:
\JLgdLLH
Both the \GS and \B sets of \J factors, displayed in \cref{tab:gd_J_factor}, are used in this analysis.
\Cref{eq:J_lkl} is also normalized, so it can also be interpreted as a probability density function (PDF) for $J_{l,\mathrm{obs}}$.
From \cref{eq:id_dm_flux}, we can also see that \sv~and $\J_l$ are degenerate when computing $\mathcal{L}_{\mathrm{dSph},l}$.
Therefore, as noted in~\cite{MAGICFermi_combo}, it is sufficient to compute $\mathcal{L}_{\mathrm{dSph},l}$ versus \sv~for a fixed value of $J_{l}$.
We used $J_{l,\mathrm{obs}}$(\GS) reported in \cref{tab:gd_J_factor}, in order to perform the profile of $\mathcal{L}$ with respect to $J_l$.
The degeneracy implies that for any $J'_l \neq J_{l,\mathrm{obs}}$ (in practice in our case we used $J'_l = J_{l,\mathrm{obs}}$(\B) to compute results from a different set of \J factors):
\jfacTrick
which is a straightforward rescaling operation that reduces the computational needs of the profiling operation since:
\rescaleTrick
In addition, \cref{eq:Jfactor_trick} enables the combination of data from different gamma-ray instruments and observed dSphs via tabulated values of $\mathcal{L}_{\mathrm{dSph},l}$, or equivalently of $\lambda$ from~\cref{eq:gd_LLH_test} as was done in this work, versus \sv.
$\mathcal{L}_{\mathrm{dSph},l}$ is computed for a fixed value of $\J_l$ and profiled with respect to all instrumental nuisance parameters $\bm{\nu_l}$, these nuisance parameters are discussed in more detail below.
These values are produced by each detector independently and therefore there is no need to share sensitive low-level information used to produce them, such as event lists.
\Cref{fig:illustration_combination} illustrates the multi-instrument combination technique used in this study with a comparison of the upper limit on \sv~obtained from the combination of the observations of four experiments towards one dSph versus the upper limit from individual instruments.
It also shows graphically the effect of the \J-factor uncertainty on the combined observations.

\begin{figure}[t]
    \centering{
    \includegraphics[width=0.7\textwidth]{figures/glory_duck/comparison/Combined_exp_technique_20TeV_4exp_J_with_without_nuisance_v2.pdf}
    }
    \caption{Illustration of the combination technique showing a comparison between $-2\ln  \lambda$ provided by four instruments (colored lines) from the observation of the same dSph without any \J nuisance and their sum, $i.e.$ the resulting combined likelihood (thin black line). According to the test statistics of \cref{eq:gd_TS}, the intersection of the likelihood profiles with the line $-2\ln  \lambda$ = 2.71 indicates the 95\% C.L. upper limit on \sv. The combined likelihood (thin black line) shows a smaller value of upper limit on \sv than those derived by individual instruments. We also show how the uncertainties on the \J factor effects the combined likelihood and degrade the upper limit on \sv (thick black line). All likelihood profiles are normalized so that the global minimum $\widehat{\svtex}$ is 0. We note that each profile depends on the observational conditions in which a target object was observed. The sensitivity of a given instrument can be degraded and the upper limits less constraining if the observations are performed in non-optimal conditions such as a high zenith angle or a short exposure time.}
    \label{fig:illustration_combination}
\end{figure}

The \textit{partial} joint likelihood function for gamma-ray observations of each dSph ($\mathcal{L}_{\text{dSph},l}$) is written as the product of the likelihood terms describing the $N_{\mathrm{exp},l}$ observations performed with any of our observatories:
\GDjointLLH
where each $\mathcal{L}_{lk}$ term refers to an observation of the $l$-th dSph with associated $k$-th instrument responses.
$ N_{\mathrm{exp},l} $ varies from dSph to dSph and can be inferred from \cref{tab:gd_tabSummary}.

Each collaboration separately analyzes their data for $\bm{\mathcal{D}_{lk}}$ corresponding to dSph $l$ and gamma-ray detector $k$, using as many common assumptions as possible in the analysis.
HAWC's treatment was described earlier in \cref{sec:gd_hawc_llh} whereas the specifics of the remaining experiments is left to the publication.
We compute the values for the likelihood functions $\mathcal{L}_{lk}$ (see~\cref{eq:measurement_combination}) for a fixed value of $\J_l$ and profile over the rest of the nuisance parameters $\bm{\nu_{lk}}$.
Then, values of $\lambda$ from \cref{eq:gd_LLH_test} are computed as a function of \sv, and shared using a common format.
Results are computed for seven annihilation channels, $W^+W^-$, $ZZ$, $b\bar{b}$, $t\bar{t}$, $e^+e^-$, $\mu^+\mu^-$, and $\tau^+\tau^-$ over 62 $m_{\chi}$ values between 5 GeV and 100 TeV provided in~\cite{Cirelli_2011}.
The \sv range is defined between $10^{-28}$ and $10^{-18}\mathrm{cm}^{3}\cdot \mathrm{s}^{-1}$, with 1001 logarithmically spaced values.
The likelihood combination, i.e. \cref{eq:dSph_combination}, and profile over the \J-factor to compute the profile likelihood ratio $\lambda$, \cref{eq:gd_LLH_test}, are carried out with two different public analysis software packages, namely \textbf{gLike}~\cite{javier_rico_2021_4601451} and \textbf{LklCom}~\cite{tjark_miener_2021_4597500}, that provide the same results~\cite{2021arXiv211201818M}.

As mentioned previously, each experiment computes the $\mathcal{L}_{lk}$ from \cref{eq:gd_LLH_test} differently.
The remainder of this section highlights the differences in this calculation across the experiments.
Four experiments, namely \textit{Fermi}-LAT, H.E.S.S., HAWC and MAGIC, use a binned likelihood to compute the $\mathcal{L}_{lk}$.
For these experiments, for each observation $\bm{\mathcal{D}_{lk}}$ of a given dSph $l$ carried out using a given gamma-ray detector $k$, the binned likelihood function is:
\gdJointLLH
where $N_{\text{E'}}$ and $N_{\text{P'}}$ are the number of considered bins in reconstructed energy and arrival direction, respectively; $\mathcal{P}$ represents a Poisson PDF for the number of gamma-ray candidate events $ N_{lk,ij} $ observed in the $i$-th bin in energy and $j$-th bin in arrival direction, when the expected number is the sum of the expected mean number of signal events $ s_{ij} $ (produced by DM annihilation) and of background events $ b_{ij} $; $ \mathcal{L}_{lk,\bm{\nu}} $ is the likelihood term for the extra $ \bm{\nu_{lk}} $ nuisance parameters that vary from one instrument $k$ to another.
The expected counts for signal events $s_{ij}$ for a given dSph $l$ and detector $k$ is given by:
\gdExpectedNS
where $ E' $ and $ E $ are the reconstructed and true energies, $ P' $ and $ P $ the reconstructed and true arrival directions; $ E'_{\text{min},i} $, $ P'_{\text{min},j} $, $ E'_{\text{max},i} $, and $ P'_{\text{max},j} $ are their lower and upper limits of the $ i $-th energy bin and the $ j $-th arrival direction bin; $ T_{\text{obs}} $ is the (dead-time corrected) total observation time; $ t $ is the time along the observations; $ \text{d}^{2}\Phi/\text{d}E\text{d}\Omega $ is the DM flux in the source region (see \cref{eq:id_dm_flux});
and $ \text{IRF} \left( E', P' \mid E, P, t \right) $ is the IRF, which can be factorized as the product of the effective collection area of the detector $ A_{\mathrm{eff}} (E, P, t) $, the PDFs for the energy estimator $ f_{E} (E' \mid E,t) $, and arrival direction $ f_{P} (P' \mid E,P,t) $ estimators.
Note that for Fermi-LAT, HAWC, MAGIC, and VERITAS the effect of the finite angular resolution is taken into account through the convolution of $d\Phi/dE d\Omega$ with $f_{P}$ in \cref{eq:gd_expected_ns}, whereas in the cases of H.E.S.S. $f_{P}$ is approximated by a delta function.
This approximation has been made in order to maintain compatibility of the result with what has been previously published.
The difference introduced by this approximation is $<5\%$ for all considered dSphs.
A more comprehensive review of the differences between the analyses of different instruments can be found in~\cite{2020Galax...8...25R}.

%%%%%%%%%%%%%%%%%%%%%%%%%%%%%%%%%%%%%%%%%%%%%%%%%%%%%%%%%%%%%
\section{HAWC Results}\label{sec:hawc_results}
%%%%%%%%%%%%%%%%%%%%%%%%%%%%%%%%%%%%%%%%%%%%%%%%%%%%%%%%%%%%%

\begin{figure}[ht]
    \centering{\includegraphics[scale=0.21]{figures/glory_duck/hawc/Combined95_GD_bb.pdf}
    \includegraphics[scale=0.215]{figures/glory_duck/hawc/Combined95_GD_ee.pdf}
    \includegraphics[scale=0.215]{figures/glory_duck/hawc/Combined95_GD_ww.pdf}
    \includegraphics[scale=0.215]{figures/glory_duck/hawc/Combined95_GD_zz.pdf}
    \includegraphics[scale=0.215]{figures/glory_duck/hawc/Combined95_GD_tautau.pdf}
    \includegraphics[scale=0.215]{figures/glory_duck/hawc/Combined95_GD_mumu.pdf}
    \includegraphics[scale=0.215]{figures/glory_duck/hawc/Combined95_GD_tt.pdf}
    }
    \caption{HAWC 95\% confidence limits on \sv~versus DM mass for seven annihilation channels with \J-factors from \GS. Limits are shown for each source individually as colored, dashed lines. Combined limit represented with solid black line. The lower the limit, the stronger the constraint. We can see the combined limit is lower than any source individually.}
 \label{fig:hawc_combined_limit}
\end{figure}

Thirteen of the 20 dSphs considered for the Glory Duck analysis are within HAWC's field of view.
These dSph are analyzed for emission from DM annihilation according to the likelihood method described in \Cref{sec:gd_ll_methods}.
The 13 likelihood profiles are then stacked to synthesize a combined limit on the dark matter cross-section, \sv.
This combination is done for the 7 SM annihilation channels used in the Glory Duck analysis.
\Cref{fig:hawc_combined_limit} shows the combined limit for all annihilation channels with HAWC only observations.
We also perform 300 studies of Poisson trials on the background.
These trials are used to produce HAWC sensitivities with $\pm1\sigma$ and $\pm2\sigma$ uncertainty bands which were shared with the other collaborators for combination.
The results on fitting to HAWC's Poisson trials of the DM hypothesis is shown in \Cref{fig:hawc_brazil_band} for all the DM annihilation channels studied for Glory Duck.

No DM was found in HAWC observations.
HAWC's limits are dominated by the dSphs Segue1 and Coma Berenices.
The remaining 11 dSphs do not contribute significantly to the limit because they are at high zenith and/or have much smaller \J factors.
Even though some remaining dSphs have large \J factors, they are towards the edge of HAWC's field of view where HAWC analysis is less sensitive.

\begin{figure}[h]
    \centering{
        \includegraphics[scale=0.21]{figures/glory_duck/hawc/CombinedTS_data_bb_.pdf}
        \includegraphics[scale=0.21]{figures/glory_duck/hawc/CombinedTS_data_ee_.pdf}
        \includegraphics[scale=0.21]{figures/glory_duck/hawc/CombinedTS_data_mumu_.pdf}
        \includegraphics[scale=0.21]{figures/glory_duck/hawc/CombinedTS_data_tautau_.pdf}
        \includegraphics[scale=0.21]{figures/glory_duck/hawc/CombinedTS_data_ww_.pdf}
        \includegraphics[scale=0.21]{figures/glory_duck/hawc/CombinedTS_data_zz_.pdf}
        \includegraphics[scale=0.21]{figures/glory_duck/hawc/CombinedTS_data_tt_.pdf}
    }
    \caption{HAWC TS values for best fit \sv~versus $m_\chi$ for seven SM annihilation channels with \J factors from \GS. The solid black line shows the combined best fit TS values. The colored, dashed lines are the TS values for each of the 13 sources HAWC studied. The lower the limit, the stronger the constraint.}\label{fig:gd_HAWC_TS}
\end{figure}

\begin{figure}[h]
    \centering{
    \includegraphics[scale=0.21]{figures/glory_duck/hawc/GD_BrazilBand_bb.pdf}
    \includegraphics[scale=0.21]{figures/glory_duck/hawc/GD_BrazilBand_ee.pdf}
    \includegraphics[scale=0.21]{figures/glory_duck/hawc/GD_BrazilBand_mumu.pdf}
    \includegraphics[scale=0.21]{figures/glory_duck/hawc/GD_BrazilBand_tautau.pdf}
    \includegraphics[scale=0.21]{figures/glory_duck/hawc/GD_BrazilBand_ww.pdf}
    \includegraphics[scale=0.21]{figures/glory_duck/hawc/GD_BrazilBand_zz.pdf}
    \includegraphics[scale=0.21]{figures/glory_duck/hawc/GD_BrazilBand_tt.pdf}
    }
    \caption{HAWC Brazil bands at 95\% confidence level on \sv versus DM mass for seven annihilation channels with \J-factors from \GS \cite{Geringer-Sameth:2014yza}. The solid line represents the combined limit from 13 dSphs. The dashed line is the expected limit. The green band is the 68\% containment. The yellow band is the 95\% containment.}
\label{fig:hawc_brazil_band}
\end{figure}

\clearpage

%%%%%%%%%%%%%%%%%%%%%%%%%%%%%%%%%%%%%%%%%%%%%%%%%%%%%%%
\section{Glory Duck Combined Results}\label{sec:results}
%%%%%%%%%%%%%%%%%%%%%%%%%%%%%%%%%%%%%%%%%%%%%%%%%%%%%%%

The crux of this analysis is that HAWC's results are combined with 4 other gamma-ray observatories: Fermi-LAT, H.E.S.S., MAGIC, and VERITAS.
\sloppy No significant DM emission was observed by any of the five instruments.
We present the upper limits on \sv assuming seven independent DM self annihilation channels, namely $W^+W^-$, $Z^+Z^-$, $b\bar{b}$, $t\bar{t}$, $e^+e^-$, $\mu^+\mu^-$, and $\tau^+\tau^-$.
The 68\% and 95\% containment bands are produced from 300 Poisson realizations of the null hypothesis corresponding to each of the combined datasets.
These 300 realizations are combined identically to dSph observations.
The containment bands and the median are extracted from the distribution of resulting limits on the null hypothesis.
These 300 realizations are obtained either by fast simulations of the OFF observations, for H.E.S.S., MAGIC, VERITAS, and HAWC, or taken from real observations of empty fields of view in the case of Fermi-LAT~\cite{2015PhRvL.115w1301A,Fermi-LAT:2016uux,2021PhRvD.103l3005D}.

The obtained limits are shown in \Cref{fig:limits-geringer-sameth} for the \GS set of \J-factors~\cite{Geringer-Sameth:2014yza} and in \Cref{fig:limits-bonnivard} for the \B set of \J-factors~\cite{Bonnivard:2014kza, Bonnivard:2015xpq}.
The combined limits are presented with their 68\% and 95\% containment bands, and are expected to be close to the median limit when no signal is present.
We observe agreement with the null hypothesis for all channels, within $2\sigma$ standard deviations, between the observed limits and the expectations given by the median limits.
Limits obtained from each detector are also indicated in the figures, where limits for all dSphs observed by the specific instrument have been combined.

\begin{figure}[h]
    \centering{
    \begin{tabular}{cc}
        \includegraphics[width=0.35\textwidth]{figures/glory_duck/limits/Glory_Duck_Annihilation_WW_Geringer-Sameth_Combination_bands.pdf} &
        \includegraphics[width=0.35\textwidth]{figures/glory_duck/limits/Glory_Duck_Annihilation_ZZ_Geringer-Sameth_Combination_bands.pdf} \\
        \includegraphics[width=0.35\textwidth]{figures/glory_duck/limits/Glory_Duck_Annihilation_bb_Geringer-Sameth_Combination_bands.pdf} &
        \includegraphics[width=0.35\textwidth]{figures/glory_duck/limits/Glory_Duck_Annihilation_tt_Geringer-Sameth_Combination_bands.pdf} \\
        \includegraphics[width=0.35\textwidth]{figures/glory_duck/limits/Glory_Duck_Annihilation_ee_Geringer-Sameth_Combination_bands.pdf} &
        \includegraphics[width=0.35\textwidth]{figures/glory_duck/limits/Glory_Duck_Annihilation_mumu_Geringer-Sameth_Combination_bands.pdf} \\
        \includegraphics[width=0.35\textwidth]{figures/glory_duck/limits/Glory_Duck_Annihilation_tautau_Geringer-Sameth_Combination_bands.pdf} &
        \end{tabular}
        }
        \caption{Upper limits at 95\% confidence level on \sv in function of the DM mass for eight annihilation channels, using the set of \J factors from Ref.~\cite{Geringer-Sameth:2014yza} (\GS set in \Cref{tab:gd_J_factor}). The black solid line represents the observed combined limit, the black dashed line is the median of the null hypothesis corresponding to the expected limit, while the green and yellow bands show the 68\% and 95\% containment bands. Combined upper limits for each individual detector are also indicated as solid, colored lines.
        The value of the thermal relic cross-section in function of the DM mass is given as the red dotted-dashed line~\cite{Bertone_2005}. Dropping below the thermal relic effectively rules out the DM mass as a candidate for primordial DM.}
    \label{fig:limits-geringer-sameth}
    \end{figure}

    \begin{figure}[h]
    \centering{
    \begin{tabular}{cc}
        \includegraphics[width=0.35\textwidth]{figures/glory_duck/limits/Glory_Duck_Annihilation_WW_Bonnivard_Combination_bands.pdf} &
        \includegraphics[width=0.35\textwidth]{figures/glory_duck/limits/Glory_Duck_Annihilation_ZZ_Bonnivard_Combination_bands.pdf} \\
        \includegraphics[width=0.35\textwidth]{figures/glory_duck/limits/Glory_Duck_Annihilation_bb_Bonnivard_Combination_bands.pdf} &
        \includegraphics[width=0.35\textwidth]{figures/glory_duck/limits/Glory_Duck_Annihilation_tt_Bonnivard_Combination_bands.pdf} \\
        \includegraphics[width=0.35\textwidth]{figures/glory_duck/limits/Glory_Duck_Annihilation_ee_Bonnivard_Combination_bands.pdf} &
        \includegraphics[width=0.35\textwidth]{figures/glory_duck/limits/Glory_Duck_Annihilation_mumu_Bonnivard_Combination_bands.pdf} \\
        \includegraphics[width=0.35\textwidth]{figures/glory_duck/limits/Glory_Duck_Annihilation_tautau_Bonnivard_Combination_bands.pdf} &
        \end{tabular}
        }
        \caption{Same as \cref{fig:limits-geringer-sameth}, using the set of \J factors from Ref.~\cite{Bonnivard:2014kza, Bonnivard:2015xpq} (\B set in \Cref{tab:gd_J_factor}).}
    \label{fig:limits-bonnivard}
    \end{figure}

\begin{figure}[h]
\centering{
\begin{tabular}{ccc}
    \includegraphics[width=0.3\textwidth]{figures/glory_duck/comparison/GD_limits_WW.pdf} &
    \includegraphics[width=0.3\textwidth]{figures/glory_duck/comparison/GD_limits_ZZ.pdf} &
    \includegraphics[width=0.3\textwidth]{figures/glory_duck/comparison/GD_limits_bb.pdf} \\
    \includegraphics[width=0.3\textwidth]{figures/glory_duck/comparison/GD_limits_tt.pdf} &
    \includegraphics[width=0.3\textwidth]{figures/glory_duck/comparison/GD_limits_ee.pdf} &
    \includegraphics[width=0.3\textwidth]{figures/glory_duck/comparison/GD_limits_mumu.pdf} \\
    \includegraphics[width=0.3\textwidth]{figures/glory_duck/comparison/GD_limits_tautau.pdf} &
    \end{tabular}
    }
    \caption{Comparisons of the combined limits at 95\% confidence level for each of the eight annihilation channels when using the \J factors from Ref.~\cite{Geringer-Sameth:2014yza} (\GS set in \Cref{tab:gd_J_factor}), plain lines, and the \J factor from Ref.~\cite{Bonnivard:2014kza, Bonnivard:2015xpq} (\B set in \Cref{tab:gd_J_factor}), dashed lines. The cross-section given by the thermal relic is also indicated~\cite{Bertone_2005}.}
\label{fig:limits-comparison}
\end{figure}

Below \textasciitilde300 GeV, the $Fermi$-LAT dominates the DM limits for all annihilation channels.
From \textasciitilde300 GeV to \textasciitilde2 TeV, $Fermi$-LAT continues to dominate for the hadronic and bosonic DM channels, yet the IACTs (H.E.S.S., MAGIC, and VERITAS) and $Fermi$-LAT all contribute to the limit for leptonic DM channels.
For DM masses between \textasciitilde2 TeV to \textasciitilde10 TeV, the IACTs dominate leptonic DM annihilation channels, whereas both the $Fermi$-LAT and the IACTs dominate bosonic and hadronic DM annihilation channels.
From \textasciitilde10 TeV to \textasciitilde100 TeV, both the IACTs and HAWC contribute significantly to the leptonic DM limit.
For hadronic and bosonic DM, the IACTs and $Fermi$-LAT both contribute strongly.

\clearpage

We notice that the limits computed using the \B set of \J-factor are always better compared to the ones calculated with the \GS set.
For the $W^+W^-$, $Z^+Z^-$, $b\bar{b}$, and $t\bar{t}$ channels, the ratio between the limits computed with the two sets of \J-factor is varying between a factor of \textasciitilde3 and \textasciitilde5 depending on the energy, with the largest ratio around 10~TeV.
For the channels $e^+e^-$, $\mu^+\mu^-$, and $\tau^+\tau^-$, the ratio lies between \textasciitilde2 to \textasciitilde6, being maximum around 1~TeV.
Examining \Cref{fig:comparison_J_1} and \Cref{fig:comparison_J_2} in \Cref{sec:gd_jfactor_systematic}, these differences are explained by the fact that the \B set provides higher \J-factors for the majority of the studied dSphs, with the notable exception of Segue~I.
The variation on the ratio of the limits for the two sets is due to different dSph dominating the limits depending on the energy.
One set, \B, pushes the range of which thermal cross-section which can be excluded to higher mass.
This comparison demonstrates the magnitude of systematic uncertainties associated with the choice of the \J-factor calculation
The \GS and \B sets present a difference in the limits for all channels of about
This difference is explained, see  \Cref{fig:comparison_J_1} and \Cref{fig:comparison_J_2}, by the fact that the \B set provides higher \J-factors for all dSph except for Segue~I.

%%%%%%%%%%%%%%%%%%%%%%%%%%%%%%%%%%%%%%%%%%%%%%%%%%%%%%%%%%%%%%%%%%%%%%%%%%%%%%%%%%%%%
\section{HAWC Systematics} \label{sec:hawc_systematic}
%%%%%%%%%%%%%%%%%%%%%%%%%%%%%%%%%%%%%%%%%%%%%%%%%%%%%%%%%%%%%%%%%%%%%%%%%%%%%%%%%%%%%

%%%%%%%%%%%%%%%%%%%%%%%%%%%%%%%%%%%%%%%%%%%%%%%%%%%%%%%%%%%%%%%%%%%%
\subsection{Inverse Compton Scattering} \label{sec:gd_ics}
%%%%%%%%%%%%%%%%%%%%%%%%%%%%%%%%%%%%%%%%%%%%%%%%%%%%%%%%%%%%%%%%%%%%
The DM-DM annihilation channels produce many high energy electrons regardless of the primary annihilation channel.
These high energy electrons can produce high energy gamma-rays through Inverse Compton Scattering (ICS).
If this effect is strong, it would change the morphology of the source and increase the total expected gamma-ray counts from any source.
The PPPC \cite{Cirelli_2011} provides tools in Mathematica for calculating the impact of ICS for an arbitrary location in the sky for a specified annihilation channel.
We calculated the change in gamma-ray counts for DM annihilation to primary $e\bar{e}$ for RA and Dec corresponding to Segue1 and Coma Berenices.
These dSphs were chosen because they are the strongest contributors to the limit.
$e\bar{e}$ was selected because it would have the largest number of high energy electrons.
The effect was found to be on the order of $~10^{-7}$ on the gamma-ray spectrum.
As a result, this systematic is not considered in our analysis.

%%%%%%%%%%%%%%%%%%%%%%%%%%%%%%%%%%%%%%%%%%%%%%%%%%%%%%%%%%%%%%%%%%%%
\subsection{Point Source Versus Extended Source Limits}\label{sec:gd_ext_limitvs_ptsrc}
%%%%%%%%%%%%%%%%%%%%%%%%%%%%%%%%%%%%%%%%%%%%%%%%%%%%%%%%%%%%%%%%%%%%

The previous DM search toward dSph approximated the dSphs as point sources \cite{Albert_2018}.
In this analysis, the dSphs are implemented as extended with J-factor distributions following those produced by \cite{Geringer-Sameth:2014yza}.
The resolution of the cited map is much finer than HAWC's angular resolution.
The vast majority of the J-factor distribution is represented on the central HAWC pixel of the dSph spatial map.
However, the neighboring 8 pixels are not negligible and contribute to our limit.

\begin{figure}[ht]
\centering{
    \includegraphics[scale=0.19]{figures/glory_duck/hawc/95Lim_Segue1_ExtVsPt.pdf}}
    \caption{Comparisons of the combined limits at 95\% confidence level for a point source analysis and extended source using ~\cite{Geringer-Sameth:2014yza} \GS J-factor distributions and PPPC \cite{Cirelli_2011} annihilation spectra. Shown are the limits for Segue1 which will have the most significant impact on the combined limit. 6 of the 7 DM annihilation channels are shown. Solid lines are extended source studies. Dashed lines are point source studies. Overall, the extended source analysis improves the limit by a factor of 2.}
\label{fig:Seg1point_versus_extended}
\end{figure}

\begin{figure}[h]
\centering{
    \includegraphics[scale=0.19]{figures/glory_duck/hawc/95Lim_ComaBerenices_ExtVsPt.pdf}}
    \caption{Same as \cref{fig:Seg1point_versus_extended} on Coma Berenices. This dSph also contributes significantly to the limit. The limits are identical in this case.}
\label{fig:ComaBpoint_versus_extended}
\end{figure}

\Cref{fig:Seg1point_versus_extended} shows a substantial improvement to the limit for Segue1.
\cref{fig:ComaBpoint_versus_extended} however showed identical limits.
These disparities are best explained by the relative difference in their J-Factors.
Both dSphs pass almost overhead of the HAWC detector, however Segue1 has the larger J-Factor between the two.
Adjacent pixels to the central pixel will therefor contribute to the limits.
This is the case for other dSph that are closer to the zenith of the HAWC detector.

Comparison plots for all sources and the combined limit can be found in the sandbox for the Glory Duck project.

%%%%%%%%%%%%%%%%%%%%%%%%%%%%%%%%%%%%%%%%%%%%%%%%%%%%%%%%%%%%%%%%%%%%
\subsection{Impact of Pointing Systematic}\label{sec:gd_pointing_sys}
%%%%%%%%%%%%%%%%%%%%%%%%%%%%%%%%%%%%%%%%%%%%%%%%%%%%%%%%%%%%%%%%%%%%

\begin{figure}
    \centering{
        \includegraphics[scale=0.45]{figures/glory_duck/dec_effect.png}
    }
    \caption{HAWC reconstructed declination minus HESS reconstructed declination plotted over HAWC longitude. There is a clear correlation with longitude and disparity between catalogs. This disparity is maximized the larger the zenith angle. Plot pulled from internal HAWC \href{https://private.hawc-observatory.org/wiki/images/3/30/HAWCMeetingOct2020-AJS-Pointing.pdf}{presentation}.}
    \label{fig:pointing_problem}
\end{figure}

During the analysis it was discovered that directional reconstruction of gamma-rays had a systematic bias at large zenith angles.
\Cref{fig:pointing_problem} demonstrates the disparity between HAWC's reconstructed declination versus the same sources reconstructed by HESS.
Shown on the presentation is dependence on the pointing systematic on declination.
New spatial profiles were generated for every dSph and limits were computed for the adjusted declination.

\Cref{fig:pointing_systematic} demonstrates the impact of this systematic for all DM annihilation channels studied by HAWC. The impact is a tiny improvement, yet mostly identical, to the combined limits.

\begin{figure}[h]
    \centering{
        \includegraphics[scale=0.21]{figures/glory_duck/hawc/PointingSystematic_GD_Combined_bb.pdf}
        \includegraphics[scale=0.21]{figures/glory_duck/hawc/PointingSystematic_GD_Combined_ee.pdf}
        \includegraphics[scale=0.21]{figures/glory_duck/hawc/PointingSystematic_GD_Combined_mumu.pdf}
        \includegraphics[scale=0.21]{figures/glory_duck/hawc/PointingSystematic_GD_Combined_tautau.pdf}
        \includegraphics[scale=0.21]{figures/glory_duck/hawc/PointingSystematic_GD_Combined_tt.pdf}
        \includegraphics[scale=0.21]{figures/glory_duck/hawc/PointingSystematic_GD_Combined_ww.pdf}
        \includegraphics[scale=0.21]{figures/glory_duck/hawc/PointingSystematic_GD_Combined_zz.pdf}
    }
    \caption{Comparison of combined limits when correcting for HAWC's pointing systematic. All DM annihilation channels are shown. The solid black line is the ratio between published limit to the declination corrected limit. The blue solid line or "Combined\_og" represented the limits computed for Glory Duck. The solid orange line or "Combined\_ad" represented the limits computed after correcting for the pointing systematic.}
    \label{fig:pointing_systematic}
\end{figure}

%%%%%%%%%%%%%%%%%%%%%%%%%%%%%%%%%%%%%%%%%%%%%%%%%%%%%%%%%
\section{\J-factor distributions}\label{sec:gd_jfactor_systematic}
%%%%%%%%%%%%%%%%%%%%%%%%%%%%%%%%%%%%%%%%%%%%%%%%%%%%%%%%%

%%%%%%%%%%%%%%%%%%%%%%%%%%%%%%%%%%%%%%%%%%%%%%%%%%%%%%%%%
\subsection{Numerical integration of \GS maps}\label{sec:gd_jfacintegration}
%%%%%%%%%%%%%%%%%%%%%%%%%%%%%%%%%%%%%%%%%%%%%%%%%%%%%%%%%

It was discovered well after the HAWC analysis was completed that the published tables from \GS \cite{Geringer_Sameth_2015} quoted median \J-factors were computed in a non-trivial manner.
The assumption had been that the published tables represented the \J-factor as a function of $\theta$ for the best global fit model on a per-source basis.
However, this is not the case.
Instead, what is published are the best fit model for each dwarf that only considers stars up to the angular separation $\theta$.
Therefore, the model is changing for each value of $\theta$ for each dwarf.
Yet, the introduced features from unique models at each $\theta$ are much smaller than the angular resolution of HAWC.
It is not expected for these effects to impact the limits and TS greatly as a result.

Median \J-factor model profiles were provided by the authors.
New maps were generated and analyzed for Segue1 and Coma Berenices.
\Cref{fig:gd_djmap_diff} shows the differential between maps generated with the method from \cref{sec:gd_jfacintegration} and from the authors of \cite{Geringer_Sameth_2015}.
These maps were reanalyzed for all SM DM annihilation channels.
Upper limits for these channels are shown in \cref{fig:gd_jfacmodel_systematic}

From \cref{fig:gd_jfacmodel_systematic}, we can see that the impact of these model differences was not substantial.
The observed impact was a fractional effect which is much smaller than the impact from selecting another DM spatial distribution model as was shown in \cref{fig:limits-comparison}.

\begin{figure}
    \centering{
    \includegraphics[scale=0.5]{figures/glory_duck/hawc/Segue1_diff.png}
    \includegraphics[scale=0.5]{figures/glory_duck/hawc/ComaB_diff.png}
    }
    \caption{Differential map of $d\J/\Omega$ from model built in \Cref{sec:gd_jfacintegration} and profiles provided directly from authors. (Top) Differential from Segue1. (bottom) Differential from Coma Berenices. Note that their scales are not the same. Segue1 shows the deepest discrepancies which is congruent with its large uncertainties. Both models show anuli where unique models become apparent.}\label{fig:gd_djmap_diff}
\end{figure}

\begin{figure}
    \centering{
        \includegraphics[scale=0.38]{figures/glory_duck/hawc/JFacModel_Systematic_GD_ComaBerenices_bb.pdf}
        \includegraphics[scale=0.38]{figures/glory_duck/hawc/JFacModel_Systematic_GD_Segue1_bb.pdf}
    }
    \caption{HAWC limits for Coma Berenices (top) and Segue1 (bottom) for two different map sets. Blue lines are limits calculated on maps with poor model representation. Orange lines are limits calculated on spatial profiles provided by the authors of \cite{Geringer_Sameth_2015}. Black line is the ratio of the poor spatial model limits to the corrected spatial models. The left y-axis measures \sv for the blue and orange lines. The right y-axis measures the ratio and is unitless.}\label{fig:gd_jfacmodel_systematic}
\end{figure}

%%%%%%%%%%%%%%%%%%%%%%%%%%%%%%%%%%%%%%%%%%%%%%%%%%%%%%%%%
\subsection{\GS Versus \B spatial models}\label{sec:gd_gsVb}
%%%%%%%%%%%%%%%%%%%%%%%%%%%%%%%%%%%%%%%%%%%%%%%%%%%%%%%%%

In this is a comparison between the \J-factors computed by Geringer-Sameth \emph{et al.}~\cite{Geringer-Sameth:2014yza} (the \GS set) and the ones computed by Bonnivard \emph{et al.}~\cite{Bonnivard:2014kza, Bonnivard:2015xpq} (the \B set).
%The differences in computation between both sets are detailed in \cref{sec:DM}.
%
The \GS \J-factors are computed through a Jeans analysis of the kinematic stellar data of the  selected dSphs, assuming a dynamic equilibrium and a spherical symmetry for the dSphs.
They adopted the generalized DM density distribution, known as Zhao-Hernquist, introduced by~\cite{Zhao:1995cp}, carrying three additional index parameters to describe the inner and outer slopes, and the break of the density profile.
Such a profile parametrization allows the reduction of the theoretical bias from the choice of a specific radial dependency on the kinematic data.
In other words, the increase of free parameters with the use of the  Zhao-Hernquist profile allows a better description of the mass density distribution of dark matter.

In addition, a constant velocity anisotropy profile and a Plummer light profile \cite{10.1093/mnras/71.5.460} for the stellar distribution were assumed.
The velocity anisotropy profile depends on the radial and tangential velocity dispersion.
However, its determination remains challenging since only the line-of-sight velocity dispersion can be derived from velocity measurements.
Therefore, the parametrization of the anisotropy profile is obtained from simulated halos (see~\cite{Hunter:2013vua} for more details).
They provide the values of the \J-factors of regions extending to various angular radius up to the outermost member star.

The \B \J-factors were computed through a Jeans analysis taking into account the systematic uncertainties induced by the DM profile parametrization, the radial velocity anisotropy profile, and the triaxiality of the halo of the dwarf galaxies.
They performed a more complete study of the dSph kinematics and dynamics than \GS for the determination of the \J-factor.
Conservative values of the \J-factors where obtained using an Einasto DM density profile~\cite{Dhar_2010}, a realistic anisotropy profile known as the Baes \& Van Hese profile~\cite{Baes:2007tx} that takes into account the inner regions can be significantly non-isotropic, and a Zhao-Hernquist light profile~\cite{Zhao:1995cp}.

For both sets, \J-factor values are provided for all dSphs as a function of the radius of the integration region~\cite{Geringer-Sameth:2014yza,Bonnivard:2014kza,Bonnivard:2015xpq}.
\Cref{tab:gd_J_factor} shows the heliocentric distance and Galactic coordinates of the twenty dSphs, together with the two sets of \J-factor values integrated up to the outermost observed star for \GS and the tidal radius for \B.
Both \J-factor sets were derived through a Jeans analysis based on the same kinematic data, except for Draco where the measurements of~\cite{2015MNRAS.448.2717W} have been adopted in the computation of the \B value.
The computations for producing the \GS and \B samples differ in the choice of the DM density, velocity anisotropy, and light profiles, for which the set \B takes into account some sources of systematic uncertainties.

\Cref{fig:comparison_J_1} and \Cref{fig:comparison_J_2} show the comparisons for the \J-factor versus the angular radius for each of the 20 dSphs used in this study.
The uncertainties provided by the authors are also indicated in the figures.
For the \GS set, the computation stops at the angular radius corresponding to the outermost observed star, while for the \B set, the computation stops at the angular radius corresponding to the tidal radius.

\begin{figure}[ht]
\centering{
    \includegraphics[scale=0.32]{figures/glory_duck/appendix/BootesI.pdf}
    \includegraphics[scale=0.32]{figures/glory_duck/appendix/CanesVenaticiI.pdf}
    \includegraphics[scale=0.32]{figures/glory_duck/appendix/CanesVenaticiII.pdf}
    \includegraphics[scale=0.32]{figures/glory_duck/appendix/Carina.pdf}
    \includegraphics[scale=0.32]{figures/glory_duck/appendix/ComaBerenices.pdf}
    \includegraphics[scale=0.32]{figures/glory_duck/appendix/Draco.pdf}
    \includegraphics[scale=0.32]{figures/glory_duck/appendix/Fornax.pdf}
    \includegraphics[scale=0.32]{figures/glory_duck/appendix/Hercules.pdf}
    \includegraphics[scale=0.32]{figures/glory_duck/appendix/LeoI.pdf}
    \includegraphics[scale=0.32]{figures/glory_duck/appendix/LeoII.pdf}
    }
    \caption{Comparisons between the \J-factors versus the angular radius for the computation of $J$ factors from Ref.~\cite{Geringer-Sameth:2014yza} (\GS set in \Cref{tab:gd_J_factor}) in blue and for the computation from Ref.~\cite{Bonnivard:2014kza, Bonnivard:2015xpq} (\B set in \cref{tab:gd_J_factor}) in orange. The solid lines represent the central value of the \J-factors while the shaded regions correspond to the 1$\sigma$ standard deviation.}
\label{fig:comparison_J_1}
\end{figure}

\begin{figure}[ht]
\centering{
    \includegraphics[scale=0.32]{figures/glory_duck/appendix/LeoIV.pdf}
    \includegraphics[scale=0.32]{figures/glory_duck/appendix/LeoT.pdf}
    \includegraphics[scale=0.32]{figures/glory_duck/appendix/LeoV.pdf}
    \includegraphics[scale=0.32]{figures/glory_duck/appendix/Sculptor.pdf}
    \includegraphics[scale=0.32]{figures/glory_duck/appendix/SegueI.pdf}
    \includegraphics[scale=0.32]{figures/glory_duck/appendix/SegueII.pdf}
    \includegraphics[scale=0.32]{figures/glory_duck/appendix/Sextans.pdf}
    \includegraphics[scale=0.32]{figures/glory_duck/appendix/UrsaMajorI.pdf}
    \includegraphics[scale=0.32]{figures/glory_duck/appendix/UrsaMajorII.pdf}
    \includegraphics[scale=0.32]{figures/glory_duck/appendix/UrsaMinor.pdf}
}
    \caption{Comparisons between the \J-factors versus the angular radius for the computation of $J$ factors from Ref.~\cite{Geringer-Sameth:2014yza} (\GS set in \cref{tab:gd_J_factor}) in blue and for the computation from Ref.~\cite{Bonnivard:2014kza, Bonnivard:2015xpq} (\B set in \cref{tab:gd_J_factor}) in orange. The solid lines represent the central value of the \J-factors while the shaded regions correspond to the 1$\sigma$ standard deviation.}
\label{fig:comparison_J_2}
\end{figure}

%%%%%%%%%%%%%%%%%%%%%%%%%%%%%%%%%%%%%%%%%%%%%%%%%%%%%%%%%
\section{Discussion and Conclusions\label{sec:gd_conclusions}}
%%%%%%%%%%%%%%%%%%%%%%%%%%%%%%%%%%%%%%%%%%%%%%%%%%%%%%%%%

In this multi-instrument analysis, we have used observations of 20 dSphs from the gamma-ray telescopes Fermi-LAT, H.E.S.S., MAGIC, VERITAS, and HAWC to perform a collective DM search annihilation signals.
The data were combined across sources and detectors to significantly increase the sensitivity of the search.
We have observed no significant deviation from the null DM hypothesis, and so present our results in terms of upper limits on the annihilation cross-section for seven potential DM annihilation channels.

Fermi-LAT brings the most stringent constraints for continuum channels below approximately 1~TeV.
The remaining detectors dominate at higher energies.
Overall, for multi-TeV DM mass, the combined DM constraints from all five telescopes are 2-3 times stronger than any individual telescope for multi-TeV DM.

Derived from observations of many dSphs, our results produce robust limits given the DM content of the dSphs is relatively well constrained.
The obtained limits span the largest mass range of any WIMP DM search.
Our combined analysis improves the sensitivity over previously published results from each detector which produces the most stringent limits on DM annihilation from dSphs.
These results are based on deep exposures of the most promising known dSphs with the currently most sensitive gamma-ray instruments.
Therefore, our results constitute a legacy of a generation of gamma-ray instruments on WIMP DM searches towards dSphs.
Our results will remain the reference in the field until a new generation of more sensitive gamma-ray instruments begin operations, or until new dSphs with higher \J-factors are discovered.

This analysis serves as a proof of concept for future multi-instrument and multi-messenger combination analyses.
With this collaborative effort, we have managed to sample over four orders in magnitude in gamma-ray energies with distinct observational techniques.
Determining the nature of DM continues to be an elusive and difficult problem.
Larger datasets with diverse measurement techniques could be essential to tackling the DM problem.
A future collaboration using similar techniques as the ones described here could grow even beyond gamma rays.
The models we used for this study include annihilation channels with neutrinos in the final state.
Advanced studies could aim to merge our results with those from neutrino observatories with large data sets.
Efforts with IceCube have already started, and the groundwork is presented in \cref{sec:ic3_dm}.

From this work, a selection of the best candidates for observations, according to the latest knowledge on stellar dynamics and modelling techniques for the derivation of the \J-factors on the potential dSphs targets, is highly desirable at the time that new experiments are starting their dark matter programs using dSphs.
Given the systematic uncertainty inherent to the derivation of the \J-factors,
an informed observational strategy would be to select both objects with the highest \J-factors that could lead to DM signal detection, and objects with robust \J-factor predictions, i.e. with kinematic measurements on many bright stars, which would strengthen the DM interpretation reliability of the observation outcome.

This analysis combines data from multiple telescopes to produce strong constraints on astrophysical objects.
From this perspective, these methods can be applied beyond just DM searches.
Almost every astrophysical study can benefit from multi-telescope, multi-wavelength gamma-ray studies and is continued in this thesis.
Moreover, HAWC and IceCube have made substantial leaps in their analysis methods and merits a refreshed analysis on DM signals from dSph.
In HAWC, the updated 2D energy binning scheme, see \cref{sec:hawc_nn} for description, split by $f_\mathrm{hit}$ and estimated energy from our Neural Network (NN) method in an effort to better estimate gamma-ray energy.
Whereas, IceCube's previous DM search towards dSph was performed when it was an incomplete detector.
Leaps have also been made on the theoretical front for both the particle physics modeling and DM density profiles.
These broad improvements have been implemented and started on both HAWC and IceCube.
The improvements for HAWC are featured in \Cref{sec:multithread} and IceCube's are in \Cref{sec:ic3_dm}.

The results of the Glory Duck collaboration have gone through collaboration review for all experiments involved.
The final step of verifying and consolidating the author list is what remains before sending the work to publication.

%%%%%%%%%%%%%%%%%%%%%%%%%%%%%%%%%%%%%%%%%%%%%%%%%%%%%%%%%%%%%%%%%%%%%%%%%%%%%%%%%%%%%
\chapter{Multithreading HAWC analyses for Dark Matter Searches} \label{sec:multithread}
%%%%%%%%%%%%%%%%%%%%%%%%%%%%%%%%%%%%%%%%%%%%%%%%%%%%%%%%%%%%%%%%%%%%%%%%%%%%%%%%%%%%%
%%%%%%%%%%%%%%%%%%%%%%%%%%%%%%%%%%%%%%%%%%%%%%%%%%%%%%%%%%%%%%%%%%%%%%%%%%%%%%%%%%%%%
\section{Introduction}\label{sec:mtd_intro}
%%%%%%%%%%%%%%%%%%%%%%%%%%%%%%%%%%%%%%%%%%%%%%%%%%%%%%%%%%%%%%%%%%%%%%%%%%%%%%%%%%%%%

%%%%%%%%%%%%%%%%%%%%%%%%%%%%%%%%%%%%%%%%%%%%%%%%%%%%%%%%%%%%%%%%%%%%%%%%%%%%%%%%%%%%%
\section{Dataset and Background}\label{sec:mtd_databgd}
%%%%%%%%%%%%%%%%%%%%%%%%%%%%%%%%%%%%%%%%%%%%%%%%%%%%%%%%%%%%%%%%%%%%%%%%%%%%%%%%%%%%%

This section enumerates the data and background methods used for HAWC's multi-threaded study of dSphs.
\Cref{sec:mtd_data} and \Cref{sec:mtd_tools} are most useful for fellow HAWC collaborators looking to replicate a multithreaded dSph DM search.

%%%%%%%%%%%%%%%%%%%%%%%%%%%%%%%%%%%%%%%%%%%%%%%
\subsection{Itemized HAWC files}\label{sec:mtd_data}
%%%%%%%%%%%%%%%%%%%%%%%%%%%%%%%%%%%%%%%%%%%%%%%

\begin{itemize}
    \item Detector Resolution: \texttt{refit-Pass5-Final-NN-detRes-zenith-dependent.root}
    \item Data Map: \texttt{Pass5-Final-NN-maptree-ch103-ch1349-zenith-dependent.root}
    \item Spectral Dictionary: \texttt{HDMSpectra\_dict\_gamma.npy}
\end{itemize}

% %%%%%%%%%%%%%%%%%%%%%%%%%%%%%%%%%%%%%%%%%%%%%%%
\subsection{Software Tools and Development}\label{sec:mtd_tools}
%%%%%%%%%%%%%%%%%%%%%%%%%%%%%%%%%%%%%%%%%%%%%%%

This analysis was performed using HAL and 3ML \cite{Abeysekara_2017, vianello2015multimission} in Python version 3.
I built software in collaboration with Michael Martin and Letrell Harris to implement the \emph{Dark Matter Spectra from the Electroweak to the Planck Scale} (HDM) \cite{HDMSpectra} and dSphs spatial model from \cite{DM_Strigari20} for HAWC analysis.
A NumPy version of this dictionary was made for Py3.
The analysis was performed using the $f_{\textrm{hit}}$ framework performed in the HAWC Crab paper \cite{Abeysekara_2017}.
The Python2 NumPy dictionary file for gamma-ray final states is \texttt{dmCirSpecDict.npy}.
The corresponding Python3 file is \texttt{DM\_CirrelliSpectrum\_dict\_gammas.npy}.
These files can also be used for decay channels and the PPPC describes how \cite{Cirelli_2011}.
All other software used for data analysis, DM profile generation, and job submission to SLURM are also kept in my sandbox for \href{https://gitlab.com/hawc-observatory/sandboxes/salaza82/glory-duck-hawc}{the Glory Duck} project.

%%%%%%%%%%%%%%%%%%%%%%%%%%%%%%%%%%%%%%%%%%%%%%%
\subsection{Data Set and Background Description} \label{sec:mtd_data_bkgd}
%%%%%%%%%%%%%%%%%%%%%%%%%%%%%%%%%%%%%%%%%%%%%%%

The HAWC data maps used for this analysis contain 1017 days of data between runs 2104 (2014-11-26) and 7476 (2017-12-20).
They were generated from pass 4.0 reconstruction.
The analysis is performed using the $f_{hit}$ energy binning scheme with bins (1-9) similar to what was done for the Crab and previous HAWC dSph analysis \cite{Abeysekara_2017,Albert_2018}.
Bin 0 was excluded as it has substantial hadronic contamination and poor angular resolution.

This analysis was done on dSphs because of their large DM mass content relative to baryonic mass.
We consider the following to estimate the background to this study.

\begin{itemize}
    \item The dSphs are small in HAWC's field of view, so the analysis is not sensitive to large or small scale anisotropies.
    \item The dSphs used in this analysis are off the galactic plane.
    \item The dSphs are baryonically faint relative to their expected dark matter content and are not expected to contain high energy gamma-ray sources.
\end{itemize}

Therefor we make no additional assumptions on the background from our sources and use HAWC's standard direct integration method for background estimation \cite{Abeysekara_2017}.
It is possible for gamma rays from DM annihilation to scatter in transit to HAWC via Inverse Compton Scattering (ICS).
This was investigated and its impact on the flux is basically zero.
Supporting information on this is in \Cref{sec:gd_ics}


%%%%%%%%%%%%%%%%%%%%%%%%%%%%%%%%%%%%%%%%%%%%%%%%%%%%%%%%%%%%%%%%%%%%%%%%%%%%%%%%%%%%%
\chapter{IceCube Neutrino Observatory}\label{sec:ice3}
%%%%%%%%%%%%%%%%%%%%%%%%%%%%%%%%%%%%%%%%%%%%%%%%%%%%%%%%%%%%%%%%%%%%%%%%%%%%%%%%%%%%%
%-----------------------------------------------------------------------------------%
\section{The Detector}
%-----------------------------------------------------------------------------------%

%-----------------------------------------------------------------------------------%
\section{Events Reconstruction and Data Acquisition}
%-----------------------------------------------------------------------------------%

%$$$$$$$$$$$$$$$$$$$$$$$$$$$$$$$$$$$$$$$$$$$$$$$$$$$$$$$$$$$$$$$$$$$$$$$$$$$$$$$$$$$%
\subsection{Angle}
%$$$$$$$$$$$$$$$$$$$$$$$$$$$$$$$$$$$$$$$$$$$$$$$$$$$$$$$$$$$$$$$$$$$$$$$$$$$$$$$$$$$%

%$$$$$$$$$$$$$$$$$$$$$$$$$$$$$$$$$$$$$$$$$$$$$$$$$$$$$$$$$$$$$$$$$$$$$$$$$$$$$$$$$$$%
\subsection{Energy}
%$$$$$$$$$$$$$$$$$$$$$$$$$$$$$$$$$$$$$$$$$$$$$$$$$$$$$$$$$$$$$$$$$$$$$$$$$$$$$$$$$$$%

%-----------------------------------------------------------------------------------%
\section{Northern Test Site}
%-----------------------------------------------------------------------------------%

%$$$$$$$$$$$$$$$$$$$$$$$$$$$$$$$$$$$$$$$$$$$$$$$$$$$$$$$$$$$$$$$$$$$$$$$$$$$$$$$$$$$%
\subsection{PIgeon remote dark rate testing}
%$$$$$$$$$$$$$$$$$$$$$$$$$$$$$$$$$$$$$$$$$$$$$$$$$$$$$$$$$$$$$$$$$$$$$$$$$$$$$$$$$$$%

%$$$$$$$$$$$$$$$$$$$$$$$$$$$$$$$$$$$$$$$$$$$$$$$$$$$$$$$$$$$$$$$$$$$$$$$$$$$$$$$$$$$%
\subsection{Bulkhead Construction}
%$$$$$$$$$$$$$$$$$$$$$$$$$$$$$$$$$$$$$$$$$$$$$$$$$$$$$$$$$$$$$$$$$$$$$$$$$$$$$$$$$$$%


%%%%%%%%%%%%%%%%%%%%%%%%%%%%%%%%%%%%%%%%%%%%%%%%%%%%%%%%%%%%%%%%%%%%%%%%%%%%%%%%%%%%%
\chapter{Heavy Dark Matter Annihilation Search with IceCube's North Sky Track Data} \label{sec:ic3_dm}
%%%%%%%%%%%%%%%%%%%%%%%%%%%%%%%%%%%%%%%%%%%%%%%%%%%%%%%%%%%%%%%%%%%%%%%%%%%%%%%%%%%%%
%%%%%%%%%%%%%%%%%%%%%%%%%%%%%%%%%%%%%%%%%%%%%%%%%%%%%%%%%%%%%%%%%%%%%%%%%%%%%%%%%%%%%
\section{Introduction} \label{sec:icDM_intro}
%%%%%%%%%%%%%%%%%%%%%%%%%%%%%%%%%%%%%%%%%%%%%%%%%%%%%%%%%%%%%%%%%%%%%%%%%%%%%%%%%%%%%

Neutrinos are another astrophyical messenger than can travel long distances without interaction.
Uniquely, they interact less readily than photons especially above PeV energies.
Neutrinos thereofre provide another window through which we can perform dark matter searches.
Neutrinos come in three flabors and so this triples the multiplicity of the particles we are searching for.

Icecube has not done a DM annihilation analysis towards dwarf galaxies for a while. \todo{cite 2013 paper}.
This is in spite of the potentially crucial sensitivity afforded from neutrino spectral lines \todo{cite dan hooper and neutrino lines.}
A lot has changed in IC3 since that last analysis (we have more strings, we have much more sophisticated analysis methods, and the theory modeling has made significant leaps.)
Therefore it is time to finally do a DM search toward dSphs.
The hope is that by laying down the important statistical foundation as well, that this work can be meshed with gamma-ray data.
IceCube is sensitive to annihilating DM to the DM ranges above 1 TeV and can produce competitive results relative to gamma ray observatories in spectral models that produce sharp neutrino features.
The goal of this analysis is to perform a DM annihilation search using the new datasets NST.
The search will only be towards dwarf spheroidal galaxies (dSph).
These sources are known for their low backgrounds and high DM contents.
Since the dataset is sensitive to the north and south, as many dSph as possible will be included.
Additionally, with annihilation, these sources can be treated as point sources with little loss to sensitivity or model dependence on how the DM is distributed.
DM masses from 500 GeV to 100 PeV are considered for this analysis.
All standard model annihilation channels available from the HDMSpectra are studied in this analysis.

Additional work is done to extract the Likelihood profiles for each DM, source hypothesis so that these data can be combined with gamma-ray observatories.
This work is considered a separate project as the statistical treatment is unique from many IceCube analyses.
The wiki for [ the combined analysis] \todo{instead point to chapter}
This chapter presents the analysis work for IC3 for DM searches toward dSphs.
This section describes the various steps and features of the analysis.
It is structure first introduces the data and how it is treated, then systematic studies of the dwarves individually. Finally, the stacked analysis and results are presented.

%%%%%%%%%%%%%%%%%%%%%%%%%%%%%%%%%%%%%%%%%%%%%%%%%%%%%%%%%%%%%%%%%%%%%%%%%%%%%%%%%%%%%
\section{Dataset and Background}\label{sec:icDM_databgd}
%%%%%%%%%%%%%%%%%%%%%%%%%%%%%%%%%%%%%%%%%%%%%%%%%%%%%%%%%%%%%%%%%%%%%%%%%%%%%%%%%%%%%

This section enumerates the data and background methods used for IceCube's study of dSphs.
\Cref{sec:icDM_data} and \Cref{sec:icDM_tools} are most useful for fellow IceCube collaborators looking to replicate this analysis.

%%%%%%%%%%%%%%%%%%%%%%%%%%%%%%%%%%%%%%%%%%%%%%%
\subsection{Itemized IceCube files}\label{sec:icDM_data}
%%%%%%%%%%%%%%%%%%%%%%%%%%%%%%%%%%%%%%%%%%%%%%%
\begin{itemize}
    \item Software Environment: \texttt{CVMFS Py3-v4.1.1}
    \item Data Sample: Northern Tracks \texttt{NY86v5p1}
    \item Analysis Software: cksy (\href{https://github.com/icecube/csky/tree/nu\_dark\_matter}{nu\_dark\_matter})
    \item Analysis wiki: \url{https://wiki.icecube.wisc.edu/index.php/Dark\_Matter\_Annihilation\_Search\_towards\_dwarf\_spheroidals\_with\_NST\_and\_DNN\_Cascades}
    \item \href{https://github.com/salaza82/IceCube_dark_matter_dsph}{Project repository}
\end{itemize}

% %%%%%%%%%%%%%%%%%%%%%%%%%%%%%%%%%%%%%%%%%%%%%%%
\subsection{Software Tools and Development}\label{sec:icDM_tools}
%%%%%%%%%%%%%%%%%%%%%%%%%%%%%%%%%%%%%%%%%%%%%%%

This analysis was performed inside IceCube's CVMFS (3.4.1.1) software environment using csky for likelihood calculations.
Csky did not come with dark matter spectral models nor could accomodate custom flux models.
We developed these capacities for single source and stacked source studies for this analysis.
The analysis code is held in a separate repository from csky.
The \texttt{nu\_dark\_matter} \href{https://github.com/icecube/csky/tree/nu\_dark\_matter}{branch of csky} manages the input of custom dark matter spectra and accompanied DM astrophysical source then calculates likelihoods with a selected data sample.
The \href{https://github.com/salaza82/IceCube_dark_matter_dsph}{IceCube Dark Matter dSph repository} manages the generation of spectral models for neutrinos, physics parameter exctraction from $n_{\mathrm{sig}}$, \J-factor per source inputs, and bookkeeping for the large parameter space.
The project repository required a secondary software environment for neutrino oscillations.
How to launch and run those calculations are documented in the project repository and the Docker image is additionally saved in \cref{sec:apdx_nu_spec}

%%%%%%%%%%%%%%%%%%%%%%%%%%%%%%%%%%%%%%%%%%%%%%%
\subsection{Data Set and Background Description} \label{sec:icDM_data_bkgd}
%%%%%%%%%%%%%%%%%%%%%%%%%%%%%%%%%%%%%%%%%%%%%%%

For this analysis, I use the Northern Sky Tracks (NST) Sample (Version V005-P01).
The sample contains up-going track-like events, usually from $\nu_\mu$ and $\nu_\tau$ and has a superior angular resolution compared to the cascade dataset.
This sample covers 10.4 years of data (IC86\_2011-2021).
The accepted neutrino energy range used for the analysis is unique from most other IceCube searches because DM spectra are very hard.
The sampled energy range is $1 < \log(E_\nu /\textrm{GeV}) < 9.51$ with step size 0.125.

The strength of a dwarf analysis is that there is no additional background consideration beyond nominal, baseline background estimations.
For NST, the nominal contribution comes from atmospheric neutrinos and isotropic astrophysical neutrinos.
We estimate the background by scrambling NST data along Right Ascension.

%%%%%%%%%%%%%%%%%%%%%%%%%%%%%%%%%%%%%%%%%%%%%%%%%%%%%%%%%%%%%%%%%%%%%%%%%%%%%%%%%%%%%
\section{Analysis}\label{sec:icDM_analysis}
%%%%%%%%%%%%%%%%%%%%%%%%%%%%%%%%%%%%%%%%%%%%%%%%%%%%%%%%%%%%%%%%%%%%%%%%%%%%%%%%%%%%%

The expected differential neutrino flux from DM-DM annihilation to standard model
particles, $d\Phi_{\nu}/dE_{\nu}$, over solid angle, $\Omega$~is described by the familiar equation.
\iddmannilation[\nu]

This is identical to past examples except that there are 3 neutrino flavors, so there are a corresponding 3 flux equations.
\cref{sec:gd_analysis} has a complete description of all the terms.
Additionaly, neutrinos oscillate between flavors which needs to be considered for the expected neutrino flux at Earth.
\Cref{sec:icDM_particlephysics} presents the particle physics model for DM annihilation.
\Cref{sec:icDM_spatialmodel} presents the spatial distributions built for each dSph.

%%%%%%%%%%%%%%%%%%%%%%%%%%%%%%%%%%%%%%%%%%%%%%%%%%
\subsection{$\frac{dN_\nu}{dE_\nu}$ - Particle Physics Component}\label{sec:icDM_particlephysics}
%%%%%%%%%%%%%%%%%%%%%%%%%%%%%%%%%%%%%%%%%%%%%%%%%%

Neutrino spectra from heavy dark matter annihilation were generated using HDMSpectra \cite{HDMSpectra} and $\chi \textrm{aro}\nu$ \cite{Charon}.
HDMSpectra simulates the decay and annihilation of heavy dark matter, for different dark matter masses and SM primary annihilation channels.
The simulation includes electroweak radiative corrections and higher order loop corrections with quarks.
This publication also pushes the simulated DM mass to the Plank scale (1 EeV), however this study will not explore that high.

An important novel feature in the spectra is that neutrino line channels will be accompanied with a low energy tail.
Thus the earth will not fully attenuate a neutrino SM channel signal from high declination sources where the neutrino flux must first traverse through the Earth.
The SM annihilation channels that feature lines include all leptonic channels. ($\nu_{e,\mu,\tau}, e, \mu, \mathrm{and} \tau$)
We use \href{https://iopscience.iop.org/article/10.1088/1475-7516/2020/10/043}{ $\chi \mathrm{aro}\nu$} to propagate and oscillate the neutrinos from the source to Earth.
Because these sources are quite large in absolute terms, and also far (order 10 kpc or more), the resulting flavor spectra are the averages of the transition probabilities \cite{Charon}:
\nuOscMatrix

When calculating the expected contribution to $n_s$, only $ \nu_\mu, \nu_\tau $ are considered as NST's effective area to $ \nu_e $ is essentially 0.
With these consideration, the expected composite neutrino spectrum is a average of the two flavors: $ (\nu_\mu + \nu_\tau)/2 $.
The spectral tables are then converted to splines to condense information, enable random sampling of the spectra, and enable faster computation times.
The spectral splines are finally implemented as a DM class in csky.
Examples of the spectra before and after propagation are shown in \cref{fig:icDM_osc_dm}.
\afterpage{%
\begin{landscape}
\centering{
    \begin{table}
    \begin{tabular}{c c c c}
        $M_\chi$ &
        $\chi\chi \rightarrow$ \parpar{b} &
        $\chi\chi \rightarrow$ \parpar{\tau} &
        $\chi\chi \rightarrow$ \parpar{\nu_\mu} \\

        \rotatebox[origin=c]{90}{1 TeV} &
        \raisebox{-.5\height}{\includegraphics[scale=0.275]{figures/ic_DM/nu_spectra_bb_1.0000TeV.pdf}} &
        \raisebox{-.5\height}{\includegraphics[scale=0.275]{figures/ic_DM/nu_spectra_tautau_1.0000TeV.pdf}} &
        \raisebox{-.5\height}{\includegraphics[scale=0.275]{figures/ic_DM/nu_spectra_numunumu_1.0000TeV.pdf}} \\

        \rotatebox[origin=c]{90}{1 PeV} &
        \raisebox{-.5\height}{\includegraphics[scale=0.275]{figures/ic_DM/nu_spectra_bb_1000.0000TeV.pdf}} &
        \raisebox{-.5\height}{\includegraphics[scale=0.275]{figures/ic_DM/nu_spectra_tautau_1000.0000TeV.pdf}} &
        \raisebox{-.5\height}{\includegraphics[scale=0.275]{figures/ic_DM/nu_spectra_numunumu_1000.0000TeV.pdf}} \\

        \rotatebox[origin=c]{90}{1 EeV} &
        \raisebox{-.5\height}{\includegraphics[scale=0.275]{figures/ic_DM/nu_spectra_bb_100000.0000TeV.pdf}} &
        \raisebox{-.5\height}{\includegraphics[scale=0.275]{figures/ic_DM/nu_spectra_tautau_100000.0000TeV.pdf}} &
        \raisebox{-.5\height}{\includegraphics[scale=0.275]{figures/ic_DM/nu_spectra_numunumu_100000.0000TeV.pdf}} \\
    \end{tabular}
    \caption{Neutrino spectra at production (left panels) and after oscillation at Earth (right panels). Blue, orange, and green lines are the $\nu_e$, $\nu_\mu$, and $\nu_\tau$ spectra respectively. Top panels show the spectra in $\frac{dN}{dE}$. Lower panels plot the flavor ratio to $\nu_e + \nu_\mu + \nu_\tau$. SM annihilation channels \parpar{b}, \parpar{\tau}, and \parpar{\nu_\mu} are shown for $M_\chi =$ 1 Pev, TeV, and EeV.}
    \end{table}\label{fig:nu_osc_dm}
}
\end{landscape}
}

%%%%%%%%%%%%%%%%%%%%%%%%%%%%%%%%%%%%%%%%%%%%%%%%%%
\subsubsection{Treatment of Neutrino Line Features}\label{sec:icDM_nu_lines}
%%%%%%%%%%%%%%%%%%%%%%%%%%%%%%%%%%%%%%%%%%%%%%%%%%

All leptonic DM annihilation channels $ \chi\chi \rightarrow [\nu_{e, \mu, \tau}, e, \mu,\tau] $ develop a prominent and narrow spectral line feature.
For all neutrino flavors, this line is visible and prominent in all mass models studied for this analysis.
For charged leptons, the feature only really shows up at the larger DM mass models.
Examples for lines in both neutrinos and charged leptons annihilation are provided in \cref{fig:icDM_osc_dm}.
This line feature is so narrow relative the sampled energy range that the MC rarely samples within the neutrino line.
As a result, often the best fit to simulation of background will always floor to TS = 0 and the signal recovery tends to be conservatve.

\begin{figure}[t]
    \centering{
        \includegraphics[scale=0.5]{figures/ic_DM/Sig_recover_line.png}
    }
    \caption{Signal recovery for 100 TeV DM annihilation into \parpar{\nu_\mu} for a source at Dec = 16.06$^\circ$. $n_\mathrm{inj}$ is the number of injected signal events in simulation. $n_s$ is the number of reconstructed signal events from the simulation. Although the uncertainties are small and tight, the reconstructed $ n_s $ are systematically underestimated.} \label{fig:sig_recovery_fail}
\end{figure}


To remedy this, a similar approach to the IceCube's decay analysis \todo{refer to Minjin's page}.
2 kernels were tested (Gaussian, uniform (flat)) to smooth out the line feature.
The widths were tuned such that the signal recovery approached unity for DM mass 100 TeV to 1 PeV.
Additionally, the tuning was performed only for a source at declination 16.06 (Segue 1).
This is to avoid confusion loss in signal recovery from too narrow a line and from Earth's attenuation of high energy neutrinos.
The convolution also needed to as close as possible preserve the integrated counts of neutrinos.
The optimized kernel window for all lines is summarized as:
\begin{itemize}
    \item Guassian kernel w/$2 \sigma$ width = 3.5E-3$\cdot m_\chi$
    \item Minimum energy included in convolution = MIN$[0.995 \cdot m_\chi, En(\nu_{line}) -4\sigma]$
    \item Maximum energy included in convolution = MAX$[1.005 \cdot m_\chi, En(\nu_{line}) +4\sigma]$
\end{itemize}
where $En(\nu_{line})$ is the neutrino energy where the neutrino line is at the maximum.

\begin{figure}[h]
    \centering{

    \begin{tabular}{cc}
        \begin{tabular}{cc}
            \rotatebox[origin=c]{90}{\small{\todo{Check y axe}}} &
            \raisebox{-.5\height}{\includegraphics[scale=0.5]{figures/ic_DM/sig_recovery_fixed copy.png}} \\

            &
            \small{$E_\nu$ (GeV)} \\
        \end{tabular} &

        \raisebox{-.5\height}{\includegraphics[scale=0.5]{figures/ic_DM/sig_recovery_fixed.png}} \\
    \end{tabular}
    }\caption{Top left panel shows the two kernels overlayed the original spectrum from Charon. delta I is the difference in the integral of the peaks with respect to the original spectrum. The vertical red line indicated where the original neutrino line is maximized. Lower right shows the signal recovers of the DM model using the Gaussian kernel with parameters enumerated above.} \label{fig:icDM_fixed_line}
\end{figure}

These parameters broadly improved the signal recovery of the line spectra.
An example is provided below.
Signal recovery plots of the full analysis are provided much further down.

%%%%%%%%%%%%%%%%%%%%%%%%%%%%%%%%%%%%%%%%%%%%%%%%%%
\subsubsection{Spline Fitting}\label{sec:icDM_splines}
%%%%%%%%%%%%%%%%%%%%%%%%%%%%%%%%%%%%%%%%%%%%%%%%%%

In an effort to reduce computational work, memory burden, and align with point source methods used for NGC1068 and Seyfert analyses, spectral splines were created and adopted for estimating the neutrino flux for the different annihilation.
Software was written to generate, handle, and calculate values on the splines.
When using splines, one has to be careful of the goodness to fit.
There are critical caveats when testing the goodness to fit to MC generated above for all channels.
\begin{itemize}
    \item The splines must be Log10(*) in Energy and dN/dE to acount for the exponential nature of the flux
    \item The fidelity of the fit matters more at $ E_\nu \thickapprox m_\chi $ where the model uncertainties are minimal and physical considerations (like the cut-off) are most apparent.
    \item The fidelity of the fit matters less at low $ E_\nu $ as the model uncertainties are large AND IceCube's sensitivity diminishes significantly below 500 GeV
    \item Total integrated counts should be well preserved, however, the resolution of the MC is much higher than IceCube's energy resolution.
    \begin{itemize}
        \item Meaning over several steps in E, the integral is preserved
        \item the step size enters the cost function
        \item Oscillating residuals, so long as they are very small and well centered, are not penalized as this gets averaged out.
    \end{itemize}
\end{itemize}

The resulting cost function to evaluate the goodness of fit was used to account for the above considerations.
\erriSpline

Where $ \hat{e_i} $ is the spline estimator's value for $x_i$. $ x_i = E_{\nu_i} / m_\chi $. $ \frac{dN_i}{dE_i} $ is the flux value from MC.
\MSEspline

I then take the RMS of the error distribution and the resulting value (err) is used to evaluate the fidelity of the spectral spline.
Each SM channel had different tolerances for 'err'. Channels with very hard cut-offs had looser tolerance for err because a lot of error would be generated from the cut-off being estimated to occur slightly early or late.
Soft channels don't have this issue and therefore the tolerance is very strict.
The table blow summarizes the tolerances for the SM channels.
\begin{table}
    \centering{
    \begin{tabular}{c|c c c c}
        \hline
        $ \chi\chi \rightarrow $ &
        GOOD &
        OK   &
        FAIL &
        Limits of err calc [$X_{min}, X_{max}$] \\
        \hline
        \hline

        $ Z^0Z^0, W^+W^- $ &
        1.0E-3 &
        1.0E-3, 1.0E-2 &
        1.0E-2 &
        MAX$[100 \mathrm{GeV}/m_\chi, 10^{-6}], 1.0 $ \\

        $ t\bar{t}, hh $ &
        1.0E-5 &
        1.0E-5, 1.0E-4 &
        1.0E-4 &
        MAX$[100 \mathrm{GeV}/m_\chi, 10^{-6}], 1.0 $ \\

        $ b\bar{b}, d\bar{d}, u\bar{u}$ &
        9.0E-7 &
        9.0E-7, 9.0E-6 &
        9.0E-6 &
        MAX$[100 \mathrm{GeV}/m_\chi, 10^{-6}], 1.0 $ \\
        \hdashline

        $ \nu\bar{\nu}_{e, \mu, \tau} $ &
        1.0E-3 &
        1.0E-3, 1.0E-2 &
        1.0E-2 &
        \makecell{MAX$[100 \mathrm{GeV}/m_\chi, 10^{-6}]$, \\ MIN$[0.995, (En(\nu_{line}) -4\sigma)/M_{\chi}] $} \\
        \hdashline

        $ e\bar{e}, \mu\bar{\mu}, \tau\bar{\tau} $ &
        1.0E-3 &
        1.0E-3, 1.0E-2 &
        1.0E-2 &
        \makecell{MAX$[100 \mathrm{GeV}/m_\chi, 10^{-6}]$, \\ MIN$[0.995, (En(\nu_{line}) -4\sigma)/M_{\chi}]$} \\

    \end{tabular}
    }
    \caption{\todo{fill me daddy}} \label{tab:spline_tolerance}
\end{table}

The errors are then plotted in two ways.
First, FAIL and OK are directly plotted with $e_i$ as a function of x, and the full spline and MC.
Second, a summary plot of all the splines is plotted and colors coded.

\begin{figure}
    \centering{
    \includegraphics[scale=0.6]{figures/ic_DM/800px-Failed_spline.png}
    }
    \caption{Example spline that failed the fit. Failed splined are corrected on a case by case basis unless the SM channel has a systematic problem fitting the splines. In this case, I made a bookkeeping error and loaded the incorrected neutrino flavor}
    \label{fig:icDM_failedspline}
\end{figure}

\Cref{fig:apdx_nu_splines} are the spline summaries and represent the current, up-to-date status of the splines.
The goal broadly is to eliminate all red and inspect yellow. $ \nu_e $ is not considered in this analysis among the neutrino final states and so no work was done to converge the spline fits for this flavor.

A Final inspection of the splines by eye was done to verify that the spline fitting did not introduce spurious features into the distribution that would corrupt the LLH fitting.

%%%%%%%%%%%%%%%%%%%%%%%%%%%%%%%%%%%%%%%%%%%%%%%%
\subsubsection{Composite Neutrino Spectra}\label{sec:icDM_composite_dNdE}
%%%%%%%%%%%%%%%%%%%%%%%%%%%%%%%%%%%%%%%%%%%%%%%

With all of the previously mentioned pieces, we are ready to fully assemble a comprehensive description of the particle physics term $dN/dE$ in \cref{eq:id_dm_flux}.
\nuIDDMFlux

Presented below are the final spectra that are used in the DM analysis. Bluer spectra are for lower DM mass models. The redder, the higher the DM mass.
Energy (x-axis) was chosen to roughly represent the energy sensitivity of NST.
These spectra are the composite (nu and tau flavors) versions. How these are combined is mentioned earlier.

\tmpfig{Show the post processed spectra that you are sampling}

%
% <center>
% {|class="wikitable" style="border-spacing: 2px; border: 1px solid darkgray;"
% |+ Output composite spectra after spline fit and line smoothing
% |-
% | <center> $ \chi\chi \rightarrow b\bar{b} $ </center>
% | <center> $ \chi\chi \rightarrow t\bar{t} $ </center>
% | <center> $ \chi\chi \rightarrow \nu_{\mu}\overline{\nu_{\mu}} $ </center>
% |-
% | [[Image:bb_rainbow.png |300px]]
% | [[Image:tt_rainbow.png |300px]]
% | [[Image:numu_rainbow.png |300px]]
% |-
% | <center> $ \chi\chi \rightarrow W^+ W^- $ </center>
% | <center> $ \chi\chi \rightarrow Z^0Z^0 $ </center>
% | <center> $ \chi\chi \rightarrow \nu_{\tau}\overline{\nu_{\tau}} $ </center>
% |-
% | [[Image:ww_rainbow.png |300px]]
% | [[Image:zz_rainbow.png |300px]]
% | [[Image:nutau_rainbow.png |300px]]
% |-
% | <center> $ \chi\chi \rightarrow \nu_{e}\overline{\nu_{e}} $ </center>
% |-
% | [[Image:nue_rainbow.png |300px]]
% |}
% </center>
%
%%%%%%%%%%%%%%%%%%%%%%%%%%%%%%%%%%%%%%%%%%%%%%%%%%
\subsection{\J - Astrophysical Component}\label{sec:icDM_spatialmodel}
%%%%%%%%%%%%%%%%%%%%%%%%%%%%%%%%%%%%%%%%%%%%%%%%%%

The expected neutrino counts from a dwarf spheroidal galaxy depends also on the the 'astrophysical factor'.
The value for this (in our specific case) J-factor for a target depends on its dark matter density distribution, $\rho_{\chi}$ and how far it is $ l $.
For this analysis, we adopt the \GS model used in \cref{sec:glory_duck} for dSph from \cite{Geringer_Sameth_2015}.
These models are based on a modified Navarro-Frenk-While (NFW) profile where the indices of the NFW (traditionally 1,3,1) are allowed to float.
More specifically, these DM distributions are described using the Zhao profile.
The Zhao profile is written as:

%
% <center>
% $
% \rho_{Zhao}(r; r_{s}, \rho_{s}, \alpha, \beta, \gamma) = \frac{\rho_{s}}{\left(\frac{r}{r_{s}}\right)^{\gamma} \left[1 + \left(\frac{r}{r_{s}}\right)^{\alpha} \right]^{(\beta-\gamma)/\alpha}  }
% $,
% </center>
% where $ r_{s} $, $\rho_{s}$ denote scale radius, scale density respectively. According to these models, the dark matter distributions of the sources are spherically symmetric. Hence, the J-factor is calculated as the following :
% <center>
% $
%
% J(\theta) = 2 \pi \int_{0}^{\theta} \sin \theta^{'} d\theta^{'} \int^{\infty}_{0}dl^{'} \rho^2_{\chi}(r[l^{'}, \theta^{'}])
%
% $,
% </center>

where $ \theta $ is the angular distance from the center of the source.
For the case annihilation, the source diameter, [https://iopscience.iop.org/article/10.1088/0004-637X/801/2/74 here] defined as the $ 2 \theta_{\mathrm{max}} $ , of these dwarves is typically under $ 1^{\circ} $ with the largest in the catalog, Fornax, extending to $ 2.61^{\circ} $.
Fornax is not in the northern sky and the remaining sources are notably below this angular size.
Therefore, the sources are treated as point sources because the typical source diameter is under 1 degree.
The J-factor used for the point source assumption is the total J emitted from $ \theta_{\mathrm{max}} $.
These values are enumerated in Geringer-Sameth 2015 and again in the table below with their coordinates.
Coordinates are given in J200.0 equatorial coordinates.
IceCube uses identical sources to \cref{tab:gd_J_factor} except we analyze source with declinations above 0.0 degrees.

%%%%%%%%%%%%%%%%%%%%%%%%%%%%%%%%%%%%%%%%%%%%%%%%%%
\subsection{Source Selection and Annihilation Channels}\label{sec:ic3_study_selection}
%%%%%%%%%%%%%%%%%%%%%%%%%%%%%%%%%%%%%%%%%%%%%%%%%%

We use all of the dSphs presented in IceCube's previous dSph DM search \cite{IC3_DM2013}.
IceCube's sources for these simulation studies include Bootes I, Canes VenaticiI, Canes Venatici II, Coma Berenices, Draco, Hercules, Leo I, Leo II, Leo V, Leo T, Segue 1, Segue 2, Ursa Major I, Ursa Major II,  and Ursa Minor.
A full description of all sources used in \Cref{tab:gd_J_factor}.
Sources with declinations less than 0.0 are excluded from this analysis.

This analysis improves on the previous IceCube dSph paper \cite{IC3_DM2013} in the following ways.
Previously, the IceCube detector was not yet completed to the 86 string configuration.
Many more dSphs will be observed, from 4 to 15.
Previously, the particle physics model used for neutrino-ray spectra from DM annihilation did not have EW corrections where they are now included \cite{HDMSpectra}.
The spectral models also predict substantial differences between the neutrino flavors, so this analysis will be the first DM dwarf analysis to discriminate betwen primary neutrino flavors.
The study performed here studies 10.4 years of data.

The SM annihilation channels probed for this study include \parpar{b}, \parpar{t}, \parpar{u}, \parpar{d}, \parpar{e}, \parpar{\mu}, \parpar{\tau}, \pp{Z}, $W^+W^-$, \parpar{\nu_e}, \parpar{\nu_\mu}, and \parpar{\nu_\tau}.

%%%%%%%%%%%%%%%%%%%%%%%%%%%%%%%%%%%%%%%%%%%%%%%%%%
\section{Likelihood Methods}\label{sec:icDM_LLH}
%%%%%%%%%%%%%%%%%%%%%%%%%%%%%%%%%%%%%%%%%%%%%%%%%%

I use the Point-Source search likelihood which is widely used in IceCube analyses.
The likelihood function is defined as the following:
\icPtSrcLLH
where  $ i $ is an event index, $S$ and $B$ are the signal PDF and background PDF respectively. For a joint analysis where the sources are stacked the likelihood is expanded in the simplified way:
\icStackLLH
Where $ L_i $ is the likelihood from the i-th source in the stacked analysis.
The test statistic definition remains the same as \cref{eq:gd_TS}


%%%%%%%%%%%%%%%%%%%%%%%%%%%%%%%%%%%%%%%%%%%%%%%%%%
\section{Background Simulation}\label{sec:icDM_bkgd_sim}
%%%%%%%%%%%%%%%%%%%%%%%%%%%%%%%%%%%%%%%%%%%%%%%%%%

Before we look at data, we must first analyze background and signal injection to validate our anlysis.
The following sections show the reults of the likelihood fitting for a suite of background trials for the DM models we set out to study in \todo{refer to the section}.
We study the TS distributions first for each source, then for the stacked analysis.

The TS distributions are not expected to behalf according to a chi-squared distribution with 1 degree of freedom.
This is in large part due to the distinct spectral shapes demonstrated earlier.
These can vary significantly between DM mass and annihilation models.
Therefor, Wilks' theorem may not be applicable to the analysis.
Instead, a critical value is defined from a large number of background trials.

I assume that TS values are physical: $ \mathrm{TS} \ge 0 $.
$\eta$ denotes the fraction of positive TS values above the threshold and written in the legend.
$ \epsilon[x] $ indicate the fraction of events where $ \mathrm{TS} < x $. For TS plots shown here, the decimal values of x are 1.0e-2 and 1.0e-3.
The following plots show the background TS distributions obtained from Segue1, a source with little Earth attenuation and large J-factor, assuming that dark matter annihilates into $b\overline{b}$.
I also show the 15 source stack TS distributions with identical DM models.

%%%%%%%%%%%%%%%%%%%%%%%%%%%%%%%%%%%%%%%%%%%%%%%%%%
\subsection{TS per Source} \label{sec:icDM_TSperSrc}
%%%%%%%%%%%%%%%%%%%%%%%%%%%%%%%%%%%%%%%%%%%%%%%%%%

Below I present the TS distributions for Segue1 and $ \chi\chi \rightarrow$ \parpar{b}. All remaining channels and source TS panels are hosted on \todo{Change this text, it will all be here}.

Although it was not expected, almost every distribution produced follows a chi2 distribution with 1 degree of freedom.
This is important for future assumptions made (in multi-messenger) and may justify statistical calculations assuming Wilk's theorem is valid.

\todo{add text saying that you show: bb, numu, and tau??? specs for Seg1 and UMa2? }

\begin{figure}[t]
    \centering{
        \includegraphics[clip, trim=5.7cm 5.2cm 5.7cm 6.7cm, scale=0.326]{figures/ic_DM/dm_plots/Segue1_bb_chi2_Masspanel.pdf}
    }\caption{\todo{update text because there is no title.} Each subplot, except the final, is the TS distribution for a specific DM mass listed in the subplot. The final subplot plots the all DM spectral models used as input for the TS distribution calculations with bluer lines indicating lower DM mass and redder indicating higher DM mass.}
    \label{fig:icDM_Seg1bb_TS}
\end{figure}

\begin{figure}[t]
    \centering{
        \includegraphics[clip, trim=5.7cm 5.2cm 5.7cm 6.7cm, scale=0.326]{figures/ic_DM/dm_plots/UrsaMajorII_bb_chi2_Masspanel.pdf}
    }\caption{\todo{update text because there is no title.} Each subplot, except the final, is the TS distribution for a specific DM mass listed in the subplot. The final subplot plots the all DM spectral models used as input for the TS distribution calculations with bluer lines indicating lower DM mass and redder indicating higher DM mass.}
    \label{fig:icDM_UMa2bb_TS}
\end{figure}

\begin{figure}[t]
    \centering{
        \includegraphics[clip, trim=5.7cm 5.2cm 5.7cm 6.7cm, scale=0.326]{figures/ic_DM/dm_plots/Segue1_tautau_chi2_Masspanel.pdf}
    }\caption{\todo{update text because there is no title.} Each subplot, except the final, is the TS distribution for a specific DM mass listed in the subplot. The final subplot plots the all DM spectral models used as input for the TS distribution calculations with bluer lines indicating lower DM mass and redder indicating higher DM mass.}
    \label{fig:icDM_Seg1tau_TS}
\end{figure}

\begin{figure}[t]
    \centering{
        \includegraphics[clip, trim=5.7cm 5.2cm 5.7cm 6.7cm, scale=0.326]{figures/ic_DM/dm_plots/UrsaMajorII_tautau_chi2_Masspanel.pdf}
    }\caption{\todo{update text because there is no title.} Each subplot, except the final, is the TS distribution for a specific DM mass listed in the subplot. The final subplot plots the all DM spectral models used as input for the TS distribution calculations with bluer lines indicating lower DM mass and redder indicating higher DM mass.}
    \label{fig:icDM_UMa2tau_TS}
\end{figure}

\begin{figure}[t]
    \centering{
        \includegraphics[clip, trim=5.7cm 5.2cm 5.7cm 6.7cm, scale=0.326]{figures/ic_DM/dm_plots/Segue1_numunumu_chi2_Masspanel.pdf}
    }\caption{\todo{update text because there is no title.} Each subplot, except the final, is the TS distribution for a specific DM mass listed in the subplot. The final subplot plots the all DM spectral models used as input for the TS distribution calculations with bluer lines indicating lower DM mass and redder indicating higher DM mass.}
    \label{fig:icDM_Seg1nutau_TS}
\end{figure}

\begin{figure}[t]
    \centering{
        \includegraphics[clip, trim=5.7cm 5.2cm 5.7cm 6.7cm, scale=0.326]{figures/ic_DM/dm_plots/UrsaMajorII_nutaunutau_chi2_Masspanel.pdf}
    }\caption{\todo{update text because there is no title.} Each subplot, except the final, is the TS distribution for a specific DM mass listed in the subplot. The final subplot plots the all DM spectral models used as input for the TS distribution calculations with bluer lines indicating lower DM mass and redder indicating higher DM mass.}
    \label{fig:icDM_UMa2nutau_TS}
\end{figure}
%
% </center>
%
% ==== stacked TS ====
%
% The presentation of these plots are identical to the previous 'per Source' section. I use csky source software to calculate the TS distributions.
% Bugs were found when implementing, however were rectified. Warning to future users performing a stacked analysis with custom spectra.
% In using the above, I am making the implicit assumption that the primary/only cause to a difference in neutrino counts from the sources is accounted for through the J-factors.
% The J-factors are therefor used as weights for the stacking where an individual source's weight is defined as:
%
% <center>
% <big>$ w_i = \frac{J_i}{\sum^{N_\mathrm{sources}}_{k=1}J_k} $</big>
% </center>
%
% where $ w_i $ is the ith source, $ J_{i/k} $ is the i/k-th source's J-factor.
%
% Below is the TS distribution for each SM annihilation channel with stacking of 15 sources.
%
% [[Media: stacked_bb_chi2_Masspanel.pdf | $ b\overline{b} $ 1/7/2024]]
%
% [[Media: stacked_ww_chi2_Masspanel.pdf | $W^+W^- $ 1/7/2024]]
%
% [[Media: stacked_zz_chi2_Masspanel.pdf | $Z^0 Z^0 $ 1/7/2024]]
%
% [[Media: stacked_tt_chi2_Masspanel.pdf | $ t\overline{t} $ 1/7/2024]]
%
% [[Media: stacked_nuenue_chi2_Masspanel.pdf | $ \nu_e\overline{\nu_e} $ 1/7/2024]]
%
% [[Media: stacked_numunumu_chi2_Masspanel.pdf | $ \nu_\mu\overline{\nu_\mu} $ 1/7/2024]]
%
% [[Media: stacked_nutaunutau_chi2_Masspanel.pdf | $ \nu_\tau\overline{\nu_\tau} $ 1/7/2024]]
%
% Each subplot, except the final, is the TS distribution for a specific DM mass listed in the subplot. The final subplot plots the all DM spectral models used as input for the TS distribution calculations with bluer lines indicating lower DM mass and redder indicating higher DM mass.
% Below is an image of bb. The full resolution pdfs were provided in links above.
%
% <center>
% [[Image: stacked_bb_chi2_Masspanel.jpeg| 800px ]]
%
% </center>
%
% === Signal Recovery ===
%
% This did have a lot of plots before but the spectra have change substantially since then. They converge well but since the computational is expensive here, I'm holding off on rerunning until the spectra are finalized.
%
% === Sensitivities ===
%
% In IceCube, we usually define the 90% confidence level (CL), as the minimum number of signal events (n_s) required to have a Type I error rate smaller than 0.5 and Type II error rate of 0.1.
% Csky performs the sweep to find n_s that satisfies the previous condition, and from n_s I use the following equation
%
% <center>
% $
% n_{s} = T_{live} \int_{0}^{\Delta\Omega}d\Omega \int_{E_{min}}^{E_{max}} dE_{\nu} A_{eff}(\hat{n}, E_{\nu}) \frac{d \Phi_{\nu}}{d\Omega d E_{\nu}} (\hat{n}, E_{\nu}),
% $
% </center>
%
% to extract the sensitivity on the dark matter annihilation cross-section.
% $ T_{live} $ is the detector livetime, $ A_{eff} $ is the effective area of the detector, and $ E_{min} $, $ E_{max} $ are the minimum, maximum energies of the expected neutrinos, respectively.
%
% Sensitivities are calculated for each source individually as if they were the only source and as a stack.
% Example plots of these plots are shown below and organized by the single source/stacked studies.
% Finally, I generated a plot with all hypotheses which is presented at the very end.
%
% ==== per Source ====
%
% ==== Stacked ====
%

%%%%%%%%%%%%%%%%%%%%%%%%%%%%%%%%%%%%%%%%%%%%%%%%%%%%%%%%%%%%%%%%%%%%%%%%%%%%%%%%%%%%%
\chapter{Nu Duck: Conclusions and Future Directions}\label{sec:nu_duck}
%%%%%%%%%%%%%%%%%%%%%%%%%%%%%%%%%%%%%%%%%%%%%%%%%%%%%%%%%%%%%%%%%%%%%%%%%%%%%%%%%%%%%
%%%%%%%%%%%%%%%%%%%%%%%%%%%%%%%%%%%%%%%%%%%%%%%%%%%%%%%%%%%%%%%%%%%%%%%%%%%%%%%%%%%%%
\section{Conclusions}\label{sec:conclusions}
%%%%%%%%%%%%%%%%%%%%%%%%%%%%%%%%%%%%%%%%%%%%%%%%%%%%%%%%%%%%%%%%%%%%%%%%%%%%%%%%%%%%%

\todo{Chat GPT the shit of everything below}
In this work, three analyses were performed with data from the HAWC and IceCube observatories in order to explore some of the fundamental questions in particle astrophysics.
Thegoal was to contribute to the understanding of the sources of cosmic rays, their acceleration mechanisms, and the nature of dark matter. The detection techniques and reconstruction methods for both observatories were described, along with the properties that make them ideal instruments to perform such searches.

This dissertation used data from the HAWC detector to probe cutting-edge physics beyond the Standard Model.
The techniques by which HAWC is able to detect cosmic gamma rays were
demonstrated and the many advantages of HAWC in probing ultra-high energy gamma-ray physics were detailed.
It was shown how HAWC data can be used to explore unanswered questions such as the nature of dark matter and the limits of Lorentz invariance.
In particular, a search for evidence of WIMP dark matter in the Milky Way Galactic Halo was performed.
To accomplish this, simulations of the dark matter density profile were combined with estimates of the HAWC sensitivity to dark matter-like energy spectra.
This allowed strong constraints on dark matter annihilation and decay from the Galactic Halo to be derived that are insensitive to the large uncertainties arising from systematics in the dark matter spatial distribution.
Multi-hundred TeV photon spectra were also significantly detected from HAWC sources within the Galactic Plane. These results lead to the strongest constraints on Lorentz invariance violation to be published at the time of writing.

The work of this dissertation was made possible by the ongoing development of new algorithms and reconstruction techniques within the HAWC collaboration.
Probing the Galactic Halo required the creation of a novel background estimation technique that relied on HAWC’s wide field of view and strong ability to discriminate between gamma rays and cosmic rays.
Meanwhile, the constraints on Lorentz invariance violation were enabled by the improved energy resolution from a machine learning technique.
HAWC has recently completed a reprocessing of all archival data using an updated set of algorithms that can lead to compelling follow-up work on these results.
Combining the new background technique with the re-optimized energy estimators will allow for Galactic dark matter to be probed at even higher masses, as well as for analyses that require precise energy resolution such as gamma-ray line searches.

%%%%%%%%%%%%%%%%%%%%%%%%%%%%%%%%%%%%%%%%%%%%%%%%%%%%%%%%%%%%%%%%%%%%%%%%%%%%%%%%%%%%%
\section{Future Directions: Multi-Messenger Dark Matter Search}\label{sec:future}
%%%%%%%%%%%%%%%%%%%%%%%%%%%%%%%%%%%%%%%%%%%%%%%%%%%%%%%%%%%%%%%%%%%%%%%%%%%%%%%%%%%%%

\begin{figure}[h]
    \centering{
        \includegraphics[scale=0.4]{figures/hdm_gamma_nu.png}
    }
    \caption{The prompt electron neutrino and photon spectrum resulting from the decay of a 2EeV DM particle to \parpar{\nu_e}, as currently being searched for at IceCube [5]. Solid curves represent the results of this work, and predict orders of magnitude more flux at certain energies than the dashed results of Pythia 8.2, one of the only existing methods to generate spectra at these masses. In both cases energy conservation is satisfied: there is a considerable contribution to a $\delta$-function at x = 1, associated with events where an initial W or Z was never emitted and thus no subsequent shower developed. Large disagreements are generically observed at these masses for electroweak dominated channels, while the agreement is better for colored initial SM states.}
    \label{fig:nu_and_gam}
\end{figure}

As I have shown previously in \cref{sec:glory_duck} and \cref{sec:multithread}, we can build a fast and robust analysis that shares tools with the field.
The hope being that IceCube can eventually combine data with gamma-ray observatories.
\tmpfig{nuetrino and bb plot with nu Sensitivities}

\SingleSpacing

\printbibliography

\DoubleSpacing

\begin{appendices}

%%%%%%%%%%%%%%%%%%%%%%%%%%%%%%%%%%%%%%%%%%%%%%%%%%%%%%%%%%%%%%%%%%%%%%%%%%%%%%%%%%%%%
% \chapter{Multi-Experiment Supplementary Figures}
%%%%%%%%%%%%%%%%%%%%%%%%%%%%%%%%%%%%%%%%%%%%%%%%%%%%%%%%%%%%%%%%%%%%%%%%%%%%%%%%%%%%%
\chapter{Multi-Experiment Supplementary Figures}\label{apdx:gd_spatial_maps}

\begin{figure}[h]
    \centering{
        \includegraphics[scale=0.33]{figures/glory_duck/hawc/GD_mass_profiles/BootesI_J_plot.pdf}
        \includegraphics[scale=0.33]{figures/glory_duck/hawc/GD_mass_profiles/CanesVenaticiI_J_plot.pdf}
        \includegraphics[scale=0.33]{figures/glory_duck/hawc/GD_mass_profiles/CanesVenaticiII_J_plot.pdf}
        \includegraphics[scale=0.33]{figures/glory_duck/hawc/GD_mass_profiles/Draco_J_plot.pdf}
        \includegraphics[scale=0.33]{figures/glory_duck/hawc/GD_mass_profiles/Hercules_J_plot.pdf}
        \includegraphics[scale=0.33]{figures/glory_duck/hawc/GD_mass_profiles/LeoI_J_plot.pdf}
        \includegraphics[scale=0.33]{figures/glory_duck/hawc/GD_mass_profiles/LeoII_J_plot.pdf}
        \includegraphics[scale=0.33]{figures/glory_duck/hawc/GD_mass_profiles/LeoIV_J_plot.pdf}
        \includegraphics[scale=0.33]{figures/glory_duck/hawc/GD_mass_profiles/Sextans_J_plot.pdf}
        \includegraphics[scale=0.33]{figures/glory_duck/hawc/GD_mass_profiles/UrsaMajorI_J_plot.pdf}
        \includegraphics[scale=0.33]{figures/glory_duck/hawc/GD_mass_profiles/UrsaMajorII_J_plot.pdf}
    }
    \caption{Sister figure to \Cref{fig:gd_spatialmodel}. Sources in the first row from left to right: Bootes I, Canes Venatici I, II. In second row: Draco, Hercules, Leo I. In the first row: Leo II, Leo IV, Sextans. In the final row: Ursa Major I, Ursa Major II.} \label{fig:apx_gd_spatialmodels}
\end{figure}

%%%%%%%%%%%%%%%%%%%%%%%%%%%%%%%%%%%%%%%%%%%%%%%%%%%%%%%%%%%%%%%%%%%%%%%%%%%%%%%%%%%%%
% \chapter{Multithreading Dark Matter Analyses Supplementary Material}
%%%%%%%%%%%%%%%%%%%%%%%%%%%%%%%%%%%%%%%%%%%%%%%%%%%%%%%%%%%%%%%%%%%%%%%%%%%%%%%%%%%%%
\begin{figure}[h]
    \centering{
    \begin{tabular}{ccc}
        \includegraphics[scale=0.27]{figures/mtd_hawc_dm/hdm_bb.png} &
        \includegraphics[scale=0.27]{figures/mtd_hawc_dm/hdm_cc.png} &
        \includegraphics[scale=0.27]{figures/mtd_hawc_dm/hdm_ee.png} \\
        \includegraphics[scale=0.27]{figures/mtd_hawc_dm/hdm_dd.png} &
        \includegraphics[scale=0.27]{figures/mtd_hawc_dm/hdm_gg.png} &
        \includegraphics[scale=0.27]{figures/mtd_hawc_dm/hdm_hh.png} \\
        \includegraphics[scale=0.27]{figures/mtd_hawc_dm/hdm_mumu.png} &
        \includegraphics[scale=0.27]{figures/mtd_hawc_dm/hdm_tautau.png} &
        \includegraphics[scale=0.27]{figures/mtd_hawc_dm/hdm_tt.png} \\
        \includegraphics[scale=0.27]{figures/mtd_hawc_dm/hdm_ss.png} &
        \includegraphics[scale=0.27]{figures/mtd_hawc_dm/hdm_uu.png} &
        \includegraphics[scale=0.27]{figures/mtd_hawc_dm/hdm_ww.png} \\
        \includegraphics[scale=0.27]{figures/mtd_hawc_dm/hdm_nuenue.png} &
        \includegraphics[scale=0.27]{figures/mtd_hawc_dm/hdm_numunumu.png} &
        \includegraphics[scale=0.27]{figures/mtd_hawc_dm/hdm_nutaunutau.png} \\
    \end{tabular}
    }\caption{Sister figure to \Cref{fig:hdm_gamma_lines} for remaining SM primary annihilation channels studied for this thesis. These did not require any post generation smoothing and so are directly pulled from \cite{HDMSpectra} with a binning scheme most helpful for a HAWC analysis.
    }\label{fig:apdx_mtd_spectra}
\end{figure}

\clearpage
%%%%%%%%%%%%%%%%%%%%%%%%%%%%%%%%%%%%%%%%%%%%%%%%%%%%%%%%%%%%%
\section{\texttt{mpu\_analysis.py}}\label{sec:apx_mpu_script}
%%%%%%%%%%%%%%%%%%%%%%%%%%%%%%%%%%%%%%%%%%%%%%%%%%%%%%%%%%%%%

\lstinputlisting[language=Python]{appendices/mpu_analysis.py}

%%%%%%%%%%%%%%%%%%%%%%%%%%%%%%%%%%%%%%%%%%%%%%%%%%%%%%%%%%%%%%%%%%%%%%%%%%%%%%%%%%%%%
% \chapter{IceCube Heavy Dark Matter Analysis Supplementary Material}\label{apdx:ic3DM_supp}
%%%%%%%%%%%%%%%%%%%%%%%%%%%%%%%%%%%%%%%%%%%%%%%%%%%%%%%%%%%%%%%%%%%%%%%%%%%%%%%%%%%%%
%%%%%%%%%%%%%%%%%%%%%%%%%%%%%%%%%%%%%%%%%%%%%%%%%%%%%%%%%%%%%
\section{Docker Image for Oscillating Neutrino Spectra}\label{sec:apdx_nu_spec}
%%%%%%%%%%%%%%%%%%%%%%%%%%%%%%%%%%%%%%%%%%%%%%%%%%%%%%%%%%%%%
\lstinputlisting[language=bash]{appendices/Dockerfile.txt}

\clearpage

%%%%%%%%%%%%%%%%%%%%%%%%%%%%%%%%%%%%%%%%%%%%%%%%%%%%%%%%%%%%%
\section{Spline Fitting Statuses} \label{sec:apdx_nu_splines}
%%%%%%%%%%%%%%%%%%%%%%%%%%%%%%%%%%%%%%%%%%%%%%%%%%%%%%%%%%%%%

\begin{figure}[ht]
    \centering{
        \includegraphics[scale=0.32]{figures/ic_DM/PID14_mse_error_chart.pdf}
        \includegraphics[scale=0.32]{figures/ic_DM/PID16_mse_error_chart.pdf}
    }
    \caption{\todo{fill me daddy}}
    \label{fig:apdx_nu_splines}
\end{figure}

\clearpage
%%%%%%%%%%%%%%%%%%%%%%%%%%%%%%%%%%%%%%%%%%%%%%%%%%%%%%%%%%%%%
\section{Segue 1 And Ursa Major II Signal Recovery} \label{sec:apdx_TS_per_src}
%%%%%%%%%%%%%%%%%%%%%%%%%%%%%%%%%%%%%%%%%%%%%%%%%%%%%%%%%%%%%
 \tmpfig{Fill this out eventually. I think I want all the plots generated first}

\end{appendices}

\end{document}