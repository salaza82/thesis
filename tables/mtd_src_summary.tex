\begin{table}[b]
\centering
    \small{\begin{tabular}{cccc}
    \hline
    \hline
    \CellTopTwo{}
    Name & Distance & $l, b$ & $\log_{10}J$~(\LS set)\\
    & \scriptsize{(kpc)} &  \scriptsize{($\degree$)} & \scriptsize{$\log_{10}(\mathrm{GeV}^2 \mathrm{cm}^{-5}\mathrm{sr})$} \\
    \hline
    \CellTopTwo{}
    Coma Berenices & $44$ & $241.89,\: 83.61$ & $19.00^{+0.36}_{-0.35}$ \\
    \CellTopTwo{}
    Draco & $76$ & $86.37,\: 34.72$ & $18.83^{+0.12}_{-0.12}$  \\
    \CellTopTwo{}
    Segue I & $23$ & $220.48,\: 50.43$ & $19.12^{+0.49}_{-0.58}$ \\
    \CellTopTwo{}
    Sextans & $86$ & $243.50,\: 42.27$ & $17.73^{+0.13}_{-0.12}$ \\
    \hline
    \hline
    \CellTopTwo{}
\end{tabular}}
    \caption{Summary of the relevant properties of the dSphs used in the present work. Column 1 lists the dSphs. Columns 2 and 3 present their heliocentric distance and galactic coordinates, respectively. Column 4 reports the \J-factors of each source given from the \LS studies and estimated $\pm 1\sigma$ uncertainties. The values $\log_{10}J$~(\LS set) \cite{DM_Strigari20} correspond to the mean \J-factor values  for a source extension truncated at $0.5^\circ$.}
    \label{tab:mtd_J_factor}
\end{table}