\begin{table}[h]
    \centering
    {\begin{tabular}{c ? c  c  c  c }
    & \multicolumn{4}{c}{$T_{p,c}$ (hr:min:s)} \\
    \hline
    \hline
    M Tasks & $T_{1,1}$ & $T_{1,2}$ & $T_{1,8}$ & $T_{1,16}$ \\
    \hline
    50 & 1:40:37.5 & 0:52:43.7 & 0:19:13.8 & 0:13:44.0 \\
    74 & 2:22:30.0 & 1:15:00.6 & 0:25:21.3 & 0:15:49.8 \\
    100 & (3:07:51.9) & 1:40:10.5 & 0:30:44.4 & 0:20:01.4 \\
    200 & (6:02:20.6) & - & 1:00:32.0 & 0:30:35.0 \\
    473 & (13:58:40.3) & - & & 1:07:53.2 \\
    \end{tabular}}
    \caption{ Timing summaries for analyses for serial and multithreaded processes. M tasks is the number of functional-parallel tasks ran for the computation. $T_{p,c}$ is a single run time in hours:minutes:seconds for runs utilizing $p$ nodes and $c$ threads. Runs are run interactively on the same computer to maximize consistency. Empty entries are indicated with '-'. $(\cdot)$ entries are estimated entries extrapolated from data earlier in the column.}\label{tab:mtd_timing_study}
\end{table}